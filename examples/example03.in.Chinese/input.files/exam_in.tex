\documentclass{ctexart}
\markboth{ {\today}}{{\today}}
\pagestyle{myheadings}
\usepackage{amsmath}

\begin{document}




PTC-EXAM-TITLE
{\LARGE {\textbf{ 八道算术和两个选择题}}}
PTC-EXAM-TITLE-END




PTC-QUESTION(101;5.0;4;1)

请计算: 
\begin{equation}
PTC-III (520; 960; 3) + PTC-III (140; 480; 7) = \nonumber 
\end{equation}

PTC-ANSWER
   PTC-PRINT-OUT(1)
PTC-ANSWER-END

PTC-SOLUTION
$PTC-PRINT-IN(1) + PTC-PRINT-IN(2)=  PTC-PRINT-OUT(1)$
PTC-SOLUTION-END

PTC-QUESTION-END




PTC-QUESTION(102;5.0;4;1)

请计算: 
\begin{equation}
PTC-III (520; 960; 3) + PTC-III (140; 480; 7) = \nonumber 
\end{equation}

PTC-ANSWER
   PTC-PRINT-OUT(1)
PTC-ANSWER-END

PTC-SOLUTION
$PTC-PRINT-IN(1) + PTC-PRINT-IN(2)=  PTC-PRINT-OUT(1)$
PTC-SOLUTION-END

PTC-QUESTION-END




PTC-QUESTION(103;5.0;4;1)

请计算: 
\begin{equation}
PTC-III (520; 960; 3) -  PTC-III (140; 480; 7) = \nonumber 
\end{equation}

PTC-ANSWER
   PTC-PRINT-OUT(1)
PTC-ANSWER-END

PTC-SOLUTION
$PTC-PRINT-IN(1) - PTC-PRINT-IN(2)=  PTC-PRINT-OUT(1)$
PTC-SOLUTION-END

PTC-QUESTION-END




PTC-QUESTION(104;5.0;4;1)

请计算: 
\begin{equation}
PTC-III (520; 960; 3) -  PTC-III (140; 480; 7) = \nonumber 
\end{equation}

PTC-ANSWER
   PTC-PRINT-OUT(1)
PTC-ANSWER-END

PTC-SOLUTION
$PTC-PRINT-IN(1) - PTC-PRINT-IN(2)=  PTC-PRINT-OUT(1)$
PTC-SOLUTION-END

PTC-QUESTION-END




PTC-QUESTION(105;5.0;4;1)

请计算: 
\begin{equation}
PTC-III (520; 960; 3)  \times   PTC-III (140; 480; 7) = \nonumber 
\end{equation}

PTC-ANSWER
   PTC-PRINT-OUT(1)
PTC-ANSWER-END

PTC-SOLUTION
$PTC-PRINT-IN(1) \times PTC-PRINT-IN(2)=  PTC-PRINT-OUT(1)$
PTC-SOLUTION-END

PTC-QUESTION-END




PTC-QUESTION(106;5.0;4;1)

请计算: 
\begin{equation}
PTC-III (520; 960; 3)  \times   PTC-III (140; 480; 7) = \nonumber 
\end{equation}

PTC-ANSWER
   PTC-PRINT-OUT(1)
PTC-ANSWER-END

PTC-SOLUTION
$PTC-PRINT-IN(1) \times PTC-PRINT-IN(2)=  PTC-PRINT-OUT(1)$
PTC-SOLUTION-END

PTC-QUESTION-END




PTC-QUESTION(107;5.0;4;1)

请计算: 
\begin{equation}
PTC-III (520; 960; 3)  \div   PTC-III (140; 480; 7) = \nonumber 
\end{equation}

PTC-ANSWER
   PTC-PRINT-OUT(1;10)
PTC-ANSWER-END

PTC-SOLUTION
$PTC-PRINT-IN(1) \div PTC-PRINT-IN(2)=  PTC-PRINT-OUT(1;10)$
PTC-SOLUTION-END

PTC-QUESTION-END




PTC-QUESTION(108;5.0;4;1)

请计算: 
\begin{equation}
\frac {PTC-III (520; 960; 3) }  { PTC-III (140; 480; 7)} = \nonumber 
\end{equation}

PTC-ANSWER
   PTC-PRINT-OUT(1;10)
PTC-ANSWER-END

PTC-SOLUTION
$PTC-PRINT-IN(1) \div PTC-PRINT-IN(2)=  PTC-PRINT-OUT(1;10)$
PTC-SOLUTION-END

PTC-QUESTION-END










PTC-QUESTION  (201;5.0;4;1)
请从下面的陈述中选择正确的。
PTC-MULTICHOICE-USER(上述都不是。)
   PTC-CHOICE (1) 加拿大的首都是渥太华。
   PTC-CHOICE (0) 加拿大的首都是多伦多。
   PTC-CHOICE (0) 加拿大的首都是蒙特利尔。
   PTC-CHOICE (0) 加拿大的首都是魁北克市。
   PTC-CHOICE (0) 加拿大的首都是卡尔加里。
   PTC-CHOICE (0) 加拿大的首都是温哥华。
PTC-MULTICHOICE-USER-END
PTC-QUESTION-END



PTC-QUESTION  (202;5.0;4;1)
科学巨匠牛顿是
PTC-MULTICHOICE-USER(上述都不是。)
   PTC-CHOICE (1) 英国人。
   PTC-CHOICE (0) 加拿大人。
   PTC-CHOICE (0) 法国人。
   PTC-CHOICE (0) 美国人。
   PTC-CHOICE (0) 意大利人。
   PTC-CHOICE (0) 以色列人。
PTC-MULTICHOICE-USER-END
PTC-QUESTION-END



PTC-EXAM-TAIL

PTC-EXAM-TAIL-END




\end{document}
