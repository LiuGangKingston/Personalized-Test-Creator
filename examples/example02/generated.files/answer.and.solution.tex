\documentclass[12pt]{article}
\markboth{ {\today}}{{\today}}
\pagestyle{myheadings}
\usepackage{amsmath}
\usepackage{epsfig}
 
\begin{document}
 
 
   
   
\newpage 
\setcounter{page}{ 
    26001 } 
   
   
\noindent{\textbf{\large{THIS IS THE ANSWER AND SOLUTION FOR}}}
   
   
 {\textbf{ \Large{ PAPER NUMBER          26 }}}
   
   
\vspace{0.2in}
   
   
\markboth{Answer and solution NOT for examinees !!!{\today}}{Answer and solution NOT for examinees !!! {\today}}
   
   
   
   
 \vspace{0.2in}
 
 
{\Huge  THIS IS AN EXAMPLE OF}
 
{\Huge  PERSONALIZED TESTS. }
 
If needed, please use the following constants.
 
 
 
\noindent\begin{tabular}{|l|l|l|}
\hline
Constant & Symbol & Value \\
\hline
Acceleration due to earth's gravity &
$g$ &
 $ 9.80 $
m/s$^2$ \\
\hline
Avogadro's number &
$N_A$ &
 $ 6.0221367 \times 10^{23} $
mol$^{-1}$ \\
\hline
Boltzmann's constant &
$k$ &
 $ 1.380658 \times 10^{-23} $
J/K \\
\hline
Coulomb's constant &
$k$ &
 $ 8.99 \times 10^{9} $
N$\cdot $m$^2$/C$^2$ \\
\hline
Electron charge magnitiude &
$e$ &
 $ 1.60217733 \times 10^{-19} $
C \\
\hline
Permeability of free space &
$\mu _0$ &
 $ 1.25663706 \times 10^{-6} $
T$\cdot $m/A \\
\hline
Permittivity of free space &
$\epsilon _0$ &
 $ 8.854187817 \times 10^{-12} $
C$^2$/(N$\cdot $m$^2$) \\
\hline
Pi &
$\pi$ &
 $ 3.14159265 $
$ $ \\
\hline
Planck's constant &
$h$ &
 $ 6.6260755 \times 10^{-34} $
J$\cdot $s \\
\hline
Mass of electron &
$m_e$ &
 $ 9.1093897 \times 10^{-31} $
kg \\
\hline
\end{tabular}
 
 
\noindent\begin{tabular}{|l|l|l|}
\hline
Constant & Symbol & Value \\
\hline
Mass of neutron &
$m_n$ &
 $ 1.6749286 \times 10^{-27} $
kg \\
\hline
Mass of proton &
$m_p$ &
 $ 1.6726231 \times 10^{-27} $
kg \\
\hline
Speed of light in vacuum &
$c$ &
 $ 299792458. $
m/s \\
\hline
Universal gravitational constant &
$G$ &
 $ 6.67259 \times 10^{-11} $
N$\cdot $m$^2$/kg$^2$ \\
\hline
Universal gas constant &
$R$ &
 $ 8.314510 $
J/(mol$\cdot $K) \\
\hline
\end{tabular}
 
 
{\textbf{\large{Please be advised}}} that in this paper there are questions from
26.1 through
26.9.
And any one of them may contain more than one sub-question, thus the total number
of sub-questions here is around 14, of which
13 should be answered.
 
\vspace{0.3in}
 
 
   
   
   
\vspace{0.2in}
   
In this paper, big questions will be generated in the following order: 
   
   
            1(          6)
 ,
            2(          1)
 ,
            3(          2)
 ,
            4(          3)
 ,
            5(          5)
 ,
            6(          4)
 ,
            7(          7)
 ,
            8(          8)
 ,
            9(          9)
 .
  
\vspace{0.2in}
  
{\textbf{\Large{QUESTION
26.1 
 (          6)
}}}
  
  
 
{\textbf{\Large{Please answer ONLY
5 of the following
6 questions (Questions
26.1.1 through
26.1.6). }}}
 
Here are still some constants for use in the following questions:
 
 
\noindent\begin{tabular}{|l|l|l|}
\hline
Constant & Symbol & Value \\
\hline
 
Boltzmann's constant &
$k$ &
 $ 1.381 \times 10^{-23} $
J/K \\
\hline
 
Avogadro's number &
$N_A$ &
 $ 6.022 \times 10^{23} $
mol$^{-1}$ \\
\hline
 
Mass of electron &
$m_e$ &
 $ 9.1093897 \times 10^{-31} $
kg \\
\hline
 
\end{tabular}
 
   
\vspace{0.2in}
   
 In this big question of CHOOSE structure,           6 questions will be generat
 ed: 
  
  
            1(         11,         26)
 ,
            2(          6,         21)
 ,
            3(          9,         24)
 ,
            4(         13,         28)
 ,
            5(         12,         27)
 ,
            6(         10,         25)
 .
  
\vspace{0.2in}
  
{\textbf{\Large{Question
26.1.1 
 (          6,         11,         26)
}}}
  
  
 
 
\noindent\vspace{0.1in}{\textbf{\Large{Solution: }}}

Since the possiblity of  % 
smoking customer is $ a =  % 
.540 $,
and the possiblity of  % 
equal or above 30 years old customer is $ b =  % 
.6600 $,
the possiblity of  % 
non-smoking customer is $ c = 1.0 - a = 1.0 -
.540
=  % 
.460 $ and the possiblity of  % 
under 30 years old
customer is $ d = 1.0 - b = 1.0 -  % 
.6600 =  % 
.3400  $.
So the possibility of  % 
 non-smoking and  % 
under 30 years old
customer is $ c \times d =  % 
.156 $.
 
 
 
 
 
\noindent\vspace{0.05in}{\textbf{\Large{Answer:}}}

The possibility of  % 
 non-smoking and  % 
under 30 years old
customer is $ (1-a)(1-b) =  % 
.156 $.
 
 
  
\vspace{0.2in}
  
{\textbf{\Large{Question
26.1.2 
 (          6,          6,         21)
}}}
  
  
 
 
\noindent\vspace{0.05in}{\textbf{\Large{Answer:}}}

We will use the Newton's Second Law:
 
\[
\mathbf{f}=m\mathbf{a}.
\]
 
Since $\mathbf{f}=( % 
70.0,  % 
2.0,  % 
-2000.0 )N$
and $m= % 
50.0 kg$, bring them into the above equation, then we get
 
\begin{eqnarray*}
\mathbf{a}&=&\frac{\mathbf{f}}m  \\
&=&\frac{(
70.0 ,
2.0 ,
-2000.0 )N
}{ % 
50.0 kg}  \\
&=&(
1.4000 ,
4.0000 \times 10^{-2},
-40.000
)ms^{-2} \\
&=&(
18144. ,
518.40 ,
-518400.
)km/h^2.
\end{eqnarray*}
 
 
 
 
 
\noindent\vspace{0.1in}{\textbf{\Large{Solution: }}}

We will use the Newton's Second Law:
 
\[
\mathbf{f}=m\mathbf{a}.
\]
 
Since $\mathbf{f}=( % 
70.0,  % 
2.0,  % 
-2000.0 )N$
and $m= % 
50.0 kg$, bring them into the above equation, then we get
 
\begin{eqnarray*}
\mathbf{a}&=&\frac{\mathbf{f}}m  \\
&=&\frac{(
70.0 ,
2.0 ,
-2000.0 )N
}{ % 
50.0 kg}  \\
&=&(
1.4000 ,
4.0000 \times 10^{-2},
-40.000
)ms^{-2} \\
&=&(
18144. ,
518.40 ,
-518400.
)km/h^2.
\end{eqnarray*}
 
 
 
  
\vspace{0.2in}
  
{\textbf{\Large{Question
26.1.3 
 (          6,          9,         24)
}}}
  
  
 
 
\noindent\vspace{0.1in}{\textbf{\Large{Solution: }}}

By using Newton's Law of Universal Gravitation:
\[
F=G \frac{(Sun's \hspace{0.1in} mass) \times (Planet's \hspace{0.1in} mass)} { (distance)^2},
\]
where
$ G= % 
6.67 \times 10^{-11}N m^{2}(kg)^{-2}$ , the forces can be easily calculated as
 
\vspace{0.2in}
 
 
\begin{tabular}{|l|l|l|l|}
\hline
The Planet & Mass ($kg$) & Distanace from Sun ($m$) & The Force ($N$)\\
\hline
Mercury  &
           $ % 
6.00000000 \times 10^{24} $   &
             $ % 
6.000000000 \times 10^{24} $    & $ % 
3.33 \times 10^{-11} $
\\  \hline
Venus    &
           $  % 
2.00 \times 10^{24}  $     &
             $ % 
4.00 \times 10^{24} $    & $ % 
2.50 \times 10^{-11} $
\\  \hline
Earth    &
           $  % 
8.00 \times 10^{24}  $     &
             $ % 
4.00 \times 10^{24} $    & $ % 
1.00 \times 10^{-10} $
\\   \hline
Mars     &
           $  % 
7.00 \times 10^{24} $     &
             $ % 
9.00 \times 10^{24} $    & $ % 
1.73 \times 10^{-11} $
\\   \hline
Jupiter  &
           $  % 
4.00 \times 10^{24} $    &
             $ % 
7.00 \times 10^{24} $    & $ % 
1.63 \times 10^{-11} $
\\  \hline
Saturn   &
           $  % 
5.00 \times 10^{24} $    &
             $ % 
8.00 \times 10^{24}  $    & $ % 
1.56 \times 10^{-11} $
\\  \hline
Uranus   &
           $  % 
3.00 \times 10^{24} $    &
             $ % 
8.00 \times 10^{24} $    & $ % 
9.38 \times 10^{-12} $
\\  \hline
Neptune  &
           $  % 
9.00 \times 10^{24} $    &
             $ % 
4.00 \times 10^{24} $    & $ % 
1.13 \times 10^{-10} $
\\  \hline
 
\end{tabular}
 
 
 
 
 
 
\noindent\vspace{0.05in}{\textbf{\Large{Answer:}}}

By using Newton's Law of Universal Gravitation:
\[
F=G \frac{(Sun's \hspace{0.1in} mass) \times (Planet's \hspace{0.1in} mass)} { (distance)^2},
\]
where
$ G= % 
6.67 \times 10^{-11} N m^{2}(kg)^{-2}$ , the forces can be easily calculated as
 
\vspace{0.2in}
 
 
\begin{tabular}{|l|l|l|l|}
\hline
The Planet & Mass ($kg$) & Distanace from Sun ($m$) & The Force ($N$)\\
\hline
Mercury  &
           $ % 
6.00000000 \times 10^{24}  $   &
             $ % 
6.000000000 \times 10^{24}$    & $ % 
3.33 \times 10^{-11} $
\\  \hline
Venus    &
           $  % 
2.00 \times 10^{24}  $     &
             $ % 
4.00 \times 10^{24} $    & $ % 
2.50 \times 10^{-11} $
\\  \hline
Earth    &
           $  % 
8.00 \times 10^{24}$     &
             $ % 
4.00 \times 10^{24} $    & $ % 
1.00 \times 10^{-10} $
\\   \hline
Mars     &
           $  % 
7.00 \times 10^{24} $     &
             $ % 
9.00 \times 10^{24}$    & $ % 
1.73 \times 10^{-11} $
\\   \hline
Jupiter  &
           $  % 
4.00 \times 10^{24}  $    &
             $ % 
7.00 \times 10^{24} $    & $ % 
1.63 \times 10^{-11}3 $
\\  \hline
Saturn   &
           $  % 
5.00 \times 10^{24}   $    &
             $ % 
8.00 \times 10^{24}  $    & $ % 
1.56 \times 10^{-11} $
\\  \hline
Uranus   &
           $  % 
3.00 \times 10^{24} $    &
             $ % 
8.00 \times 10^{24}$    & $ % 
9.38 \times 10^{-12} $
\\  \hline
Neptune  &
           $  % 
9.00 \times 10^{24}  $    &
             $ % 
4.00 \times 10^{24} $    & $ % 
1.13 \times 10^{-10} $
\\  \hline
 
\end{tabular}
 
 
 
 
  
\vspace{0.2in}
  
{\textbf{\Large{Question
26.1.4 
 (          6,         13,         28)
}}}
  
  
 
 
\noindent\vspace{0.05in}{\textbf{\Large{Answer:}}}

5;
 
6;
 
The operation is  % 
SUBTRACTION and the result is
$ % 
-1.0000$.
 
 
 
  
\vspace{0.2in}
  
{\textbf{\Large{Question
26.1.5 
 (          6,         12,         27)
}}}
  
  
 
 
\noindent\vspace{0.1in}{\textbf{\Large{Solution: }}}

Since the possiblity of  % 
 non-smoking customer is $ a =  % 
.660 $,
and the possiblity of  % 
equal-or-above 30 years old customer is $ b =  % 
.3000 $,
the possiblity of  % 
smoking customer is $ c = 1.0 - a = 1.0 -
.660
=  % 
.340 $ and the possiblity of  % 
under 30 years old
customer is $ d = 1.0 - b = 1.0 -  % 
.3000 =  % 
.7000  $.
Then
 
\noindent
\begin{tabular}{|l|l|}
\hline
Customer & Possibility \\
\hline
smoking  and  % 
equal-or-above 30 years old  &
  $ % 
.340 \times  % 
.3000 =  % 
.102$ \\
\hline
smoking  and  % 
under 30 years old &
  $ % 
.340 \times  % 
.7000 =  % 
.238$ \\
\hline
 non-smoking and  % 
equal-or-above 30 years old  &
  $ % 
.660 \times  % 
.3000 =  % 
.198$ \\
\hline
 non-smoking and  % 
under 30 years old &
  $ % 
.660 \times  % 
.7000 =  % 
.462$ \\
\hline
\end{tabular}
 
\noindent
And the total summation of all possibilities is $  % 
1.000 $.
 
 
 
 
 
 
\noindent\vspace{0.05in}{\textbf{\Large{Answer:}}}

 
\noindent
\begin{tabular}{|l|l|}
\hline
Customer & Possibility \\
\hline
smoking  and  % 
equal-or-above 30 years old &
  $ % 
.102$ \\
\hline
smoking  and  % 
under 30 years old &
  $ % 
.238$ \\
\hline
 non-smoking and  % 
equal-or-above 30 years old &
  $ % 
.198$ \\
\hline
 non-smoking and  % 
under 30 years old &
  $ % 
.462$ \\
\hline
\end{tabular}
 
\noindent
 And the total summation of all possibilities is $  % 
1.000 $.
 
 
 
  
\vspace{0.2in}
  
{\textbf{\Large{Question
26.1.6 
 (          6,         10,         25)
}}}
  
  
 
 
\noindent\vspace{0.05in}{\textbf{\Large{Auto-answer:}}}
 
 
\noindent{\textbf{\large{
C.}}}
A truck
 
 
\noindent{\textbf{\large{
D.}}}
An airplane
 
 
 
 
   
   
\vspace{0.3in}
{\textbf{\LARGE{You have done all the above? A very good beginning, please go ahead.}}}
More constants the
Mass of electron
$m_e$$ =
9.109390 \times 10^{-31} $
kg
,
Universal gas constant
$R$$ =
8.315 $
J/(mol$\cdot $K)
,
$e$$ =
1.60217733 \times 10^{-19} $
C
, and
$m_p$$ =
1.6726231 \times 10^{-27} $
kg
%
may be very helpful.
\vspace{0.3in}
   
   
  
\vspace{0.2in}
  
{\textbf{\Large{QUESTION
26.2 
 (          1,          1,          1)
}}}
  
  


 
 
\noindent\vspace{0.05in}{\textbf{\Large{Auto-answer:}}}
 
 
\noindent{\textbf{\large{
C.}}}
The accelaration is $  %
(
1.80,
.18,
-160.00)
ms^{-2} $.
 
 
 
 
 
 
\noindent\vspace{0.05in}{\textbf{\Large{Answer:}}}

The correct answer from the choices is


\noindent{\textbf{\large{
C.}}}
The accelaration is $  %
(
1.80,
.18,
-160.00)
ms^{-2} $.
 
 
 
 
 
\noindent\vspace{0.1in}{\textbf{\Large{Solution: }}}

We will use the Newton's Second Law:
 
\[
\mathbf{f}=m\mathbf{a}.
\]
 
Since $\mathbf{f}= % 
(90.0 , 9.0 , -8000.0) N$
and $m= % 
50.0000kg$, bring them into the above equation, then we get
 
\begin{eqnarray*}
\mathbf{a}&=&\frac{\mathbf{f}}m  \\
&=&\frac{ % 
(90.0 , 9.0 , -8000.0) N}{ % 
50.0000kg}  \\
&=& % 
(1.80 , .18 , -160.00) ms^{-2}
\end{eqnarray*}
 
 
 
  
\vspace{0.2in}
  
{\textbf{\Large{QUESTION
26.3 
 (          2,          2,          2)
}}}
  
  
 
 
\noindent\vspace{0.05in}{\textbf{\Large{Auto-answer:}}}
 
 
\noindent{\textbf{\large{
A.}}}
The accelaration is
$(
1.3793ms^{-2},
1117.2km/h^2,
-155.17ms^{-2}
).
$
 
 
 
 
 
 
\noindent\vspace{0.1in}{\textbf{\Large{Solution: }}}

We will use the Newton's Second Law:
 
\[
\mathbf{f}=m\mathbf{a}.
\]
 
Since $\mathbf{f}=( % 
80.000,  % 
5.0000,  % 
-9000.0 )N$
and $m= % 
58.0000kg$, bring them into the above equation, then we get
 
\begin{eqnarray*}
\mathbf{a}&=&\frac{\mathbf{f}}m  \\
&=&\frac{(
80.000 ,
5.0000 ,
-9000.0 )N
}{ % 
58.0000 kg}  \\
&=&(
1.3793 ,
8.6207 \times 10^{-2},
-155.17
)ms^{-2} \\
&=&(
17876. ,
1117.2 ,
-2.0110 \times 10^{6}
)km/h^2.
\end{eqnarray*}
 
 
 
  
\vspace{0.2in}
  
{\textbf{\Large{QUESTION
26.4 
 (          3,          3,          3)
}}}
  
  
 
 
\noindent\vspace{0.05in}{\textbf{\Large{Auto-answer:}}}
 
 
\noindent{\textbf{\large{
F.}}}
 None of above.
 
 
 
 
  
\vspace{0.2in}
  
{\textbf{\Large{QUESTION
26.5 
 (          5,          5,          5)
}}}
  
  
 
 
\noindent\vspace{0.05in}{\textbf{\Large{Answer:}}}

 
\noindent\begin{tabular}{|l|l|}\hline The correct & \\
          answer &  % 
$F$ \\ \hline \end{tabular}
1. $ % 
78$ is an  % 
odd number.
 
\noindent\begin{tabular}{|l|l|}\hline The correct & \\
          answer &  % 
$T$ \\ \hline \end{tabular}
2.  % 
Toronto is in  % 
Ontario province.
 
\noindent\begin{tabular}{|l|l|}\hline The correct & \\
          answer &  % 
$F$ \\ \hline \end{tabular}
3.  % 
$\mathbf{F}=m\mathbf{a}$ is a mathmatical form of  % 
Newton's Law of Universal Gravitation.
 
 
 
  
\vspace{0.2in}
  
{\textbf{\Large{QUESTION
26.6 
 (          4,          4,          4)
}}}
  
  
 
 
\noindent\vspace{0.05in}{\textbf{\Large{Auto-answer:}}}
  
  
\begin{tabular}{|l|l|l|}
 \hline
 Column Left & Column Right  & Answers       \\ 
 \hline
{\textbf{\large{
A.}}}
er
  & 
ASDF(:)
 & 
{\textbf{\large{
D.}}}
 \\ 
 \hline
{\textbf{\large{
B.}}}
Er
  & 
b
 & 
{\textbf{\large{
C.}}}
 \\ 
 \hline
{\textbf{\large{
C.}}}
B
  & 
eR
 & 
{\textbf{\large{
A.}}}
, 
{\textbf{\large{
B.}}}
 \\ 
 \hline
{\textbf{\large{
D.}}}
asdf(:)
  & 
a
 & 
{\textbf{\large{
E.}}}
 \\ 
 \hline
{\textbf{\large{
E.}}}
A
  & 
ER
 & 
{\textbf{\large{
A.}}}
, 
{\textbf{\large{
B.}}}
 \\ 
 \hline
 \end{tabular}
  
  
 
 
 
 
   
   
\vspace{0.3in}
{\textbf{\LARGE{You have done all the above? Excellent! Not much left, please continue.}}}
\vspace{0.3in}
   
   
  
\vspace{0.2in}
  
{\textbf{\Large{QUESTION
26.7 
 (          7,         14,         50)
}}}
  
  
 
 
\noindent\vspace{0.05in}{\textbf{\Large{Auto-answer:}}}
 
 
\noindent{\textbf{\large{
A.}}}
  The accelaration is $  %
(
1.55,
.12,
-120.69)
ms^{-2} $.
 
 
 
 
 
 
\noindent\vspace{0.1in}{\textbf{\Large{Solution: }}}

We will use the Newton's Second Law:
 
\[
\mathbf{f}=m\mathbf{a}.
\]
 
Since $\mathbf{f}= % 
(90.0 , 7.0 , -7000.0) N$
and $m= % 
58.0kg$, bring them into the above equation, then we get
 
\begin{eqnarray*}
\mathbf{a}&=&\frac{\mathbf{f}}m  \\
&=&\frac{ % 
(90.0 , 7.0 , -7000.0) N}{ % 
58.0kg}  \\
&=& % 
(1.55 , .12 , -120.69) ms^{-2}
\end{eqnarray*}
 
 
 
  
\vspace{0.2in}
  
{\textbf{\Large{QUESTION
26.8 
 (          8,         15,         60)
}}}
  
  
 
 
\noindent\vspace{0.05in}{\textbf{\Large{Answer:}}}

 
$\left( \begin{array}{ccccccccccccccc}
           4 & 
           7 & 
           5 & 
           6 \\ 
           6 & 
           6 & 
           7 & 
           5 \\ 
           4 & 
           4 & 
           4 & 
           4
\end{array}\right) \times
\left( \begin{array}{c}
           2 \\ 
           2 \\ 
           2 \\ 
           2
\end{array}\right)  =
\left( \begin{array}{c}
          44 \\ 
          48 \\ 
          32
\end{array}\right)  $
 
$  % 
 \left( \begin{array}
 {
 c
 c
 }
 \varepsilon & 
 \rho \\ 
 \sigma & 
 \beta \\ 
 \Lambda & 
 \Delta \\ 
 \Omega & 
                    \Xi
 \end{array} \right)
 \left( \begin{array}
 {
 c
 }
 \gamma \\ 
 \gamma
 \end{array} \right)
=
  \left( \begin{array}
 {
 c
 }
 \varepsilon \times  \gamma   +  \rho \times  \gamma \\ 
 \sigma \times  \gamma   +  \beta \times  \gamma \\ 
 \Lambda \times  \gamma   +  \Delta \times  \gamma \\ 
 \Omega \times  \gamma   +                     \Xi \times  \gamma
 \end{array} \right)
$
 
 
 
 
 
\noindent\vspace{0.1in}{\textbf{\Large{Solution: }}}

 
 
  
\vspace{0.2in}
  
{\textbf{\Large{QUESTION
26.9 
 (          9,         16,         70)
}}}
  
  


 
 
\noindent\vspace{0.05in}{\textbf{\Large{Answer:}}}

-7,  % 
11
 
 
 
 
 
\noindent\vspace{0.1in}{\textbf{\Large{Solution: }}}

Roots to the equation
\begin{eqnarray*}
7 \times x^2  % 
-28
                 \times x    % 
-539 =0
\end{eqnarray*}
are  % 
-7 and  % 
11 .
 
Let us verity  % 
-7 first:
$  % 
7 \times x^2  % 
-28
                 \times x    % 
-539
  = % 
343+( % 
196)+( % 
-539)
  = % 
539+( % 
-539)
  = % 
0
$
 
Then verity  % 
11:
$  % 
7 \times x^2  % 
-28
                 \times x    % 
-539
  = % 
847+( % 
-308)+( % 
-539)
  = % 
539+( % 
-539)
  = % 
0
$
 
 
 
   
   
 \vspace{0.2in}
Here are still some constants for use:
 
 
\noindent\begin{tabular}{|l|l|l|}
\hline
Constant & Symbol & Value \\
\hline
 
Mass of proton &
$m_p$ &
 $ 1.6726231 \times 10^{-27} $
kg \\
\hline
 
Boltzmann's constant &
$k$ &
 $ 1.381 \times 10^{-23} $
J/K \\
\hline
 
\end{tabular}
 
Thank you very much for answering these questions!
 
{\textbf{\large{Please be advised}}} that in this paper there are questions from
26.1 through
26.9.
And any one of them may contain more than one sub-question, thus the total number
of sub-questions here is around 14, of which
13 should be answered.
 
   
   
   
   
\vspace{1.0in} 
{\textbf{\large{ *** END OF PAPER, THANKS *** }}} 
   
   
\hspace{1.0in} By: 
         239(         26,          34)
   
   
   
   
\newpage 
\setcounter{page}{ 
    27001 } 
   
   
\noindent{\textbf{\large{THIS IS THE ANSWER AND SOLUTION FOR}}}
   
   
 {\textbf{ \Large{ PAPER NUMBER          27 }}}
   
   
\vspace{0.2in}
   
   
\markboth{Answer and solution NOT for examinees !!!{\today}}{Answer and solution NOT for examinees !!! {\today}}
   
   
   
   
 \vspace{0.2in}
 
 
{\Huge  THIS IS AN EXAMPLE OF}
 
{\Huge  PERSONALIZED TESTS. }
 
If needed, please use the following constants.
 
 
 
\noindent\begin{tabular}{|l|l|l|}
\hline
Constant & Symbol & Value \\
\hline
Acceleration due to earth's gravity &
$g$ &
 $ 9.80 $
m/s$^2$ \\
\hline
Avogadro's number &
$N_A$ &
 $ 6.0221367 \times 10^{23} $
mol$^{-1}$ \\
\hline
Boltzmann's constant &
$k$ &
 $ 1.380658 \times 10^{-23} $
J/K \\
\hline
Coulomb's constant &
$k$ &
 $ 8.99 \times 10^{9} $
N$\cdot $m$^2$/C$^2$ \\
\hline
Electron charge magnitiude &
$e$ &
 $ 1.60217733 \times 10^{-19} $
C \\
\hline
Permeability of free space &
$\mu _0$ &
 $ 1.25663706 \times 10^{-6} $
T$\cdot $m/A \\
\hline
Permittivity of free space &
$\epsilon _0$ &
 $ 8.854187817 \times 10^{-12} $
C$^2$/(N$\cdot $m$^2$) \\
\hline
Pi &
$\pi$ &
 $ 3.14159265 $
$ $ \\
\hline
Planck's constant &
$h$ &
 $ 6.6260755 \times 10^{-34} $
J$\cdot $s \\
\hline
Mass of electron &
$m_e$ &
 $ 9.1093897 \times 10^{-31} $
kg \\
\hline
\end{tabular}
 
 
\noindent\begin{tabular}{|l|l|l|}
\hline
Constant & Symbol & Value \\
\hline
Mass of neutron &
$m_n$ &
 $ 1.6749286 \times 10^{-27} $
kg \\
\hline
Mass of proton &
$m_p$ &
 $ 1.6726231 \times 10^{-27} $
kg \\
\hline
Speed of light in vacuum &
$c$ &
 $ 299792458. $
m/s \\
\hline
Universal gravitational constant &
$G$ &
 $ 6.67259 \times 10^{-11} $
N$\cdot $m$^2$/kg$^2$ \\
\hline
Universal gas constant &
$R$ &
 $ 8.314510 $
J/(mol$\cdot $K) \\
\hline
\end{tabular}
 
 
{\textbf{\large{Please be advised}}} that in this paper there are questions from
27.1 through
27.9.
And any one of them may contain more than one sub-question, thus the total number
of sub-questions here is around 14, of which
13 should be answered.
 
\vspace{0.3in}
 
 
   
   
   
\vspace{0.2in}
   
In this paper, big questions will be generated in the following order: 
   
   
            1(          6)
 ,
            2(          4)
 ,
            3(          3)
 ,
            4(          2)
 ,
            5(          1)
 ,
            6(          5)
 ,
            7(          8)
 ,
            8(          7)
 ,
            9(          9)
 .
  
\vspace{0.2in}
  
{\textbf{\Large{QUESTION
27.1 
 (          6)
}}}
  
  
 
{\textbf{\Large{Please answer ONLY
5 of the following
6 questions (Questions
27.1.1 through
27.1.6). }}}
 
Here are still some constants for use in the following questions:
 
 
\noindent\begin{tabular}{|l|l|l|}
\hline
Constant & Symbol & Value \\
\hline
 
Boltzmann's constant &
$k$ &
 $ 1.381 \times 10^{-23} $
J/K \\
\hline
 
Avogadro's number &
$N_A$ &
 $ 6.022 \times 10^{23} $
mol$^{-1}$ \\
\hline
 
Mass of electron &
$m_e$ &
 $ 9.1093897 \times 10^{-31} $
kg \\
\hline
 
\end{tabular}
 
   
\vspace{0.2in}
   
 In this big question of CHOOSE structure,           6 questions will be generat
 ed: 
  
  
            1(          8,         23)
 ,
            2(         10,         25)
 ,
            3(          6,         21)
 ,
            4(         11,         26)
 ,
            5(         13,         28)
 ,
            6(          7,         22)
 .
  
\vspace{0.2in}
  
{\textbf{\Large{Question
27.1.1 
 (          6,          8,         23)
}}}
  
  
 
 
\noindent\vspace{0.05in}{\textbf{\Large{Auto-answer:}}}
 
 
\noindent{\textbf{\large{
E.}}}
none of these.
 
 
 
 
 
 
\noindent\vspace{0.1in}{\textbf{\Large{Solution: }}}

We will use the Newton's Second Law:
 
\[
\mathbf{f}=m\mathbf{a}.
\]
 
Since $\mathbf{f}=( % 
90.0,  % 
6.0,  % 
-3000.0 )N$
and $m= % 
52.0kg$, bring them into the above equation, then we get
 
\begin{eqnarray*}
\mathbf{a}&=&\frac{\mathbf{f}}m  \\
&=&\frac{(
90.0 ,
6.0 ,
-3000.0 )N
}{ % 
52.0 kg}  \\
&=&(
1.7308 ,
.11538,
-57.692
)ms^{-2} \\
&=&(
22431. ,
1495.4 ,
-747692.
)km/h^2.
\end{eqnarray*}
 
 
 
  
\vspace{0.2in}
  
{\textbf{\Large{Question
27.1.2 
 (          6,         10,         25)
}}}
  
  
 
 
\noindent\vspace{0.05in}{\textbf{\Large{Auto-answer:}}}
 
 
\noindent{\textbf{\large{
C.}}}
A truck
 
 
\noindent{\textbf{\large{
D.}}}
An airplane
 
 
 
 
  
\vspace{0.2in}
  
{\textbf{\Large{Question
27.1.3 
 (          6,          6,         21)
}}}
  
  
 
 
\noindent\vspace{0.05in}{\textbf{\Large{Answer:}}}

We will use the Newton's Second Law:
 
\[
\mathbf{f}=m\mathbf{a}.
\]
 
Since $\mathbf{f}=( % 
50.0,  % 
5.0,  % 
-5000.0 )N$
and $m= % 
50.0 kg$, bring them into the above equation, then we get
 
\begin{eqnarray*}
\mathbf{a}&=&\frac{\mathbf{f}}m  \\
&=&\frac{(
50.0 ,
5.0 ,
-5000.0 )N
}{ % 
50.0 kg}  \\
&=&(
1.0000 ,
.10000,
-100.00
)ms^{-2} \\
&=&(
12960. ,
1296.0 ,
-1.2960 \times 10^{6}
)km/h^2.
\end{eqnarray*}
 
 
 
 
 
\noindent\vspace{0.1in}{\textbf{\Large{Solution: }}}

We will use the Newton's Second Law:
 
\[
\mathbf{f}=m\mathbf{a}.
\]
 
Since $\mathbf{f}=( % 
50.0,  % 
5.0,  % 
-5000.0 )N$
and $m= % 
50.0 kg$, bring them into the above equation, then we get
 
\begin{eqnarray*}
\mathbf{a}&=&\frac{\mathbf{f}}m  \\
&=&\frac{(
50.0 ,
5.0 ,
-5000.0 )N
}{ % 
50.0 kg}  \\
&=&(
1.0000 ,
.10000,
-100.00
)ms^{-2} \\
&=&(
12960. ,
1296.0 ,
-1.2960 \times 10^{6}
)km/h^2.
\end{eqnarray*}
 
 
 
  
\vspace{0.2in}
  
{\textbf{\Large{Question
27.1.4 
 (          6,         11,         26)
}}}
  
  
 
 
\noindent\vspace{0.1in}{\textbf{\Large{Solution: }}}

Since the possiblity of  % 
smoking customer is $ a =  % 
7.0 \times 10^{-2} $,
and the possiblity of  % 
equal or above 30 years old customer is $ b =  % 
.8200 $,
the possiblity of  % 
non-smoking customer is $ c = 1.0 - a = 1.0 -
7.0 \times 10^{-2}
=  % 
.930 $ and the possiblity of  % 
under 30 years old
customer is $ d = 1.0 - b = 1.0 -  % 
.8200 =  % 
.1800  $.
So the possibility of  % 
 non-smoking and  % 
under 30 years old
customer is $ c \times d =  % 
.167 $.
 
 
 
 
 
\noindent\vspace{0.05in}{\textbf{\Large{Answer:}}}

The possibility of  % 
 non-smoking and  % 
under 30 years old
customer is $ (1-a)(1-b) =  % 
.167 $.
 
 
  
\vspace{0.2in}
  
{\textbf{\Large{Question
27.1.5 
 (          6,         13,         28)
}}}
  
  
 
 
\noindent\vspace{0.05in}{\textbf{\Large{Answer:}}}

5;
 
4;
 
The operation is  % 
MULTIPLICATION and the result is
$ % 
20.000$.
 
 
 
  
\vspace{0.2in}
  
{\textbf{\Large{Question
27.1.6 
 (          6,          7,         22)
}}}
  
  
 
 
\noindent\vspace{0.05in}{\textbf{\Large{Auto-answer:}}}
 
 
\noindent{\textbf{\large{
I.}}}
The accelaration (vector) is
$(
7476.9,
747.69 ,
-747692.
)km/h^2.
$
 
 
 
 
 
 
\noindent\vspace{0.1in}{\textbf{\Large{Solution: }}}

We will use the Newton's Second Law:
 
\[
\mathbf{f}=m\mathbf{a}.
\]
 
Since $\mathbf{f}=( % 
30.0,  % 
3.0,  % 
-3000.0 )N$
and $m= % 
52.0 kg$, bring them into the above equation, then we get
 
\begin{eqnarray*}
\mathbf{a}&=&\frac{\mathbf{f}}m  \\
&=&\frac{(
30.0 ,
3.0 ,
-3000.0 )N
}{ % 
52.0 kg}  \\
&=&(
.57692 ,
5.7692 \times 10^{-2},
-57.692
)ms^{-2} \\
&=&(
7476.9 ,
747.69 ,
-747692.
)km/h^2.
\end{eqnarray*}
 
 
 
   
   
\vspace{0.3in}
{\textbf{\LARGE{You have done all the above? A very good beginning, please go ahead.}}}
More constants the
Mass of electron
$m_e$$ =
9.109390 \times 10^{-31} $
kg
,
Universal gas constant
$R$$ =
8.315 $
J/(mol$\cdot $K)
,
$e$$ =
1.60217733 \times 10^{-19} $
C
, and
$m_p$$ =
1.6726231 \times 10^{-27} $
kg
%
may be very helpful.
\vspace{0.3in}
   
   
  
\vspace{0.2in}
  
{\textbf{\Large{QUESTION
27.2 
 (          4,          4,          4)
}}}
  
  
 
 
\noindent\vspace{0.05in}{\textbf{\Large{Auto-answer:}}}
  
  
\begin{tabular}{|l|l|l|}
 \hline
 Column Left & Column Right  & Answers       \\ 
 \hline
{\textbf{\large{
A.}}}
er
  & 
b
 & 
{\textbf{\large{
C.}}}
 \\ 
 \hline
{\textbf{\large{
B.}}}
 A= %
6/ %
2

  & 
ER
 & 
{\textbf{\large{
A.}}}
 \\ 
 \hline
{\textbf{\large{
C.}}}
B
  & 
YJH
 & 
{\textbf{\large{
E.}}}
 \\ 
 \hline
{\textbf{\large{
D.}}}
asdf(:)
  & 
 a= %
3
 & 
{\textbf{\large{
B.}}}
 \\ 
 \hline
{\textbf{\large{
E.}}}
yjh
  & 
ASDF(:)
 & 
{\textbf{\large{
D.}}}
 \\ 
 \hline
 \end{tabular}
  
  
 
 
 
 
  
\vspace{0.2in}
  
{\textbf{\Large{QUESTION
27.3 
 (          3,          3,          3)
}}}
  
  
 
 
\noindent\vspace{0.05in}{\textbf{\Large{Auto-answer:}}}
 
 
\noindent{\textbf{\large{
A.}}}
Canada has  %
10 provinces and  %
3 territories.
 
 
 
 
  
\vspace{0.2in}
  
{\textbf{\Large{QUESTION
27.4 
 (          2,          2,          2)
}}}
  
  
 
 
\noindent\vspace{0.05in}{\textbf{\Large{Auto-answer:}}}
 
 
\noindent{\textbf{\large{
E.}}}
The accelaration is
$(
1.3793ms^{-2},
2011.0km/h^2,
-155.17ms^{-2}
).
$
 
 
 
 
 
 
\noindent\vspace{0.1in}{\textbf{\Large{Solution: }}}

We will use the Newton's Second Law:
 
\[
\mathbf{f}=m\mathbf{a}.
\]
 
Since $\mathbf{f}=( % 
80.000,  % 
9.0000,  % 
-9000.0 )N$
and $m= % 
58.0000kg$, bring them into the above equation, then we get
 
\begin{eqnarray*}
\mathbf{a}&=&\frac{\mathbf{f}}m  \\
&=&\frac{(
80.000 ,
9.0000 ,
-9000.0 )N
}{ % 
58.0000 kg}  \\
&=&(
1.3793 ,
.15517,
-155.17
)ms^{-2} \\
&=&(
17876. ,
2011.0 ,
-2.0110 \times 10^{6}
)km/h^2.
\end{eqnarray*}
 
 
 
  
\vspace{0.2in}
  
{\textbf{\Large{QUESTION
27.5 
 (          1,          1,          1)
}}}
  
  


 
 
\noindent\vspace{0.05in}{\textbf{\Large{Auto-answer:}}}
 
 
\noindent{\textbf{\large{
D.}}}
The accelaration is $  %
(
.769,
3.8 \times 10^{-2},
-38.462)
ms^{-2} $.
 
 
 
 
 
 
\noindent\vspace{0.05in}{\textbf{\Large{Answer:}}}

The correct answer from the choices is


\noindent{\textbf{\large{
D.}}}
The accelaration is $  %
(
.769,
3.8 \times 10^{-2},
-38.462)
ms^{-2} $.
 
 
 
 
 
\noindent\vspace{0.1in}{\textbf{\Large{Solution: }}}

We will use the Newton's Second Law:
 
\[
\mathbf{f}=m\mathbf{a}.
\]
 
Since $\mathbf{f}= % 
(40.0 , 2.0 , -2000.0) N$
and $m= % 
52.0000kg$, bring them into the above equation, then we get
 
\begin{eqnarray*}
\mathbf{a}&=&\frac{\mathbf{f}}m  \\
&=&\frac{ % 
(40.0 , 2.0 , -2000.0) N}{ % 
52.0000kg}  \\
&=& % 
(.769 , 3.8 \times 10^{-2} , -38.462) ms^{-2}
\end{eqnarray*}
 
 
 
  
\vspace{0.2in}
  
{\textbf{\Large{QUESTION
27.6 
 (          5,          5,          5)
}}}
  
  
 
 
\noindent\vspace{0.05in}{\textbf{\Large{Answer:}}}

 
\noindent\begin{tabular}{|l|l|}\hline The correct & \\
          answer &  % 
$F$ \\ \hline \end{tabular}
1. $ % 
47$ is an  % 
even number.
 
\noindent\begin{tabular}{|l|l|}\hline The correct & \\
          answer &  % 
$F$ \\ \hline \end{tabular}
2.  % 
Montreal is in  % 
Ontario province.
 
\noindent\begin{tabular}{|l|l|}\hline The correct & \\
          answer &  % 
$T$ \\ \hline \end{tabular}
3.  % 
$\mathbf{F}=m\mathbf{a}$ is a mathmatical form of  % 
the Newton's Second Law.
 
 
 
   
   
\vspace{0.3in}
{\textbf{\LARGE{You have done all the above? Excellent! Not much left, please continue.}}}
\vspace{0.3in}
   
   
  
\vspace{0.2in}
  
{\textbf{\Large{QUESTION
27.7 
 (          8,         15,         60)
}}}
  
  
 
 
\noindent\vspace{0.05in}{\textbf{\Large{Answer:}}}

 
$\left( \begin{array}{ccccccccccccccc}
           5 & 
           7 & 
           7 & 
           6 \\ 
           5 & 
           4 & 
           6 & 
           5 \\ 
           6 & 
           6 & 
           5 & 
           5
\end{array}\right) \times
\left( \begin{array}{c}
           2 \\ 
           2 \\ 
           2 \\ 
           2
\end{array}\right)  =
\left( \begin{array}{c}
          50 \\ 
          40 \\ 
          44
\end{array}\right)  $
 
$  % 
 \left( \begin{array}
 {
 c
 c
 }
                    \zeta & 
 \Theta \\ 
                    \Xi & 
 \Theta \\ 
 \eta & 
 \gamma \\ 
 \rho & 
 \delta
 \end{array} \right)
 \left( \begin{array}
 {
 c
 }
 \beta \\ 
 \beta
 \end{array} \right)
=
  \left( \begin{array}
 {
 c
 }
                    \zeta \times  \beta   +  \Theta \times  \beta \\ 
                    \Xi \times  \beta   +  \Theta \times  \beta \\ 
 \eta \times  \beta   +  \gamma \times  \beta \\ 
 \rho \times  \beta   +  \delta \times  \beta
 \end{array} \right)
$
 
 
 
 
 
\noindent\vspace{0.1in}{\textbf{\Large{Solution: }}}

 
 
  
\vspace{0.2in}
  
{\textbf{\Large{QUESTION
27.8 
 (          7,         14,         50)
}}}
  
  
 
 
\noindent\vspace{0.05in}{\textbf{\Large{Auto-answer:}}}
 
 
\noindent{\textbf{\large{
B.}}}
  The accelaration is $  %
(
1.38,
.14,
-137.93)
ms^{-2} $.
 
 
 
 
 
 
\noindent\vspace{0.1in}{\textbf{\Large{Solution: }}}

We will use the Newton's Second Law:
 
\[
\mathbf{f}=m\mathbf{a}.
\]
 
Since $\mathbf{f}= % 
(80.0 , 8.0 , -8000.0) N$
and $m= % 
58.0kg$, bring them into the above equation, then we get
 
\begin{eqnarray*}
\mathbf{a}&=&\frac{\mathbf{f}}m  \\
&=&\frac{ % 
(80.0 , 8.0 , -8000.0) N}{ % 
58.0kg}  \\
&=& % 
(1.38 , .14 , -137.93) ms^{-2}
\end{eqnarray*}
 
 
 
  
\vspace{0.2in}
  
{\textbf{\Large{QUESTION
27.9 
 (          9,         16,         70)
}}}
  
  


 
 
\noindent\vspace{0.05in}{\textbf{\Large{Answer:}}}

25,  % 
-13
 
 
 
 
 
\noindent\vspace{0.1in}{\textbf{\Large{Solution: }}}

Roots to the equation
\begin{eqnarray*}
9 \times x^2  % 
-108
                 \times x    % 
-2925 =0
\end{eqnarray*}
are  % 
25 and  % 
-13 .
 
Let us verity  % 
25 first:
$  % 
9 \times x^2  % 
-108
                 \times x    % 
-2925
  = % 
5625+( % 
-2700)+( % 
-2925)
  = % 
2925+( % 
-2925)
  = % 
0
$
 
Then verity  % 
-13:
$  % 
9 \times x^2  % 
-108
                 \times x    % 
-2925
  = % 
1521+( % 
1404)+( % 
-2925)
  = % 
2925+( % 
-2925)
  = % 
0
$
 
 
 
   
   
 \vspace{0.2in}
Here are still some constants for use:
 
 
\noindent\begin{tabular}{|l|l|l|}
\hline
Constant & Symbol & Value \\
\hline
 
Mass of proton &
$m_p$ &
 $ 1.6726231 \times 10^{-27} $
kg \\
\hline
 
Boltzmann's constant &
$k$ &
 $ 1.381 \times 10^{-23} $
J/K \\
\hline
 
\end{tabular}
 
Thank you very much for answering these questions!
 
{\textbf{\large{Please be advised}}} that in this paper there are questions from
27.1 through
27.9.
And any one of them may contain more than one sub-question, thus the total number
of sub-questions here is around 14, of which
13 should be answered.
 
   
   
   
   
\vspace{1.0in} 
{\textbf{\large{ *** END OF PAPER, THANKS *** }}} 
   
   
\hspace{1.0in} By: 
         239(         26,          34)
   
   
   
   
\newpage 
\setcounter{page}{ 
    28001 } 
   
   
\noindent{\textbf{\large{THIS IS THE ANSWER AND SOLUTION FOR}}}
   
   
 {\textbf{ \Large{ PAPER NUMBER          28 }}}
   
   
\vspace{0.2in}
   
   
\markboth{Answer and solution NOT for examinees !!!{\today}}{Answer and solution NOT for examinees !!! {\today}}
   
   
   
   
 \vspace{0.2in}
 
 
{\Huge  THIS IS AN EXAMPLE OF}
 
{\Huge  PERSONALIZED TESTS. }
 
If needed, please use the following constants.
 
 
 
\noindent\begin{tabular}{|l|l|l|}
\hline
Constant & Symbol & Value \\
\hline
Acceleration due to earth's gravity &
$g$ &
 $ 9.80 $
m/s$^2$ \\
\hline
Avogadro's number &
$N_A$ &
 $ 6.0221367 \times 10^{23} $
mol$^{-1}$ \\
\hline
Boltzmann's constant &
$k$ &
 $ 1.380658 \times 10^{-23} $
J/K \\
\hline
Coulomb's constant &
$k$ &
 $ 8.99 \times 10^{9} $
N$\cdot $m$^2$/C$^2$ \\
\hline
Electron charge magnitiude &
$e$ &
 $ 1.60217733 \times 10^{-19} $
C \\
\hline
Permeability of free space &
$\mu _0$ &
 $ 1.25663706 \times 10^{-6} $
T$\cdot $m/A \\
\hline
Permittivity of free space &
$\epsilon _0$ &
 $ 8.854187817 \times 10^{-12} $
C$^2$/(N$\cdot $m$^2$) \\
\hline
Pi &
$\pi$ &
 $ 3.14159265 $
$ $ \\
\hline
Planck's constant &
$h$ &
 $ 6.6260755 \times 10^{-34} $
J$\cdot $s \\
\hline
Mass of electron &
$m_e$ &
 $ 9.1093897 \times 10^{-31} $
kg \\
\hline
\end{tabular}
 
 
\noindent\begin{tabular}{|l|l|l|}
\hline
Constant & Symbol & Value \\
\hline
Mass of neutron &
$m_n$ &
 $ 1.6749286 \times 10^{-27} $
kg \\
\hline
Mass of proton &
$m_p$ &
 $ 1.6726231 \times 10^{-27} $
kg \\
\hline
Speed of light in vacuum &
$c$ &
 $ 299792458. $
m/s \\
\hline
Universal gravitational constant &
$G$ &
 $ 6.67259 \times 10^{-11} $
N$\cdot $m$^2$/kg$^2$ \\
\hline
Universal gas constant &
$R$ &
 $ 8.314510 $
J/(mol$\cdot $K) \\
\hline
\end{tabular}
 
 
{\textbf{\large{Please be advised}}} that in this paper there are questions from
28.1 through
28.9.
And any one of them may contain more than one sub-question, thus the total number
of sub-questions here is around 14, of which
13 should be answered.
 
\vspace{0.3in}
 
 
   
   
   
\vspace{0.2in}
   
In this paper, big questions will be generated in the following order: 
   
   
            1(          6)
 ,
            2(          5)
 ,
            3(          3)
 ,
            4(          4)
 ,
            5(          1)
 ,
            6(          2)
 ,
            7(          8)
 ,
            8(          7)
 ,
            9(          9)
 .
  
\vspace{0.2in}
  
{\textbf{\Large{QUESTION
28.1 
 (          6)
}}}
  
  
 
{\textbf{\Large{Please answer ONLY
5 of the following
6 questions (Questions
28.1.1 through
28.1.6). }}}
 
Here are still some constants for use in the following questions:
 
 
\noindent\begin{tabular}{|l|l|l|}
\hline
Constant & Symbol & Value \\
\hline
 
Boltzmann's constant &
$k$ &
 $ 1.381 \times 10^{-23} $
J/K \\
\hline
 
Avogadro's number &
$N_A$ &
 $ 6.022 \times 10^{23} $
mol$^{-1}$ \\
\hline
 
Mass of electron &
$m_e$ &
 $ 9.1093897 \times 10^{-31} $
kg \\
\hline
 
\end{tabular}
 
   
\vspace{0.2in}
   
 In this big question of CHOOSE structure,           6 questions will be generat
 ed: 
  
  
            1(         11,         26)
 ,
            2(          7,         22)
 ,
            3(         10,         25)
 ,
            4(          6,         21)
 ,
            5(         12,         27)
 ,
            6(          9,         24)
 .
  
\vspace{0.2in}
  
{\textbf{\Large{Question
28.1.1 
 (          6,         11,         26)
}}}
  
  
 
 
\noindent\vspace{0.1in}{\textbf{\Large{Solution: }}}

Since the possiblity of  % 
smoking customer is $ a =  % 
.580 $,
and the possiblity of  % 
 under 30 years old customer is $ b =  % 
.6200 $,
the possiblity of  % 
non-smoking customer is $ c = 1.0 - a = 1.0 -
.580
=  % 
.420 $ and the possiblity of  % 
equal or above 30 years old
customer is $ d = 1.0 - b = 1.0 -  % 
.6200 =  % 
.3800  $.
So the possibility of  % 
 non-smoking and  % 
equal or above 30 years old
customer is $ c \times d =  % 
.160 $.
 
 
 
 
 
\noindent\vspace{0.05in}{\textbf{\Large{Answer:}}}

The possibility of  % 
 non-smoking and  % 
equal or above 30 years old
customer is $ (1-a)(1-b) =  % 
.160 $.
 
 
  
\vspace{0.2in}
  
{\textbf{\Large{Question
28.1.2 
 (          6,          7,         22)
}}}
  
  
 
 
\noindent\vspace{0.05in}{\textbf{\Large{Auto-answer:}}}
 
 
\noindent{\textbf{\large{
C.}}}
The accelaration (vector) is
$(
17876.,
893.79 ,
-1.3407 \times 10^{6}
)km/h^2.
$
 
 
 
 
 
 
\noindent\vspace{0.1in}{\textbf{\Large{Solution: }}}

We will use the Newton's Second Law:
 
\[
\mathbf{f}=m\mathbf{a}.
\]
 
Since $\mathbf{f}=( % 
80.0,  % 
4.0,  % 
-6000.0 )N$
and $m= % 
58.0 kg$, bring them into the above equation, then we get
 
\begin{eqnarray*}
\mathbf{a}&=&\frac{\mathbf{f}}m  \\
&=&\frac{(
80.0 ,
4.0 ,
-6000.0 )N
}{ % 
58.0 kg}  \\
&=&(
1.3793 ,
6.8966 \times 10^{-2},
-103.45
)ms^{-2} \\
&=&(
17876. ,
893.79 ,
-1.3407 \times 10^{6}
)km/h^2.
\end{eqnarray*}
 
 
 
  
\vspace{0.2in}
  
{\textbf{\Large{Question
28.1.3 
 (          6,         10,         25)
}}}
  
  
 
 
\noindent\vspace{0.05in}{\textbf{\Large{Auto-answer:}}}
 
 
\noindent{\textbf{\large{
C.}}}
An airplane
 
 
\noindent{\textbf{\large{
D.}}}
A truck
 
 
 
 
  
\vspace{0.2in}
  
{\textbf{\Large{Question
28.1.4 
 (          6,          6,         21)
}}}
  
  
 
 
\noindent\vspace{0.05in}{\textbf{\Large{Answer:}}}

We will use the Newton's Second Law:
 
\[
\mathbf{f}=m\mathbf{a}.
\]
 
Since $\mathbf{f}=( % 
70.0,  % 
4.0,  % 
-9000.0 )N$
and $m= % 
56.0 kg$, bring them into the above equation, then we get
 
\begin{eqnarray*}
\mathbf{a}&=&\frac{\mathbf{f}}m  \\
&=&\frac{(
70.0 ,
4.0 ,
-9000.0 )N
}{ % 
56.0 kg}  \\
&=&(
1.2500 ,
7.1429 \times 10^{-2},
-160.71
)ms^{-2} \\
&=&(
16200. ,
925.71 ,
-2.0829 \times 10^{6}
)km/h^2.
\end{eqnarray*}
 
 
 
 
 
\noindent\vspace{0.1in}{\textbf{\Large{Solution: }}}

We will use the Newton's Second Law:
 
\[
\mathbf{f}=m\mathbf{a}.
\]
 
Since $\mathbf{f}=( % 
70.0,  % 
4.0,  % 
-9000.0 )N$
and $m= % 
56.0 kg$, bring them into the above equation, then we get
 
\begin{eqnarray*}
\mathbf{a}&=&\frac{\mathbf{f}}m  \\
&=&\frac{(
70.0 ,
4.0 ,
-9000.0 )N
}{ % 
56.0 kg}  \\
&=&(
1.2500 ,
7.1429 \times 10^{-2},
-160.71
)ms^{-2} \\
&=&(
16200. ,
925.71 ,
-2.0829 \times 10^{6}
)km/h^2.
\end{eqnarray*}
 
 
 
  
\vspace{0.2in}
  
{\textbf{\Large{Question
28.1.5 
 (          6,         12,         27)
}}}
  
  
 
 
\noindent\vspace{0.1in}{\textbf{\Large{Solution: }}}

Since the possiblity of  % 
smoking customer is $ a =  % 
.120 $,
and the possiblity of  % 
equal-or-above 30 years old customer is $ b =  % 
.7000 $,
the possiblity of  % 
non-smoking customer is $ c = 1.0 - a = 1.0 -
.120
=  % 
.880 $ and the possiblity of  % 
under 30 years old
customer is $ d = 1.0 - b = 1.0 -  % 
.7000 =  % 
.3000  $.
Then
 
\noindent
\begin{tabular}{|l|l|}
\hline
Customer & Possibility \\
\hline
smoking  and  % 
equal-or-above 30 years old  &
  $ % 
.120 \times  % 
.7000 =  % 
8.40 \times 10^{-2}$ \\
\hline
smoking  and  % 
under 30 years old &
  $ % 
.120 \times  % 
.3000 =  % 
3.60 \times 10^{-2}$ \\
\hline
 non-smoking and  % 
equal-or-above 30 years old  &
  $ % 
.880 \times  % 
.7000 =  % 
.616$ \\
\hline
 non-smoking and  % 
under 30 years old &
  $ % 
.880 \times  % 
.3000 =  % 
.264$ \\
\hline
\end{tabular}
 
\noindent
And the total summation of all possibilities is $  % 
1.000 $.
 
 
 
 
 
 
\noindent\vspace{0.05in}{\textbf{\Large{Answer:}}}

 
\noindent
\begin{tabular}{|l|l|}
\hline
Customer & Possibility \\
\hline
smoking  and  % 
equal-or-above 30 years old &
  $ % 
8.40 \times 10^{-2}$ \\
\hline
smoking  and  % 
under 30 years old &
  $ % 
3.60 \times 10^{-2}$ \\
\hline
 non-smoking and  % 
equal-or-above 30 years old &
  $ % 
.616$ \\
\hline
 non-smoking and  % 
under 30 years old &
  $ % 
.264$ \\
\hline
\end{tabular}
 
\noindent
 And the total summation of all possibilities is $  % 
1.000 $.
 
 
 
  
\vspace{0.2in}
  
{\textbf{\Large{Question
28.1.6 
 (          6,          9,         24)
}}}
  
  
 
 
\noindent\vspace{0.1in}{\textbf{\Large{Solution: }}}

By using Newton's Law of Universal Gravitation:
\[
F=G \frac{(Sun's \hspace{0.1in} mass) \times (Planet's \hspace{0.1in} mass)} { (distance)^2},
\]
where
$ G= % 
6.67 \times 10^{-11}N m^{2}(kg)^{-2}$ , the forces can be easily calculated as
 
\vspace{0.2in}
 
 
\begin{tabular}{|l|l|l|l|}
\hline
The Planet & Mass ($kg$) & Distanace from Sun ($m$) & The Force ($N$)\\
\hline
Mercury  &
           $ % 
5.00000000 \times 10^{24} $   &
             $ % 
2.000000000 \times 10^{24} $    & $ % 
7.50 \times 10^{-10} $
\\  \hline
Venus    &
           $  % 
6.00 \times 10^{24}  $     &
             $ % 
4.00 \times 10^{24} $    & $ % 
2.25 \times 10^{-10} $
\\  \hline
Earth    &
           $  % 
7.00 \times 10^{24}  $     &
             $ % 
5.00 \times 10^{24} $    & $ % 
1.68 \times 10^{-10} $
\\   \hline
Mars     &
           $  % 
7.00 \times 10^{24} $     &
             $ % 
7.00 \times 10^{24} $    & $ % 
8.58 \times 10^{-11} $
\\   \hline
Jupiter  &
           $  % 
5.00 \times 10^{24} $    &
             $ % 
3.00 \times 10^{24} $    & $ % 
3.33 \times 10^{-10} $
\\  \hline
Saturn   &
           $  % 
7.00 \times 10^{24} $    &
             $ % 
6.00 \times 10^{24}  $    & $ % 
1.17 \times 10^{-10} $
\\  \hline
Uranus   &
           $  % 
9.00 \times 10^{24} $    &
             $ % 
6.00 \times 10^{24} $    & $ % 
1.50 \times 10^{-10} $
\\  \hline
Neptune  &
           $  % 
5.00 \times 10^{24} $    &
             $ % 
7.00 \times 10^{24} $    & $ % 
6.13 \times 10^{-11} $
\\  \hline
 
\end{tabular}
 
 
 
 
 
 
\noindent\vspace{0.05in}{\textbf{\Large{Answer:}}}

By using Newton's Law of Universal Gravitation:
\[
F=G \frac{(Sun's \hspace{0.1in} mass) \times (Planet's \hspace{0.1in} mass)} { (distance)^2},
\]
where
$ G= % 
6.67 \times 10^{-11} N m^{2}(kg)^{-2}$ , the forces can be easily calculated as
 
\vspace{0.2in}
 
 
\begin{tabular}{|l|l|l|l|}
\hline
The Planet & Mass ($kg$) & Distanace from Sun ($m$) & The Force ($N$)\\
\hline
Mercury  &
           $ % 
5.00000000 \times 10^{24}  $   &
             $ % 
2.000000000 \times 10^{24}$    & $ % 
7.50 \times 10^{-10} $
\\  \hline
Venus    &
           $  % 
6.00 \times 10^{24}  $     &
             $ % 
4.00 \times 10^{24} $    & $ % 
2.25 \times 10^{-10} $
\\  \hline
Earth    &
           $  % 
7.00 \times 10^{24}$     &
             $ % 
5.00 \times 10^{24} $    & $ % 
1.68 \times 10^{-10} $
\\   \hline
Mars     &
           $  % 
7.00 \times 10^{24} $     &
             $ % 
7.00 \times 10^{24}$    & $ % 
8.58 \times 10^{-11} $
\\   \hline
Jupiter  &
           $  % 
5.00 \times 10^{24}  $    &
             $ % 
3.00 \times 10^{24} $    & $ % 
3.33 \times 10^{-10}3 $
\\  \hline
Saturn   &
           $  % 
7.00 \times 10^{24}   $    &
             $ % 
6.00 \times 10^{24}  $    & $ % 
1.17 \times 10^{-10} $
\\  \hline
Uranus   &
           $  % 
9.00 \times 10^{24} $    &
             $ % 
6.00 \times 10^{24}$    & $ % 
1.50 \times 10^{-10} $
\\  \hline
Neptune  &
           $  % 
5.00 \times 10^{24}  $    &
             $ % 
7.00 \times 10^{24} $    & $ % 
6.13 \times 10^{-11} $
\\  \hline
 
\end{tabular}
 
 
 
 
   
   
\vspace{0.3in}
{\textbf{\LARGE{You have done all the above? A very good beginning, please go ahead.}}}
More constants the
Mass of electron
$m_e$$ =
9.109390 \times 10^{-31} $
kg
,
Universal gas constant
$R$$ =
8.315 $
J/(mol$\cdot $K)
,
$e$$ =
1.60217733 \times 10^{-19} $
C
, and
$m_p$$ =
1.6726231 \times 10^{-27} $
kg
%
may be very helpful.
\vspace{0.3in}
   
   
  
\vspace{0.2in}
  
{\textbf{\Large{QUESTION
28.2 
 (          5,          5,          5)
}}}
  
  
 
 
\noindent\vspace{0.05in}{\textbf{\Large{Answer:}}}

 
\noindent\begin{tabular}{|l|l|}\hline The correct & \\
          answer &  % 
$T$ \\ \hline \end{tabular}
1. $ % 
80$ is an  % 
even number.
 
\noindent\begin{tabular}{|l|l|}\hline The correct & \\
          answer &  % 
$T$ \\ \hline \end{tabular}
2.  % 
Toronto is in  % 
Ontario province.
 
\noindent\begin{tabular}{|l|l|}\hline The correct & \\
          answer &  % 
$F$ \\ \hline \end{tabular}
3.  % 
$\left| \mathbf{F}\right| =Gm_1m_2r^{-2}$ is a mathmatical form of  % 
the Newton's Second Law.
 
 
 
  
\vspace{0.2in}
  
{\textbf{\Large{QUESTION
28.3 
 (          3,          3,          3)
}}}
  
  
 
 
\noindent\vspace{0.05in}{\textbf{\Large{Auto-answer:}}}
 
 
\noindent{\textbf{\large{
A.}}}
Canada has  %
10 provinces and  %
3 territories.
 
 
 
 
  
\vspace{0.2in}
  
{\textbf{\Large{QUESTION
28.4 
 (          4,          4,          4)
}}}
  
  
 
 
\noindent\vspace{0.05in}{\textbf{\Large{Auto-answer:}}}
  
  
\begin{tabular}{|l|l|l|}
 \hline
 Column Left & Column Right  & Answers       \\ 
 \hline
{\textbf{\large{
A.}}}
asdf(:)
  & 
b
 & 
{\textbf{\large{
B.}}}
 \\ 
 \hline
{\textbf{\large{
B.}}}
B
  & 
a
 & 
{\textbf{\large{
D.}}}
 \\ 
 \hline
{\textbf{\large{
C.}}}
yjh
  & 
YJH
 & 
{\textbf{\large{
C.}}}
 \\ 
 \hline
{\textbf{\large{
D.}}}
A
  & 
eR
 & 
{\textbf{\large{
E.}}}
 \\ 
 \hline
{\textbf{\large{
E.}}}
er
  & 
ASDF(:)
 & 
{\textbf{\large{
A.}}}
 \\ 
 \hline
 \end{tabular}
  
  
 
 
 
 
  
\vspace{0.2in}
  
{\textbf{\Large{QUESTION
28.5 
 (          1,          1,          1)
}}}
  
  


 
 
\noindent\vspace{0.05in}{\textbf{\Large{Auto-answer:}}}
 
 
\noindent{\textbf{\large{
G.}}}
The accelaration is $  %
(
1.80,
8.0 \times 10^{-2},
-60.000)
ms^{-2} $.
 
 
 
 
 
 
\noindent\vspace{0.05in}{\textbf{\Large{Answer:}}}

The correct answer from the choices is


\noindent{\textbf{\large{
G.}}}
The accelaration is $  %
(
1.80,
8.0 \times 10^{-2},
-60.000)
ms^{-2} $.
 
 
 
 
 
\noindent\vspace{0.1in}{\textbf{\Large{Solution: }}}

We will use the Newton's Second Law:
 
\[
\mathbf{f}=m\mathbf{a}.
\]
 
Since $\mathbf{f}= % 
(90.0 , 4.0 , -3000.0) N$
and $m= % 
50.0000kg$, bring them into the above equation, then we get
 
\begin{eqnarray*}
\mathbf{a}&=&\frac{\mathbf{f}}m  \\
&=&\frac{ % 
(90.0 , 4.0 , -3000.0) N}{ % 
50.0000kg}  \\
&=& % 
(1.80 , 8.0 \times 10^{-2} , -60.000) ms^{-2}
\end{eqnarray*}
 
 
 
  
\vspace{0.2in}
  
{\textbf{\Large{QUESTION
28.6 
 (          2,          2,          2)
}}}
  
  
 
 
\noindent\vspace{0.05in}{\textbf{\Large{Auto-answer:}}}
 
 
\noindent{\textbf{\large{
E.}}}
The accelaration is
$(
1.6667ms^{-2},
1680.0km/h^2,
-148.15ms^{-2}
).
$
 
 
 
 
 
 
\noindent\vspace{0.1in}{\textbf{\Large{Solution: }}}

We will use the Newton's Second Law:
 
\[
\mathbf{f}=m\mathbf{a}.
\]
 
Since $\mathbf{f}=( % 
90.000,  % 
7.0000,  % 
-8000.0 )N$
and $m= % 
54.0000kg$, bring them into the above equation, then we get
 
\begin{eqnarray*}
\mathbf{a}&=&\frac{\mathbf{f}}m  \\
&=&\frac{(
90.000 ,
7.0000 ,
-8000.0 )N
}{ % 
54.0000 kg}  \\
&=&(
1.6667 ,
.12963,
-148.15
)ms^{-2} \\
&=&(
21600. ,
1680.0 ,
-1.9200 \times 10^{6}
)km/h^2.
\end{eqnarray*}
 
 
 
   
   
\vspace{0.3in}
{\textbf{\LARGE{You have done all the above? Excellent! Not much left, please continue.}}}
\vspace{0.3in}
   
   
  
\vspace{0.2in}
  
{\textbf{\Large{QUESTION
28.7 
 (          8,         15,         60)
}}}
  
  
 
 
\noindent\vspace{0.05in}{\textbf{\Large{Answer:}}}

 
$\left( \begin{array}{ccccccccccccccc}
           6 & 
           5 & 
           6 & 
           4 \\ 
           4 & 
           5 & 
           4 & 
           6 \\ 
           5 & 
           6 & 
           5 & 
           4
\end{array}\right) \times
\left( \begin{array}{c}
           2 \\ 
           2 \\ 
           2 \\ 
           2
\end{array}\right)  =
\left( \begin{array}{c}
          42 \\ 
          38 \\ 
          40
\end{array}\right)  $
 
$  % 
 \left( \begin{array}
 {
 c
 c
 }
 \beta & 
 \Gamma \\ 
 \epsilon & 
 \beta \\ 
 \eta & 
 \beta \\ 
                    \Xi & 
 \epsilon
 \end{array} \right)
 \left( \begin{array}
 {
 c
 }
 \beta \\ 
 \gamma
 \end{array} \right)
=
  \left( \begin{array}
 {
 c
 }
 \beta \times  \beta   +  \Gamma \times  \gamma \\ 
 \epsilon \times  \beta   +  \beta \times  \gamma \\ 
 \eta \times  \beta   +  \beta \times  \gamma \\ 
                    \Xi \times  \beta   +  \epsilon \times  \gamma
 \end{array} \right)
$
 
 
 
 
 
\noindent\vspace{0.1in}{\textbf{\Large{Solution: }}}

 
 
  
\vspace{0.2in}
  
{\textbf{\Large{QUESTION
28.8 
 (          7,         14,         50)
}}}
  
  
 
 
\noindent\vspace{0.05in}{\textbf{\Large{Auto-answer:}}}
 
 
\noindent{\textbf{\large{
B.}}}
  The accelaration is $  %
(
1.60,
.10,
-180.00)
ms^{-2} $.
 
 
 
 
 
 
\noindent\vspace{0.1in}{\textbf{\Large{Solution: }}}

We will use the Newton's Second Law:
 
\[
\mathbf{f}=m\mathbf{a}.
\]
 
Since $\mathbf{f}= % 
(80.0 , 5.0 , -9000.0) N$
and $m= % 
50.0kg$, bring them into the above equation, then we get
 
\begin{eqnarray*}
\mathbf{a}&=&\frac{\mathbf{f}}m  \\
&=&\frac{ % 
(80.0 , 5.0 , -9000.0) N}{ % 
50.0kg}  \\
&=& % 
(1.60 , .10 , -180.00) ms^{-2}
\end{eqnarray*}
 
 
 
  
\vspace{0.2in}
  
{\textbf{\Large{QUESTION
28.9 
 (          9,         16,         70)
}}}
  
  


 
 
\noindent\vspace{0.05in}{\textbf{\Large{Answer:}}}

17,  % 
-31
 
 
 
 
 
\noindent\vspace{0.1in}{\textbf{\Large{Solution: }}}

Roots to the equation
\begin{eqnarray*}
15 \times x^2  % 
+  % 
210
                 \times x    % 
-7905 =0
\end{eqnarray*}
are  % 
17 and  % 
-31 .
 
Let us verity  % 
17 first:
$  % 
15 \times x^2  % 
+  % 
210
                 \times x    % 
-7905
  = % 
4335+( % 
3570)+( % 
-7905)
  = % 
7905+( % 
-7905)
  = % 
0
$
 
Then verity  % 
-31:
$  % 
15 \times x^2  % 
+  % 
210
                 \times x    % 
-7905
  = % 
14415+( % 
-6510)+( % 
-7905)
  = % 
7905+( % 
-7905)
  = % 
0
$
 
 
 
   
   
 \vspace{0.2in}
Here are still some constants for use:
 
 
\noindent\begin{tabular}{|l|l|l|}
\hline
Constant & Symbol & Value \\
\hline
 
Mass of proton &
$m_p$ &
 $ 1.6726231 \times 10^{-27} $
kg \\
\hline
 
Boltzmann's constant &
$k$ &
 $ 1.381 \times 10^{-23} $
J/K \\
\hline
 
\end{tabular}
 
Thank you very much for answering these questions!
 
{\textbf{\large{Please be advised}}} that in this paper there are questions from
28.1 through
28.9.
And any one of them may contain more than one sub-question, thus the total number
of sub-questions here is around 14, of which
13 should be answered.
 
   
   
   
   
\vspace{1.0in} 
{\textbf{\large{ *** END OF PAPER, THANKS *** }}} 
   
   
\hspace{1.0in} By: 
         239(         26,          34)
   
   
   
   
\newpage 
\setcounter{page}{ 
    29001 } 
   
   
\noindent{\textbf{\large{THIS IS THE ANSWER AND SOLUTION FOR}}}
   
   
 {\textbf{ \Large{ PAPER NUMBER          29 }}}
   
   
\vspace{0.2in}
   
   
\markboth{Answer and solution NOT for examinees !!!{\today}}{Answer and solution NOT for examinees !!! {\today}}
   
   
   
   
 \vspace{0.2in}
 
 
{\Huge  THIS IS AN EXAMPLE OF}
 
{\Huge  PERSONALIZED TESTS. }
 
If needed, please use the following constants.
 
 
 
\noindent\begin{tabular}{|l|l|l|}
\hline
Constant & Symbol & Value \\
\hline
Acceleration due to earth's gravity &
$g$ &
 $ 9.80 $
m/s$^2$ \\
\hline
Avogadro's number &
$N_A$ &
 $ 6.0221367 \times 10^{23} $
mol$^{-1}$ \\
\hline
Boltzmann's constant &
$k$ &
 $ 1.380658 \times 10^{-23} $
J/K \\
\hline
Coulomb's constant &
$k$ &
 $ 8.99 \times 10^{9} $
N$\cdot $m$^2$/C$^2$ \\
\hline
Electron charge magnitiude &
$e$ &
 $ 1.60217733 \times 10^{-19} $
C \\
\hline
Permeability of free space &
$\mu _0$ &
 $ 1.25663706 \times 10^{-6} $
T$\cdot $m/A \\
\hline
Permittivity of free space &
$\epsilon _0$ &
 $ 8.854187817 \times 10^{-12} $
C$^2$/(N$\cdot $m$^2$) \\
\hline
Pi &
$\pi$ &
 $ 3.14159265 $
$ $ \\
\hline
Planck's constant &
$h$ &
 $ 6.6260755 \times 10^{-34} $
J$\cdot $s \\
\hline
Mass of electron &
$m_e$ &
 $ 9.1093897 \times 10^{-31} $
kg \\
\hline
\end{tabular}
 
 
\noindent\begin{tabular}{|l|l|l|}
\hline
Constant & Symbol & Value \\
\hline
Mass of neutron &
$m_n$ &
 $ 1.6749286 \times 10^{-27} $
kg \\
\hline
Mass of proton &
$m_p$ &
 $ 1.6726231 \times 10^{-27} $
kg \\
\hline
Speed of light in vacuum &
$c$ &
 $ 299792458. $
m/s \\
\hline
Universal gravitational constant &
$G$ &
 $ 6.67259 \times 10^{-11} $
N$\cdot $m$^2$/kg$^2$ \\
\hline
Universal gas constant &
$R$ &
 $ 8.314510 $
J/(mol$\cdot $K) \\
\hline
\end{tabular}
 
 
{\textbf{\large{Please be advised}}} that in this paper there are questions from
29.1 through
29.9.
And any one of them may contain more than one sub-question, thus the total number
of sub-questions here is around 14, of which
13 should be answered.
 
\vspace{0.3in}
 
 
   
   
   
\vspace{0.2in}
   
In this paper, big questions will be generated in the following order: 
   
   
            1(          6)
 ,
            2(          2)
 ,
            3(          3)
 ,
            4(          5)
 ,
            5(          1)
 ,
            6(          4)
 ,
            7(          7)
 ,
            8(          8)
 ,
            9(          9)
 .
  
\vspace{0.2in}
  
{\textbf{\Large{QUESTION
29.1 
 (          6)
}}}
  
  
 
{\textbf{\Large{Please answer ONLY
5 of the following
6 questions (Questions
29.1.1 through
29.1.6). }}}
 
Here are still some constants for use in the following questions:
 
 
\noindent\begin{tabular}{|l|l|l|}
\hline
Constant & Symbol & Value \\
\hline
 
Boltzmann's constant &
$k$ &
 $ 1.381 \times 10^{-23} $
J/K \\
\hline
 
Avogadro's number &
$N_A$ &
 $ 6.022 \times 10^{23} $
mol$^{-1}$ \\
\hline
 
Mass of electron &
$m_e$ &
 $ 9.1093897 \times 10^{-31} $
kg \\
\hline
 
\end{tabular}
 
   
\vspace{0.2in}
   
 In this big question of CHOOSE structure,           6 questions will be generat
 ed: 
  
  
            1(          8,         23)
 ,
            2(         11,         26)
 ,
            3(          9,         24)
 ,
            4(         13,         28)
 ,
            5(         12,         27)
 ,
            6(          7,         22)
 .
  
\vspace{0.2in}
  
{\textbf{\Large{Question
29.1.1 
 (          6,          8,         23)
}}}
  
  
 
 
\noindent\vspace{0.05in}{\textbf{\Large{Auto-answer:}}}
 
 
\noindent{\textbf{\large{
C.}}}
The accelaration is
$(
.40000ms^{-2},
.10000ms^{-2},
-2.3328 \times 10^{6}km/h^2
).
$
 
 
 
 
 
 
\noindent\vspace{0.1in}{\textbf{\Large{Solution: }}}

We will use the Newton's Second Law:
 
\[
\mathbf{f}=m\mathbf{a}.
\]
 
Since $\mathbf{f}=( % 
20.0,  % 
5.0,  % 
-9000.0 )N$
and $m= % 
50.0kg$, bring them into the above equation, then we get
 
\begin{eqnarray*}
\mathbf{a}&=&\frac{\mathbf{f}}m  \\
&=&\frac{(
20.0 ,
5.0 ,
-9000.0 )N
}{ % 
50.0 kg}  \\
&=&(
.40000 ,
.10000,
-180.00
)ms^{-2} \\
&=&(
5184.0 ,
1296.0 ,
-2.3328 \times 10^{6}
)km/h^2.
\end{eqnarray*}
 
 
 
  
\vspace{0.2in}
  
{\textbf{\Large{Question
29.1.2 
 (          6,         11,         26)
}}}
  
  
 
 
\noindent\vspace{0.1in}{\textbf{\Large{Solution: }}}

Since the possiblity of  % 
smoking customer is $ a =  % 
.660 $,
and the possiblity of  % 
equal or above 30 years old customer is $ b =  % 
.4000 $,
the possiblity of  % 
non-smoking customer is $ c = 1.0 - a = 1.0 -
.660
=  % 
.340 $ and the possiblity of  % 
under 30 years old
customer is $ d = 1.0 - b = 1.0 -  % 
.4000 =  % 
.6000  $.
So the possibility of  % 
 non-smoking and  % 
under 30 years old
customer is $ c \times d =  % 
.204 $.
 
 
 
 
 
\noindent\vspace{0.05in}{\textbf{\Large{Answer:}}}

The possibility of  % 
 non-smoking and  % 
under 30 years old
customer is $ (1-a)(1-b) =  % 
.204 $.
 
 
  
\vspace{0.2in}
  
{\textbf{\Large{Question
29.1.3 
 (          6,          9,         24)
}}}
  
  
 
 
\noindent\vspace{0.1in}{\textbf{\Large{Solution: }}}

By using Newton's Law of Universal Gravitation:
\[
F=G \frac{(Sun's \hspace{0.1in} mass) \times (Planet's \hspace{0.1in} mass)} { (distance)^2},
\]
where
$ G= % 
6.67 \times 10^{-11}N m^{2}(kg)^{-2}$ , the forces can be easily calculated as
 
\vspace{0.2in}
 
 
\begin{tabular}{|l|l|l|l|}
\hline
The Planet & Mass ($kg$) & Distanace from Sun ($m$) & The Force ($N$)\\
\hline
Mercury  &
           $ % 
3.00000000 \times 10^{24} $   &
             $ % 
8.000000000 \times 10^{24} $    & $ % 
2.50 \times 10^{-11} $
\\  \hline
Venus    &
           $  % 
6.00 \times 10^{24}  $     &
             $ % 
9.00 \times 10^{24} $    & $ % 
3.95 \times 10^{-11} $
\\  \hline
Earth    &
           $  % 
7.00 \times 10^{24}  $     &
             $ % 
4.00 \times 10^{24} $    & $ % 
2.33 \times 10^{-10} $
\\   \hline
Mars     &
           $  % 
6.00 \times 10^{24} $     &
             $ % 
2.00 \times 10^{24} $    & $ % 
8.00 \times 10^{-10} $
\\   \hline
Jupiter  &
           $  % 
9.00 \times 10^{24} $    &
             $ % 
3.00 \times 10^{24} $    & $ % 
5.34 \times 10^{-10} $
\\  \hline
Saturn   &
           $  % 
4.00 \times 10^{24} $    &
             $ % 
8.00 \times 10^{24}  $    & $ % 
3.33 \times 10^{-11} $
\\  \hline
Uranus   &
           $  % 
4.00 \times 10^{24} $    &
             $ % 
6.00 \times 10^{24} $    & $ % 
5.93 \times 10^{-11} $
\\  \hline
Neptune  &
           $  % 
9.00 \times 10^{24} $    &
             $ % 
3.00 \times 10^{24} $    & $ % 
5.34 \times 10^{-10} $
\\  \hline
 
\end{tabular}
 
 
 
 
 
 
\noindent\vspace{0.05in}{\textbf{\Large{Answer:}}}

By using Newton's Law of Universal Gravitation:
\[
F=G \frac{(Sun's \hspace{0.1in} mass) \times (Planet's \hspace{0.1in} mass)} { (distance)^2},
\]
where
$ G= % 
6.67 \times 10^{-11} N m^{2}(kg)^{-2}$ , the forces can be easily calculated as
 
\vspace{0.2in}
 
 
\begin{tabular}{|l|l|l|l|}
\hline
The Planet & Mass ($kg$) & Distanace from Sun ($m$) & The Force ($N$)\\
\hline
Mercury  &
           $ % 
3.00000000 \times 10^{24}  $   &
             $ % 
8.000000000 \times 10^{24}$    & $ % 
2.50 \times 10^{-11} $
\\  \hline
Venus    &
           $  % 
6.00 \times 10^{24}  $     &
             $ % 
9.00 \times 10^{24} $    & $ % 
3.95 \times 10^{-11} $
\\  \hline
Earth    &
           $  % 
7.00 \times 10^{24}$     &
             $ % 
4.00 \times 10^{24} $    & $ % 
2.33 \times 10^{-10} $
\\   \hline
Mars     &
           $  % 
6.00 \times 10^{24} $     &
             $ % 
2.00 \times 10^{24}$    & $ % 
8.00 \times 10^{-10} $
\\   \hline
Jupiter  &
           $  % 
9.00 \times 10^{24}  $    &
             $ % 
3.00 \times 10^{24} $    & $ % 
5.34 \times 10^{-10}3 $
\\  \hline
Saturn   &
           $  % 
4.00 \times 10^{24}   $    &
             $ % 
8.00 \times 10^{24}  $    & $ % 
3.33 \times 10^{-11} $
\\  \hline
Uranus   &
           $  % 
4.00 \times 10^{24} $    &
             $ % 
6.00 \times 10^{24}$    & $ % 
5.93 \times 10^{-11} $
\\  \hline
Neptune  &
           $  % 
9.00 \times 10^{24}  $    &
             $ % 
3.00 \times 10^{24} $    & $ % 
5.34 \times 10^{-10} $
\\  \hline
 
\end{tabular}
 
 
 
 
  
\vspace{0.2in}
  
{\textbf{\Large{Question
29.1.4 
 (          6,         13,         28)
}}}
  
  
 
 
\noindent\vspace{0.05in}{\textbf{\Large{Answer:}}}

7;
 
8;
 
The operation is  % 
ADDITION and the result is
$ % 
15.000$.
 
 
 
  
\vspace{0.2in}
  
{\textbf{\Large{Question
29.1.5 
 (          6,         12,         27)
}}}
  
  
 
 
\noindent\vspace{0.1in}{\textbf{\Large{Solution: }}}

Since the possiblity of  % 
smoking customer is $ a =  % 
.790 $,
and the possiblity of  % 
equal-or-above 30 years old customer is $ b =  % 
.6200 $,
the possiblity of  % 
non-smoking customer is $ c = 1.0 - a = 1.0 -
.790
=  % 
.210 $ and the possiblity of  % 
under 30 years old
customer is $ d = 1.0 - b = 1.0 -  % 
.6200 =  % 
.3800  $.
Then
 
\noindent
\begin{tabular}{|l|l|}
\hline
Customer & Possibility \\
\hline
smoking  and  % 
equal-or-above 30 years old  &
  $ % 
.790 \times  % 
.6200 =  % 
.490$ \\
\hline
smoking  and  % 
under 30 years old &
  $ % 
.790 \times  % 
.3800 =  % 
.300$ \\
\hline
 non-smoking and  % 
equal-or-above 30 years old  &
  $ % 
.210 \times  % 
.6200 =  % 
.130$ \\
\hline
 non-smoking and  % 
under 30 years old &
  $ % 
.210 \times  % 
.3800 =  % 
7.98 \times 10^{-2}$ \\
\hline
\end{tabular}
 
\noindent
And the total summation of all possibilities is $  % 
1.000 $.
 
 
 
 
 
 
\noindent\vspace{0.05in}{\textbf{\Large{Answer:}}}

 
\noindent
\begin{tabular}{|l|l|}
\hline
Customer & Possibility \\
\hline
smoking  and  % 
equal-or-above 30 years old &
  $ % 
.490$ \\
\hline
smoking  and  % 
under 30 years old &
  $ % 
.300$ \\
\hline
 non-smoking and  % 
equal-or-above 30 years old &
  $ % 
.130$ \\
\hline
 non-smoking and  % 
under 30 years old &
  $ % 
7.98 \times 10^{-2}$ \\
\hline
\end{tabular}
 
\noindent
 And the total summation of all possibilities is $  % 
1.000 $.
 
 
 
  
\vspace{0.2in}
  
{\textbf{\Large{Question
29.1.6 
 (          6,          7,         22)
}}}
  
  
 
 
\noindent\vspace{0.05in}{\textbf{\Large{Auto-answer:}}}
 
 
\noindent{\textbf{\large{
C.}}}
The accelaration (vector) is
$(
7476.9,
747.69 ,
-498462.
)km/h^2.
$
 
 
 
 
 
 
\noindent\vspace{0.1in}{\textbf{\Large{Solution: }}}

We will use the Newton's Second Law:
 
\[
\mathbf{f}=m\mathbf{a}.
\]
 
Since $\mathbf{f}=( % 
30.0,  % 
3.0,  % 
-2000.0 )N$
and $m= % 
52.0 kg$, bring them into the above equation, then we get
 
\begin{eqnarray*}
\mathbf{a}&=&\frac{\mathbf{f}}m  \\
&=&\frac{(
30.0 ,
3.0 ,
-2000.0 )N
}{ % 
52.0 kg}  \\
&=&(
.57692 ,
5.7692 \times 10^{-2},
-38.462
)ms^{-2} \\
&=&(
7476.9 ,
747.69 ,
-498462.
)km/h^2.
\end{eqnarray*}
 
 
 
   
   
\vspace{0.3in}
{\textbf{\LARGE{You have done all the above? A very good beginning, please go ahead.}}}
More constants the
Mass of electron
$m_e$$ =
9.109390 \times 10^{-31} $
kg
,
Universal gas constant
$R$$ =
8.315 $
J/(mol$\cdot $K)
,
$e$$ =
1.60217733 \times 10^{-19} $
C
, and
$m_p$$ =
1.6726231 \times 10^{-27} $
kg
%
may be very helpful.
\vspace{0.3in}
   
   
  
\vspace{0.2in}
  
{\textbf{\Large{QUESTION
29.2 
 (          2,          2,          2)
}}}
  
  
 
 
\noindent\vspace{0.05in}{\textbf{\Large{Auto-answer:}}}
 
 
\noindent{\textbf{\large{
E.}}}
The accelaration is
$(
.55556ms^{-2},
720.00km/h^2,
-111.11ms^{-2}
).
$
 
 
 
 
 
 
\noindent\vspace{0.1in}{\textbf{\Large{Solution: }}}

We will use the Newton's Second Law:
 
\[
\mathbf{f}=m\mathbf{a}.
\]
 
Since $\mathbf{f}=( % 
30.000,  % 
3.0000,  % 
-6000.0 )N$
and $m= % 
54.0000kg$, bring them into the above equation, then we get
 
\begin{eqnarray*}
\mathbf{a}&=&\frac{\mathbf{f}}m  \\
&=&\frac{(
30.000 ,
3.0000 ,
-6000.0 )N
}{ % 
54.0000 kg}  \\
&=&(
.55556 ,
5.5556 \times 10^{-2},
-111.11
)ms^{-2} \\
&=&(
7200.0 ,
720.00 ,
-1.4400 \times 10^{6}
)km/h^2.
\end{eqnarray*}
 
 
 
  
\vspace{0.2in}
  
{\textbf{\Large{QUESTION
29.3 
 (          3,          3,          3)
}}}
  
  
 
 
\noindent\vspace{0.05in}{\textbf{\Large{Auto-answer:}}}
 
 
\noindent{\textbf{\large{
E.}}}
Canada has  %
10 provinces and  %
3 territories.
 
 
 
 
  
\vspace{0.2in}
  
{\textbf{\Large{QUESTION
29.4 
 (          5,          5,          5)
}}}
  
  
 
 
\noindent\vspace{0.05in}{\textbf{\Large{Answer:}}}

 
\noindent\begin{tabular}{|l|l|}\hline The correct & \\
          answer &  % 
$T$ \\ \hline \end{tabular}
1. $ % 
30$ is an  % 
even number.
 
\noindent\begin{tabular}{|l|l|}\hline The correct & \\
          answer &  % 
$F$ \\ \hline \end{tabular}
2.  % 
Montreal is in  % 
Ontario province.
 
\noindent\begin{tabular}{|l|l|}\hline The correct & \\
          answer &  % 
$T$ \\ \hline \end{tabular}
3.  % 
$\mathbf{F}=m\mathbf{a}$ is a mathmatical form of  % 
the Newton's Second Law.
 
 
 
  
\vspace{0.2in}
  
{\textbf{\Large{QUESTION
29.5 
 (          1,          1,          1)
}}}
  
  


 
 
\noindent\vspace{0.05in}{\textbf{\Large{Auto-answer:}}}
 
 
\noindent{\textbf{\large{
E.}}}
The accelaration is $  %
(
.800,
.14,
-100.00)
ms^{-2} $.
 
 
 
 
 
 
\noindent\vspace{0.05in}{\textbf{\Large{Answer:}}}

The correct answer from the choices is


\noindent{\textbf{\large{
E.}}}
The accelaration is $  %
(
.800,
.14,
-100.00)
ms^{-2} $.
 
 
 
 
 
\noindent\vspace{0.1in}{\textbf{\Large{Solution: }}}

We will use the Newton's Second Law:
 
\[
\mathbf{f}=m\mathbf{a}.
\]
 
Since $\mathbf{f}= % 
(40.0 , 7.0 , -5000.0) N$
and $m= % 
50.0000kg$, bring them into the above equation, then we get
 
\begin{eqnarray*}
\mathbf{a}&=&\frac{\mathbf{f}}m  \\
&=&\frac{ % 
(40.0 , 7.0 , -5000.0) N}{ % 
50.0000kg}  \\
&=& % 
(.800 , .14 , -100.00) ms^{-2}
\end{eqnarray*}
 
 
 
  
\vspace{0.2in}
  
{\textbf{\Large{QUESTION
29.6 
 (          4,          4,          4)
}}}
  
  
 
 
\noindent\vspace{0.05in}{\textbf{\Large{Auto-answer:}}}
  
  
\begin{tabular}{|l|l|l|}
 \hline
 Column Left & Column Right  & Answers       \\ 
 \hline
{\textbf{\large{
A.}}}
Er
  & 
YJH
 & 
{\textbf{\large{
E.}}}
 \\ 
 \hline
{\textbf{\large{
B.}}}
C
  & 
eR
 & 
{\textbf{\large{
A.}}}
, 
{\textbf{\large{
C.}}}
 \\ 
 \hline
{\textbf{\large{
C.}}}
er
  & 
b
 & 
{\textbf{\large{
D.}}}
 \\ 
 \hline
{\textbf{\large{
D.}}}
B
  & 
ER
 & 
{\textbf{\large{
A.}}}
, 
{\textbf{\large{
C.}}}
 \\ 
 \hline
{\textbf{\large{
E.}}}
yjh
  & 
c
 & 
{\textbf{\large{
B.}}}
 \\ 
 \hline
 \end{tabular}
  
  
 
 
 
 
   
   
\vspace{0.3in}
{\textbf{\LARGE{You have done all the above? Excellent! Not much left, please continue.}}}
\vspace{0.3in}
   
   
  
\vspace{0.2in}
  
{\textbf{\Large{QUESTION
29.7 
 (          7,         14,         50)
}}}
  
  
 
 
\noindent\vspace{0.05in}{\textbf{\Large{Auto-answer:}}}
 
 
\noindent{\textbf{\large{
C.}}}
  The accelaration is $  %
(
1.54,
.19,
-57.692)
ms^{-2} $.
 
 
 
 
 
 
\noindent\vspace{0.1in}{\textbf{\Large{Solution: }}}

We will use the Newton's Second Law:
 
\[
\mathbf{f}=m\mathbf{a}.
\]
 
Since $\mathbf{f}= % 
(80.0 , 10.0 , -3000.0) N$
and $m= % 
52.0kg$, bring them into the above equation, then we get
 
\begin{eqnarray*}
\mathbf{a}&=&\frac{\mathbf{f}}m  \\
&=&\frac{ % 
(80.0 , 10.0 , -3000.0) N}{ % 
52.0kg}  \\
&=& % 
(1.54 , .19 , -57.692) ms^{-2}
\end{eqnarray*}
 
 
 
  
\vspace{0.2in}
  
{\textbf{\Large{QUESTION
29.8 
 (          8,         15,         60)
}}}
  
  
 
 
\noindent\vspace{0.05in}{\textbf{\Large{Answer:}}}

 
$\left( \begin{array}{ccccccccccccccc}
           5 & 
           6 & 
           5 & 
           5 \\ 
           5 & 
           5 & 
           7 & 
           4 \\ 
           4 & 
           6 & 
           6 & 
           6
\end{array}\right) \times
\left( \begin{array}{c}
           2 \\ 
           2 \\ 
           2 \\ 
           2
\end{array}\right)  =
\left( \begin{array}{c}
          42 \\ 
          42 \\ 
          44
\end{array}\right)  $
 
$  % 
 \left( \begin{array}
 {
 c
 c
 }
 \Gamma & 
 \Gamma \\ 
 \sigma & 
                    \Xi \\ 
 \Lambda & 
 \delta \\ 
 \delta & 
 \rho
 \end{array} \right)
 \left( \begin{array}
 {
 c
 }
 \beta \\ 
 \beta
 \end{array} \right)
=
  \left( \begin{array}
 {
 c
 }
 \Gamma \times  \beta   +  \Gamma \times  \beta \\ 
 \sigma \times  \beta   +                     \Xi \times  \beta \\ 
 \Lambda \times  \beta   +  \delta \times  \beta \\ 
 \delta \times  \beta   +  \rho \times  \beta
 \end{array} \right)
$
 
 
 
 
 
\noindent\vspace{0.1in}{\textbf{\Large{Solution: }}}

 
 
  
\vspace{0.2in}
  
{\textbf{\Large{QUESTION
29.9 
 (          9,         16,         70)
}}}
  
  


 
 
\noindent\vspace{0.05in}{\textbf{\Large{Answer:}}}

21,  % 
-7
 
 
 
 
 
\noindent\vspace{0.1in}{\textbf{\Large{Solution: }}}

Roots to the equation
\begin{eqnarray*}
-15 \times x^2  % 
+  % 
210
                 \times x    % 
+  % 
2205 =0
\end{eqnarray*}
are  % 
21 and  % 
-7 .
 
Let us verity  % 
21 first:
$  % 
-15 \times x^2  % 
+  % 
210
                 \times x    % 
+  % 
2205
  = % 
-6615+( % 
4410)+( % 
2205)
  = % 
-2205+( % 
2205)
  = % 
0
$
 
Then verity  % 
-7:
$  % 
-15 \times x^2  % 
+  % 
210
                 \times x    % 
+  % 
2205
  = % 
-735+( % 
-1470)+( % 
2205)
  = % 
-2205+( % 
2205)
  = % 
0
$
 
 
 
   
   
 \vspace{0.2in}
Here are still some constants for use:
 
 
\noindent\begin{tabular}{|l|l|l|}
\hline
Constant & Symbol & Value \\
\hline
 
Mass of proton &
$m_p$ &
 $ 1.6726231 \times 10^{-27} $
kg \\
\hline
 
Boltzmann's constant &
$k$ &
 $ 1.381 \times 10^{-23} $
J/K \\
\hline
 
\end{tabular}
 
Thank you very much for answering these questions!
 
{\textbf{\large{Please be advised}}} that in this paper there are questions from
29.1 through
29.9.
And any one of them may contain more than one sub-question, thus the total number
of sub-questions here is around 14, of which
13 should be answered.
 
   
   
   
   
\vspace{1.0in} 
{\textbf{\large{ *** END OF PAPER, THANKS *** }}} 
   
   
\hspace{1.0in} By: 
         239(         26,          34)
   
   
   
   
\newpage 
\setcounter{page}{ 
    30001 } 
   
   
\noindent{\textbf{\large{THIS IS THE ANSWER AND SOLUTION FOR}}}
   
   
 {\textbf{ \Large{ PAPER NUMBER          30 }}}
   
   
\vspace{0.2in}
   
   
\markboth{Answer and solution NOT for examinees !!!{\today}}{Answer and solution NOT for examinees !!! {\today}}
   
   
   
   
 \vspace{0.2in}
 
 
{\Huge  THIS IS AN EXAMPLE OF}
 
{\Huge  PERSONALIZED TESTS. }
 
If needed, please use the following constants.
 
 
 
\noindent\begin{tabular}{|l|l|l|}
\hline
Constant & Symbol & Value \\
\hline
Acceleration due to earth's gravity &
$g$ &
 $ 9.80 $
m/s$^2$ \\
\hline
Avogadro's number &
$N_A$ &
 $ 6.0221367 \times 10^{23} $
mol$^{-1}$ \\
\hline
Boltzmann's constant &
$k$ &
 $ 1.380658 \times 10^{-23} $
J/K \\
\hline
Coulomb's constant &
$k$ &
 $ 8.99 \times 10^{9} $
N$\cdot $m$^2$/C$^2$ \\
\hline
Electron charge magnitiude &
$e$ &
 $ 1.60217733 \times 10^{-19} $
C \\
\hline
Permeability of free space &
$\mu _0$ &
 $ 1.25663706 \times 10^{-6} $
T$\cdot $m/A \\
\hline
Permittivity of free space &
$\epsilon _0$ &
 $ 8.854187817 \times 10^{-12} $
C$^2$/(N$\cdot $m$^2$) \\
\hline
Pi &
$\pi$ &
 $ 3.14159265 $
$ $ \\
\hline
Planck's constant &
$h$ &
 $ 6.6260755 \times 10^{-34} $
J$\cdot $s \\
\hline
Mass of electron &
$m_e$ &
 $ 9.1093897 \times 10^{-31} $
kg \\
\hline
\end{tabular}
 
 
\noindent\begin{tabular}{|l|l|l|}
\hline
Constant & Symbol & Value \\
\hline
Mass of neutron &
$m_n$ &
 $ 1.6749286 \times 10^{-27} $
kg \\
\hline
Mass of proton &
$m_p$ &
 $ 1.6726231 \times 10^{-27} $
kg \\
\hline
Speed of light in vacuum &
$c$ &
 $ 299792458. $
m/s \\
\hline
Universal gravitational constant &
$G$ &
 $ 6.67259 \times 10^{-11} $
N$\cdot $m$^2$/kg$^2$ \\
\hline
Universal gas constant &
$R$ &
 $ 8.314510 $
J/(mol$\cdot $K) \\
\hline
\end{tabular}
 
 
{\textbf{\large{Please be advised}}} that in this paper there are questions from
30.1 through
30.9.
And any one of them may contain more than one sub-question, thus the total number
of sub-questions here is around 14, of which
13 should be answered.
 
\vspace{0.3in}
 
 
   
   
   
\vspace{0.2in}
   
In this paper, big questions will be generated in the following order: 
   
   
            1(          6)
 ,
            2(          4)
 ,
            3(          3)
 ,
            4(          1)
 ,
            5(          5)
 ,
            6(          2)
 ,
            7(          8)
 ,
            8(          7)
 ,
            9(          9)
 .
  
\vspace{0.2in}
  
{\textbf{\Large{QUESTION
30.1 
 (          6)
}}}
  
  
 
{\textbf{\Large{Please answer ONLY
5 of the following
6 questions (Questions
30.1.1 through
30.1.6). }}}
 
Here are still some constants for use in the following questions:
 
 
\noindent\begin{tabular}{|l|l|l|}
\hline
Constant & Symbol & Value \\
\hline
 
Boltzmann's constant &
$k$ &
 $ 1.381 \times 10^{-23} $
J/K \\
\hline
 
Avogadro's number &
$N_A$ &
 $ 6.022 \times 10^{23} $
mol$^{-1}$ \\
\hline
 
Mass of electron &
$m_e$ &
 $ 9.1093897 \times 10^{-31} $
kg \\
\hline
 
\end{tabular}
 
   
\vspace{0.2in}
   
 In this big question of CHOOSE structure,           6 questions will be generat
 ed: 
  
  
            1(         11,         26)
 ,
            2(          6,         21)
 ,
            3(         12,         27)
 ,
            4(          8,         23)
 ,
            5(         10,         25)
 ,
            6(         13,         28)
 .
  
\vspace{0.2in}
  
{\textbf{\Large{Question
30.1.1 
 (          6,         11,         26)
}}}
  
  
 
 
\noindent\vspace{0.1in}{\textbf{\Large{Solution: }}}

Since the possiblity of  % 
smoking customer is $ a =  % 
.150 $,
and the possiblity of  % 
equal or above 30 years old customer is $ b =  % 
.3600 $,
the possiblity of  % 
non-smoking customer is $ c = 1.0 - a = 1.0 -
.150
=  % 
.850 $ and the possiblity of  % 
under 30 years old
customer is $ d = 1.0 - b = 1.0 -  % 
.3600 =  % 
.6400  $.
So the possibility of  % 
 non-smoking and  % 
under 30 years old
customer is $ c \times d =  % 
.544 $.
 
 
 
 
 
\noindent\vspace{0.05in}{\textbf{\Large{Answer:}}}

The possibility of  % 
 non-smoking and  % 
under 30 years old
customer is $ (1-a)(1-b) =  % 
.544 $.
 
 
  
\vspace{0.2in}
  
{\textbf{\Large{Question
30.1.2 
 (          6,          6,         21)
}}}
  
  
 
 
\noindent\vspace{0.05in}{\textbf{\Large{Answer:}}}

We will use the Newton's Second Law:
 
\[
\mathbf{f}=m\mathbf{a}.
\]
 
Since $\mathbf{f}=( % 
90.0,  % 
4.0,  % 
-8000.0 )N$
and $m= % 
56.0 kg$, bring them into the above equation, then we get
 
\begin{eqnarray*}
\mathbf{a}&=&\frac{\mathbf{f}}m  \\
&=&\frac{(
90.0 ,
4.0 ,
-8000.0 )N
}{ % 
56.0 kg}  \\
&=&(
1.6071 ,
7.1429 \times 10^{-2},
-142.86
)ms^{-2} \\
&=&(
20829. ,
925.71 ,
-1.8514 \times 10^{6}
)km/h^2.
\end{eqnarray*}
 
 
 
 
 
\noindent\vspace{0.1in}{\textbf{\Large{Solution: }}}

We will use the Newton's Second Law:
 
\[
\mathbf{f}=m\mathbf{a}.
\]
 
Since $\mathbf{f}=( % 
90.0,  % 
4.0,  % 
-8000.0 )N$
and $m= % 
56.0 kg$, bring them into the above equation, then we get
 
\begin{eqnarray*}
\mathbf{a}&=&\frac{\mathbf{f}}m  \\
&=&\frac{(
90.0 ,
4.0 ,
-8000.0 )N
}{ % 
56.0 kg}  \\
&=&(
1.6071 ,
7.1429 \times 10^{-2},
-142.86
)ms^{-2} \\
&=&(
20829. ,
925.71 ,
-1.8514 \times 10^{6}
)km/h^2.
\end{eqnarray*}
 
 
 
  
\vspace{0.2in}
  
{\textbf{\Large{Question
30.1.3 
 (          6,         12,         27)
}}}
  
  
 
 
\noindent\vspace{0.1in}{\textbf{\Large{Solution: }}}

Since the possiblity of  % 
smoking customer is $ a =  % 
.520 $,
and the possiblity of  % 
equal-or-above 30 years old customer is $ b =  % 
.2600 $,
the possiblity of  % 
non-smoking customer is $ c = 1.0 - a = 1.0 -
.520
=  % 
.480 $ and the possiblity of  % 
under 30 years old
customer is $ d = 1.0 - b = 1.0 -  % 
.2600 =  % 
.7400  $.
Then
 
\noindent
\begin{tabular}{|l|l|}
\hline
Customer & Possibility \\
\hline
smoking  and  % 
equal-or-above 30 years old  &
  $ % 
.520 \times  % 
.2600 =  % 
.135$ \\
\hline
smoking  and  % 
under 30 years old &
  $ % 
.520 \times  % 
.7400 =  % 
.385$ \\
\hline
 non-smoking and  % 
equal-or-above 30 years old  &
  $ % 
.480 \times  % 
.2600 =  % 
.125$ \\
\hline
 non-smoking and  % 
under 30 years old &
  $ % 
.480 \times  % 
.7400 =  % 
.355$ \\
\hline
\end{tabular}
 
\noindent
And the total summation of all possibilities is $  % 
1.000 $.
 
 
 
 
 
 
\noindent\vspace{0.05in}{\textbf{\Large{Answer:}}}

 
\noindent
\begin{tabular}{|l|l|}
\hline
Customer & Possibility \\
\hline
smoking  and  % 
equal-or-above 30 years old &
  $ % 
.135$ \\
\hline
smoking  and  % 
under 30 years old &
  $ % 
.385$ \\
\hline
 non-smoking and  % 
equal-or-above 30 years old &
  $ % 
.125$ \\
\hline
 non-smoking and  % 
under 30 years old &
  $ % 
.355$ \\
\hline
\end{tabular}
 
\noindent
 And the total summation of all possibilities is $  % 
1.000 $.
 
 
 
  
\vspace{0.2in}
  
{\textbf{\Large{Question
30.1.4 
 (          6,          8,         23)
}}}
  
  
 
 
\noindent\vspace{0.05in}{\textbf{\Large{Auto-answer:}}}
 
 
\noindent{\textbf{\large{
B.}}}
The accelaration is
$(
.92593ms^{-2},
.12963ms^{-2},
-1.2000 \times 10^{6}km/h^2
).
$
 
 
 
 
 
 
\noindent\vspace{0.1in}{\textbf{\Large{Solution: }}}

We will use the Newton's Second Law:
 
\[
\mathbf{f}=m\mathbf{a}.
\]
 
Since $\mathbf{f}=( % 
50.0,  % 
7.0,  % 
-5000.0 )N$
and $m= % 
54.0kg$, bring them into the above equation, then we get
 
\begin{eqnarray*}
\mathbf{a}&=&\frac{\mathbf{f}}m  \\
&=&\frac{(
50.0 ,
7.0 ,
-5000.0 )N
}{ % 
54.0 kg}  \\
&=&(
.92593 ,
.12963,
-92.593
)ms^{-2} \\
&=&(
12000. ,
1680.0 ,
-1.2000 \times 10^{6}
)km/h^2.
\end{eqnarray*}
 
 
 
  
\vspace{0.2in}
  
{\textbf{\Large{Question
30.1.5 
 (          6,         10,         25)
}}}
  
  
 
 
\noindent\vspace{0.05in}{\textbf{\Large{Auto-answer:}}}
 
 
\noindent{\textbf{\large{
C.}}}
A truck
 
 
\noindent{\textbf{\large{
D.}}}
An airplane
 
 
 
 
  
\vspace{0.2in}
  
{\textbf{\Large{Question
30.1.6 
 (          6,         13,         28)
}}}
  
  
 
 
\noindent\vspace{0.05in}{\textbf{\Large{Answer:}}}

5;
 
2;
 
The operation is  % 
ADDITION and the result is
$ % 
7.0000$.
 
 
 
   
   
\vspace{0.3in}
{\textbf{\LARGE{You have done all the above? A very good beginning, please go ahead.}}}
More constants the
Mass of electron
$m_e$$ =
9.109390 \times 10^{-31} $
kg
,
Universal gas constant
$R$$ =
8.315 $
J/(mol$\cdot $K)
,
$e$$ =
1.60217733 \times 10^{-19} $
C
, and
$m_p$$ =
1.6726231 \times 10^{-27} $
kg
%
may be very helpful.
\vspace{0.3in}
   
   
  
\vspace{0.2in}
  
{\textbf{\Large{QUESTION
30.2 
 (          4,          4,          4)
}}}
  
  
 
 
\noindent\vspace{0.05in}{\textbf{\Large{Auto-answer:}}}
  
  
\begin{tabular}{|l|l|l|}
 \hline
 Column Left & Column Right  & Answers       \\ 
 \hline
{\textbf{\large{
A.}}}
C
  & 
YJH
 & 
{\textbf{\large{
D.}}}
 \\ 
 \hline
{\textbf{\large{
B.}}}
er
  & 
ER
 & 
{\textbf{\large{
B.}}}
, 
{\textbf{\large{
C.}}}
 \\ 
 \hline
{\textbf{\large{
C.}}}
Er
  & 
c
 & 
{\textbf{\large{
A.}}}
 \\ 
 \hline
{\textbf{\large{
D.}}}
yjh
  & 
 a= %
3
 & 
{\textbf{\large{
E.}}}
 \\ 
 \hline
{\textbf{\large{
E.}}}
 A= %
6/ %
2

  & 
eR
 & 
{\textbf{\large{
B.}}}
, 
{\textbf{\large{
C.}}}
 \\ 
 \hline
 \end{tabular}
  
  
 
 
 
 
  
\vspace{0.2in}
  
{\textbf{\Large{QUESTION
30.3 
 (          3,          3,          3)
}}}
  
  
 
 
\noindent\vspace{0.05in}{\textbf{\Large{Auto-answer:}}}
 
 
\noindent{\textbf{\large{
B.}}}
Canada has  %
10 provinces and  %
3 territories.
 
 
 
 
  
\vspace{0.2in}
  
{\textbf{\Large{QUESTION
30.4 
 (          1,          1,          1)
}}}
  
  


 
 
\noindent\vspace{0.05in}{\textbf{\Large{Auto-answer:}}}
 
 
\noindent{\textbf{\large{
E.}}}
The accelaration is $  %
(
.536,
.14,
-125.00)
ms^{-2} $.
 
 
 
 
 
 
\noindent\vspace{0.05in}{\textbf{\Large{Answer:}}}

The correct answer from the choices is


\noindent{\textbf{\large{
E.}}}
The accelaration is $  %
(
.536,
.14,
-125.00)
ms^{-2} $.
 
 
 
 
 
\noindent\vspace{0.1in}{\textbf{\Large{Solution: }}}

We will use the Newton's Second Law:
 
\[
\mathbf{f}=m\mathbf{a}.
\]
 
Since $\mathbf{f}= % 
(30.0 , 8.0 , -7000.0) N$
and $m= % 
56.0000kg$, bring them into the above equation, then we get
 
\begin{eqnarray*}
\mathbf{a}&=&\frac{\mathbf{f}}m  \\
&=&\frac{ % 
(30.0 , 8.0 , -7000.0) N}{ % 
56.0000kg}  \\
&=& % 
(.536 , .14 , -125.00) ms^{-2}
\end{eqnarray*}
 
 
 
  
\vspace{0.2in}
  
{\textbf{\Large{QUESTION
30.5 
 (          5,          5,          5)
}}}
  
  
 
 
\noindent\vspace{0.05in}{\textbf{\Large{Answer:}}}

 
\noindent\begin{tabular}{|l|l|}\hline The correct & \\
          answer &  % 
$T$ \\ \hline \end{tabular}
1. $ % 
28$ is an  % 
even number.
 
\noindent\begin{tabular}{|l|l|}\hline The correct & \\
          answer &  % 
$T$ \\ \hline \end{tabular}
2.  % 
Montreal is in  % 
Quebec province.
 
\noindent\begin{tabular}{|l|l|}\hline The correct & \\
          answer &  % 
$T$ \\ \hline \end{tabular}
3.  % 
$\mathbf{F}=m\mathbf{a}$ is a mathmatical form of  % 
the Newton's Second Law.
 
 
 
  
\vspace{0.2in}
  
{\textbf{\Large{QUESTION
30.6 
 (          2,          2,          2)
}}}
  
  
 
 
\noindent\vspace{0.05in}{\textbf{\Large{Auto-answer:}}}
 
 
\noindent{\textbf{\large{
B.}}}
The accelaration is
$(
1.4815ms^{-2},
1200.0km/h^2,
-166.67ms^{-2}
).
$
 
 
 
 
 
 
\noindent\vspace{0.1in}{\textbf{\Large{Solution: }}}

We will use the Newton's Second Law:
 
\[
\mathbf{f}=m\mathbf{a}.
\]
 
Since $\mathbf{f}=( % 
80.000,  % 
5.0000,  % 
-9000.0 )N$
and $m= % 
54.0000kg$, bring them into the above equation, then we get
 
\begin{eqnarray*}
\mathbf{a}&=&\frac{\mathbf{f}}m  \\
&=&\frac{(
80.000 ,
5.0000 ,
-9000.0 )N
}{ % 
54.0000 kg}  \\
&=&(
1.4815 ,
9.2593 \times 10^{-2},
-166.67
)ms^{-2} \\
&=&(
19200. ,
1200.0 ,
-2.1600 \times 10^{6}
)km/h^2.
\end{eqnarray*}
 
 
 
   
   
\vspace{0.3in}
{\textbf{\LARGE{You have done all the above? Excellent! Not much left, please continue.}}}
\vspace{0.3in}
   
   
  
\vspace{0.2in}
  
{\textbf{\Large{QUESTION
30.7 
 (          8,         15,         60)
}}}
  
  
 
 
\noindent\vspace{0.05in}{\textbf{\Large{Answer:}}}

 
$\left( \begin{array}{ccccccccccccccc}
           7 & 
           4 & 
           5 & 
           7 \\ 
           4 & 
           5 & 
           6 & 
           4 \\ 
           7 & 
           5 & 
           5 & 
           7
\end{array}\right) \times
\left( \begin{array}{c}
           2 \\ 
           2 \\ 
           2 \\ 
           2
\end{array}\right)  =
\left( \begin{array}{c}
          46 \\ 
          38 \\ 
          48
\end{array}\right)  $
 
$  % 
 \left( \begin{array}
 {
 c
 c
 }
 \rho & 
 \beta \\ 
                    \zeta & 
 \Theta \\ 
 \Lambda & 
 \Psi \\ 
 \Gamma & 
 \Gamma
 \end{array} \right)
 \left( \begin{array}
 {
 c
 }
 \beta \\ 
 \beta
 \end{array} \right)
=
  \left( \begin{array}
 {
 c
 }
 \rho \times  \beta   +  \beta \times  \beta \\ 
                    \zeta \times  \beta   +  \Theta \times  \beta \\ 
 \Lambda \times  \beta   +  \Psi \times  \beta \\ 
 \Gamma \times  \beta   +  \Gamma \times  \beta
 \end{array} \right)
$
 
 
 
 
 
\noindent\vspace{0.1in}{\textbf{\Large{Solution: }}}

 
 
  
\vspace{0.2in}
  
{\textbf{\Large{QUESTION
30.8 
 (          7,         14,         50)
}}}
  
  
 
 
\noindent\vspace{0.05in}{\textbf{\Large{Auto-answer:}}}
 
 
\noindent{\textbf{\large{
C.}}}
  The accelaration is $  %
(
1.67,
3.7 \times 10^{-2},
-111.11)
ms^{-2} $.
 
 
 
 
 
 
\noindent\vspace{0.1in}{\textbf{\Large{Solution: }}}

We will use the Newton's Second Law:
 
\[
\mathbf{f}=m\mathbf{a}.
\]
 
Since $\mathbf{f}= % 
(90.0 , 2.0 , -6000.0) N$
and $m= % 
54.0kg$, bring them into the above equation, then we get
 
\begin{eqnarray*}
\mathbf{a}&=&\frac{\mathbf{f}}m  \\
&=&\frac{ % 
(90.0 , 2.0 , -6000.0) N}{ % 
54.0kg}  \\
&=& % 
(1.67 , 3.7 \times 10^{-2} , -111.11) ms^{-2}
\end{eqnarray*}
 
 
 
  
\vspace{0.2in}
  
{\textbf{\Large{QUESTION
30.9 
 (          9,         16,         70)
}}}
  
  


 
 
\noindent\vspace{0.05in}{\textbf{\Large{Answer:}}}

5,  % 
-7
 
 
 
 
 
\noindent\vspace{0.1in}{\textbf{\Large{Solution: }}}

Roots to the equation
\begin{eqnarray*}
-15 \times x^2  % 
-30
                 \times x    % 
+  % 
525 =0
\end{eqnarray*}
are  % 
5 and  % 
-7 .
 
Let us verity  % 
5 first:
$  % 
-15 \times x^2  % 
-30
                 \times x    % 
+  % 
525
  = % 
-375+( % 
-150)+( % 
525)
  = % 
-525+( % 
525)
  = % 
0
$
 
Then verity  % 
-7:
$  % 
-15 \times x^2  % 
-30
                 \times x    % 
+  % 
525
  = % 
-735+( % 
210)+( % 
525)
  = % 
-525+( % 
525)
  = % 
0
$
 
 
 
   
   
 \vspace{0.2in}
Here are still some constants for use:
 
 
\noindent\begin{tabular}{|l|l|l|}
\hline
Constant & Symbol & Value \\
\hline
 
Mass of proton &
$m_p$ &
 $ 1.6726231 \times 10^{-27} $
kg \\
\hline
 
Boltzmann's constant &
$k$ &
 $ 1.381 \times 10^{-23} $
J/K \\
\hline
 
\end{tabular}
 
Thank you very much for answering these questions!
 
{\textbf{\large{Please be advised}}} that in this paper there are questions from
30.1 through
30.9.
And any one of them may contain more than one sub-question, thus the total number
of sub-questions here is around 14, of which
13 should be answered.
 
   
   
   
   
\vspace{1.0in} 
{\textbf{\large{ *** END OF PAPER, THANKS *** }}} 
   
   
\hspace{1.0in} By: 
         239(         26,          34)
   
   
   
   
\newpage 
\setcounter{page}{ 
    31001 } 
   
   
\noindent{\textbf{\large{THIS IS THE ANSWER AND SOLUTION FOR}}}
   
   
 {\textbf{ \Large{ PAPER NUMBER          31 }}}
   
   
\vspace{0.2in}
   
   
\markboth{Answer and solution NOT for examinees !!!{\today}}{Answer and solution NOT for examinees !!! {\today}}
   
   
   
   
 \vspace{0.2in}
 
 
{\Huge  THIS IS AN EXAMPLE OF}
 
{\Huge  PERSONALIZED TESTS. }
 
If needed, please use the following constants.
 
 
 
\noindent\begin{tabular}{|l|l|l|}
\hline
Constant & Symbol & Value \\
\hline
Acceleration due to earth's gravity &
$g$ &
 $ 9.80 $
m/s$^2$ \\
\hline
Avogadro's number &
$N_A$ &
 $ 6.0221367 \times 10^{23} $
mol$^{-1}$ \\
\hline
Boltzmann's constant &
$k$ &
 $ 1.380658 \times 10^{-23} $
J/K \\
\hline
Coulomb's constant &
$k$ &
 $ 8.99 \times 10^{9} $
N$\cdot $m$^2$/C$^2$ \\
\hline
Electron charge magnitiude &
$e$ &
 $ 1.60217733 \times 10^{-19} $
C \\
\hline
Permeability of free space &
$\mu _0$ &
 $ 1.25663706 \times 10^{-6} $
T$\cdot $m/A \\
\hline
Permittivity of free space &
$\epsilon _0$ &
 $ 8.854187817 \times 10^{-12} $
C$^2$/(N$\cdot $m$^2$) \\
\hline
Pi &
$\pi$ &
 $ 3.14159265 $
$ $ \\
\hline
Planck's constant &
$h$ &
 $ 6.6260755 \times 10^{-34} $
J$\cdot $s \\
\hline
Mass of electron &
$m_e$ &
 $ 9.1093897 \times 10^{-31} $
kg \\
\hline
\end{tabular}
 
 
\noindent\begin{tabular}{|l|l|l|}
\hline
Constant & Symbol & Value \\
\hline
Mass of neutron &
$m_n$ &
 $ 1.6749286 \times 10^{-27} $
kg \\
\hline
Mass of proton &
$m_p$ &
 $ 1.6726231 \times 10^{-27} $
kg \\
\hline
Speed of light in vacuum &
$c$ &
 $ 299792458. $
m/s \\
\hline
Universal gravitational constant &
$G$ &
 $ 6.67259 \times 10^{-11} $
N$\cdot $m$^2$/kg$^2$ \\
\hline
Universal gas constant &
$R$ &
 $ 8.314510 $
J/(mol$\cdot $K) \\
\hline
\end{tabular}
 
 
{\textbf{\large{Please be advised}}} that in this paper there are questions from
31.1 through
31.9.
And any one of them may contain more than one sub-question, thus the total number
of sub-questions here is around 14, of which
13 should be answered.
 
\vspace{0.3in}
 
 
   
   
   
\vspace{0.2in}
   
In this paper, big questions will be generated in the following order: 
   
   
            1(          6)
 ,
            2(          3)
 ,
            3(          4)
 ,
            4(          2)
 ,
            5(          5)
 ,
            6(          1)
 ,
            7(          8)
 ,
            8(          7)
 ,
            9(          9)
 .
  
\vspace{0.2in}
  
{\textbf{\Large{QUESTION
31.1 
 (          6)
}}}
  
  
 
{\textbf{\Large{Please answer ONLY
5 of the following
6 questions (Questions
31.1.1 through
31.1.6). }}}
 
Here are still some constants for use in the following questions:
 
 
\noindent\begin{tabular}{|l|l|l|}
\hline
Constant & Symbol & Value \\
\hline
 
Boltzmann's constant &
$k$ &
 $ 1.381 \times 10^{-23} $
J/K \\
\hline
 
Avogadro's number &
$N_A$ &
 $ 6.022 \times 10^{23} $
mol$^{-1}$ \\
\hline
 
Mass of electron &
$m_e$ &
 $ 9.1093897 \times 10^{-31} $
kg \\
\hline
 
\end{tabular}
 
   
\vspace{0.2in}
   
 In this big question of CHOOSE structure,           6 questions will be generat
 ed: 
  
  
            1(          9,         24)
 ,
            2(         13,         28)
 ,
            3(         11,         26)
 ,
            4(          7,         22)
 ,
            5(          8,         23)
 ,
            6(         12,         27)
 .
  
\vspace{0.2in}
  
{\textbf{\Large{Question
31.1.1 
 (          6,          9,         24)
}}}
  
  
 
 
\noindent\vspace{0.1in}{\textbf{\Large{Solution: }}}

By using Newton's Law of Universal Gravitation:
\[
F=G \frac{(Sun's \hspace{0.1in} mass) \times (Planet's \hspace{0.1in} mass)} { (distance)^2},
\]
where
$ G= % 
6.67 \times 10^{-11}N m^{2}(kg)^{-2}$ , the forces can be easily calculated as
 
\vspace{0.2in}
 
 
\begin{tabular}{|l|l|l|l|}
\hline
The Planet & Mass ($kg$) & Distanace from Sun ($m$) & The Force ($N$)\\
\hline
Mercury  &
           $ % 
7.00000000 \times 10^{24} $   &
             $ % 
5.000000000 \times 10^{24} $    & $ % 
9.34 \times 10^{-11} $
\\  \hline
Venus    &
           $  % 
2.00 \times 10^{24}  $     &
             $ % 
6.00 \times 10^{24} $    & $ % 
1.85 \times 10^{-11} $
\\  \hline
Earth    &
           $  % 
9.00 \times 10^{24}  $     &
             $ % 
6.00 \times 10^{24} $    & $ % 
8.34 \times 10^{-11} $
\\   \hline
Mars     &
           $  % 
2.00 \times 10^{24} $     &
             $ % 
5.00 \times 10^{24} $    & $ % 
2.67 \times 10^{-11} $
\\   \hline
Jupiter  &
           $  % 
5.00 \times 10^{24} $    &
             $ % 
5.00 \times 10^{24} $    & $ % 
6.67 \times 10^{-11} $
\\  \hline
Saturn   &
           $  % 
4.00 \times 10^{24} $    &
             $ % 
2.00 \times 10^{24}  $    & $ % 
3.33 \times 10^{-10} $
\\  \hline
Uranus   &
           $  % 
7.00 \times 10^{24} $    &
             $ % 
2.00 \times 10^{24} $    & $ % 
5.84 \times 10^{-10} $
\\  \hline
Neptune  &
           $  % 
4.00 \times 10^{24} $    &
             $ % 
4.00 \times 10^{24} $    & $ % 
8.34 \times 10^{-11} $
\\  \hline
 
\end{tabular}
 
 
 
 
 
 
\noindent\vspace{0.05in}{\textbf{\Large{Answer:}}}

By using Newton's Law of Universal Gravitation:
\[
F=G \frac{(Sun's \hspace{0.1in} mass) \times (Planet's \hspace{0.1in} mass)} { (distance)^2},
\]
where
$ G= % 
6.67 \times 10^{-11} N m^{2}(kg)^{-2}$ , the forces can be easily calculated as
 
\vspace{0.2in}
 
 
\begin{tabular}{|l|l|l|l|}
\hline
The Planet & Mass ($kg$) & Distanace from Sun ($m$) & The Force ($N$)\\
\hline
Mercury  &
           $ % 
7.00000000 \times 10^{24}  $   &
             $ % 
5.000000000 \times 10^{24}$    & $ % 
9.34 \times 10^{-11} $
\\  \hline
Venus    &
           $  % 
2.00 \times 10^{24}  $     &
             $ % 
6.00 \times 10^{24} $    & $ % 
1.85 \times 10^{-11} $
\\  \hline
Earth    &
           $  % 
9.00 \times 10^{24}$     &
             $ % 
6.00 \times 10^{24} $    & $ % 
8.34 \times 10^{-11} $
\\   \hline
Mars     &
           $  % 
2.00 \times 10^{24} $     &
             $ % 
5.00 \times 10^{24}$    & $ % 
2.67 \times 10^{-11} $
\\   \hline
Jupiter  &
           $  % 
5.00 \times 10^{24}  $    &
             $ % 
5.00 \times 10^{24} $    & $ % 
6.67 \times 10^{-11}3 $
\\  \hline
Saturn   &
           $  % 
4.00 \times 10^{24}   $    &
             $ % 
2.00 \times 10^{24}  $    & $ % 
3.33 \times 10^{-10} $
\\  \hline
Uranus   &
           $  % 
7.00 \times 10^{24} $    &
             $ % 
2.00 \times 10^{24}$    & $ % 
5.84 \times 10^{-10} $
\\  \hline
Neptune  &
           $  % 
4.00 \times 10^{24}  $    &
             $ % 
4.00 \times 10^{24} $    & $ % 
8.34 \times 10^{-11} $
\\  \hline
 
\end{tabular}
 
 
 
 
  
\vspace{0.2in}
  
{\textbf{\Large{Question
31.1.2 
 (          6,         13,         28)
}}}
  
  
 
 
\noindent\vspace{0.05in}{\textbf{\Large{Answer:}}}

7;
 
2;
 
The operation is  % 
SUBTRACTION and the result is
$ % 
5.0000$.
 
 
 
  
\vspace{0.2in}
  
{\textbf{\Large{Question
31.1.3 
 (          6,         11,         26)
}}}
  
  
 
 
\noindent\vspace{0.1in}{\textbf{\Large{Solution: }}}

Since the possiblity of  % 
smoking customer is $ a =  % 
.970 $,
and the possiblity of  % 
equal or above 30 years old customer is $ b =  % 
6.00 \times 10^{-2} $,
the possiblity of  % 
non-smoking customer is $ c = 1.0 - a = 1.0 -
.970
=  % 
3.00 \times 10^{-2} $ and the possiblity of  % 
under 30 years old
customer is $ d = 1.0 - b = 1.0 -  % 
6.00 \times 10^{-2} =  % 
.9400  $.
So the possibility of  % 
 non-smoking and  % 
under 30 years old
customer is $ c \times d =  % 
2.82 \times 10^{-2} $.
 
 
 
 
 
\noindent\vspace{0.05in}{\textbf{\Large{Answer:}}}

The possibility of  % 
 non-smoking and  % 
under 30 years old
customer is $ (1-a)(1-b) =  % 
2.82 \times 10^{-2} $.
 
 
  
\vspace{0.2in}
  
{\textbf{\Large{Question
31.1.4 
 (          6,          7,         22)
}}}
  
  
 
 
\noindent\vspace{0.05in}{\textbf{\Large{Auto-answer:}}}
 
 
\noindent{\textbf{\large{
D.}}}
The accelaration (vector) is
$(
8937.9,
1787.6 ,
-446897.
)km/h^2.
$
 
 
 
 
 
 
\noindent\vspace{0.1in}{\textbf{\Large{Solution: }}}

We will use the Newton's Second Law:
 
\[
\mathbf{f}=m\mathbf{a}.
\]
 
Since $\mathbf{f}=( % 
40.0,  % 
8.0,  % 
-2000.0 )N$
and $m= % 
58.0 kg$, bring them into the above equation, then we get
 
\begin{eqnarray*}
\mathbf{a}&=&\frac{\mathbf{f}}m  \\
&=&\frac{(
40.0 ,
8.0 ,
-2000.0 )N
}{ % 
58.0 kg}  \\
&=&(
.68966 ,
.13793,
-34.483
)ms^{-2} \\
&=&(
8937.9 ,
1787.6 ,
-446897.
)km/h^2.
\end{eqnarray*}
 
 
 
  
\vspace{0.2in}
  
{\textbf{\Large{Question
31.1.5 
 (          6,          8,         23)
}}}
  
  
 
 
\noindent\vspace{0.05in}{\textbf{\Large{Auto-answer:}}}
 
 
\noindent{\textbf{\large{
A.}}}
The accelaration is
$(
1.7308ms^{-2},
.17308ms^{-2},
-747692.km/h^2
).
$
 
 
 
 
 
 
\noindent\vspace{0.1in}{\textbf{\Large{Solution: }}}

We will use the Newton's Second Law:
 
\[
\mathbf{f}=m\mathbf{a}.
\]
 
Since $\mathbf{f}=( % 
90.0,  % 
9.0,  % 
-3000.0 )N$
and $m= % 
52.0kg$, bring them into the above equation, then we get
 
\begin{eqnarray*}
\mathbf{a}&=&\frac{\mathbf{f}}m  \\
&=&\frac{(
90.0 ,
9.0 ,
-3000.0 )N
}{ % 
52.0 kg}  \\
&=&(
1.7308 ,
.17308,
-57.692
)ms^{-2} \\
&=&(
22431. ,
2243.1 ,
-747692.
)km/h^2.
\end{eqnarray*}
 
 
 
  
\vspace{0.2in}
  
{\textbf{\Large{Question
31.1.6 
 (          6,         12,         27)
}}}
  
  
 
 
\noindent\vspace{0.1in}{\textbf{\Large{Solution: }}}

Since the possiblity of  % 
smoking customer is $ a =  % 
.470 $,
and the possiblity of  % 
equal-or-above 30 years old customer is $ b =  % 
.1600 $,
the possiblity of  % 
non-smoking customer is $ c = 1.0 - a = 1.0 -
.470
=  % 
.530 $ and the possiblity of  % 
under 30 years old
customer is $ d = 1.0 - b = 1.0 -  % 
.1600 =  % 
.8400  $.
Then
 
\noindent
\begin{tabular}{|l|l|}
\hline
Customer & Possibility \\
\hline
smoking  and  % 
equal-or-above 30 years old  &
  $ % 
.470 \times  % 
.1600 =  % 
7.52 \times 10^{-2}$ \\
\hline
smoking  and  % 
under 30 years old &
  $ % 
.470 \times  % 
.8400 =  % 
.395$ \\
\hline
 non-smoking and  % 
equal-or-above 30 years old  &
  $ % 
.530 \times  % 
.1600 =  % 
8.48 \times 10^{-2}$ \\
\hline
 non-smoking and  % 
under 30 years old &
  $ % 
.530 \times  % 
.8400 =  % 
.445$ \\
\hline
\end{tabular}
 
\noindent
And the total summation of all possibilities is $  % 
1.000 $.
 
 
 
 
 
 
\noindent\vspace{0.05in}{\textbf{\Large{Answer:}}}

 
\noindent
\begin{tabular}{|l|l|}
\hline
Customer & Possibility \\
\hline
smoking  and  % 
equal-or-above 30 years old &
  $ % 
7.52 \times 10^{-2}$ \\
\hline
smoking  and  % 
under 30 years old &
  $ % 
.395$ \\
\hline
 non-smoking and  % 
equal-or-above 30 years old &
  $ % 
8.48 \times 10^{-2}$ \\
\hline
 non-smoking and  % 
under 30 years old &
  $ % 
.445$ \\
\hline
\end{tabular}
 
\noindent
 And the total summation of all possibilities is $  % 
1.000 $.
 
 
 
   
   
\vspace{0.3in}
{\textbf{\LARGE{You have done all the above? A very good beginning, please go ahead.}}}
More constants the
Mass of electron
$m_e$$ =
9.109390 \times 10^{-31} $
kg
,
Universal gas constant
$R$$ =
8.315 $
J/(mol$\cdot $K)
,
$e$$ =
1.60217733 \times 10^{-19} $
C
, and
$m_p$$ =
1.6726231 \times 10^{-27} $
kg
%
may be very helpful.
\vspace{0.3in}
   
   
  
\vspace{0.2in}
  
{\textbf{\Large{QUESTION
31.2 
 (          3,          3,          3)
}}}
  
  
 
 
\noindent\vspace{0.05in}{\textbf{\Large{Auto-answer:}}}
 
 
\noindent{\textbf{\large{
D.}}}
Canada has  %
10 provinces and  %
3 territories.
 
 
 
 
  
\vspace{0.2in}
  
{\textbf{\Large{QUESTION
31.3 
 (          4,          4,          4)
}}}
  
  
 
 
\noindent\vspace{0.05in}{\textbf{\Large{Auto-answer:}}}
  
  
\begin{tabular}{|l|l|l|}
 \hline
 Column Left & Column Right  & Answers       \\ 
 \hline
{\textbf{\large{
A.}}}
yjh
  & 
b
 & 
{\textbf{\large{
B.}}}
 \\ 
 \hline
{\textbf{\large{
B.}}}
B
  & 
ER
 & 
{\textbf{\large{
C.}}}
 \\ 
 \hline
{\textbf{\large{
C.}}}
Er
  & 
 a= %
2
 & 
{\textbf{\large{
E.}}}
 \\ 
 \hline
{\textbf{\large{
D.}}}
A
  & 
YJH
 & 
{\textbf{\large{
A.}}}
 \\ 
 \hline
{\textbf{\large{
E.}}}
 A= %
4/ %
2

  & 
a
 & 
{\textbf{\large{
D.}}}
 \\ 
 \hline
 \end{tabular}
  
  
 
 
 
 
  
\vspace{0.2in}
  
{\textbf{\Large{QUESTION
31.4 
 (          2,          2,          2)
}}}
  
  
 
 
\noindent\vspace{0.05in}{\textbf{\Large{Auto-answer:}}}
 
 
\noindent{\textbf{\large{
B.}}}
The accelaration is
$(
1.2000ms^{-2},
1296.0km/h^2,
-120.00ms^{-2}
).
$
 
 
 
 
 
 
\noindent\vspace{0.1in}{\textbf{\Large{Solution: }}}

We will use the Newton's Second Law:
 
\[
\mathbf{f}=m\mathbf{a}.
\]
 
Since $\mathbf{f}=( % 
60.000,  % 
5.0000,  % 
-6000.0 )N$
and $m= % 
50.0000kg$, bring them into the above equation, then we get
 
\begin{eqnarray*}
\mathbf{a}&=&\frac{\mathbf{f}}m  \\
&=&\frac{(
60.000 ,
5.0000 ,
-6000.0 )N
}{ % 
50.0000 kg}  \\
&=&(
1.2000 ,
.10000,
-120.00
)ms^{-2} \\
&=&(
15552. ,
1296.0 ,
-1.5552 \times 10^{6}
)km/h^2.
\end{eqnarray*}
 
 
 
  
\vspace{0.2in}
  
{\textbf{\Large{QUESTION
31.5 
 (          5,          5,          5)
}}}
  
  
 
 
\noindent\vspace{0.05in}{\textbf{\Large{Answer:}}}

 
\noindent\begin{tabular}{|l|l|}\hline The correct & \\
          answer &  % 
$F$ \\ \hline \end{tabular}
1. $ % 
37$ is an  % 
even number.
 
\noindent\begin{tabular}{|l|l|}\hline The correct & \\
          answer &  % 
$F$ \\ \hline \end{tabular}
2.  % 
Hull is in  % 
Ontario province.
 
\noindent\begin{tabular}{|l|l|}\hline The correct & \\
          answer &  % 
$F$ \\ \hline \end{tabular}
3.  % 
$\mathbf{F}=m\mathbf{a}$ is a mathmatical form of  % 
Newton's Law of Universal Gravitation.
 
 
 
  
\vspace{0.2in}
  
{\textbf{\Large{QUESTION
31.6 
 (          1,          1,          1)
}}}
  
  


 
 
\noindent\vspace{0.05in}{\textbf{\Large{Auto-answer:}}}
 
 
\noindent{\textbf{\large{
E.}}}
The accelaration is $  %
(
.893,
8.9 \times 10^{-2},
-160.71)
ms^{-2} $.
 
 
 
 
 
 
\noindent\vspace{0.05in}{\textbf{\Large{Answer:}}}

The correct answer from the choices is


\noindent{\textbf{\large{
E.}}}
The accelaration is $  %
(
.893,
8.9 \times 10^{-2},
-160.71)
ms^{-2} $.
 
 
 
 
 
\noindent\vspace{0.1in}{\textbf{\Large{Solution: }}}

We will use the Newton's Second Law:
 
\[
\mathbf{f}=m\mathbf{a}.
\]
 
Since $\mathbf{f}= % 
(50.0 , 5.0 , -9000.0) N$
and $m= % 
56.0000kg$, bring them into the above equation, then we get
 
\begin{eqnarray*}
\mathbf{a}&=&\frac{\mathbf{f}}m  \\
&=&\frac{ % 
(50.0 , 5.0 , -9000.0) N}{ % 
56.0000kg}  \\
&=& % 
(.893 , 8.9 \times 10^{-2} , -160.71) ms^{-2}
\end{eqnarray*}
 
 
 
   
   
\vspace{0.3in}
{\textbf{\LARGE{You have done all the above? Excellent! Not much left, please continue.}}}
\vspace{0.3in}
   
   
  
\vspace{0.2in}
  
{\textbf{\Large{QUESTION
31.7 
 (          8,         15,         60)
}}}
  
  
 
 
\noindent\vspace{0.05in}{\textbf{\Large{Answer:}}}

 
$\left( \begin{array}{ccccccccccccccc}
           4 & 
           6 & 
           5 & 
           6 \\ 
           5 & 
           4 & 
           5 & 
           6 \\ 
           6 & 
           5 & 
           5 & 
           5
\end{array}\right) \times
\left( \begin{array}{c}
           2 \\ 
           2 \\ 
           2 \\ 
           2
\end{array}\right)  =
\left( \begin{array}{c}
          42 \\ 
          40 \\ 
          42
\end{array}\right)  $
 
$  % 
 \left( \begin{array}
 {
 c
 c
 }
 \Phi & 
 \gamma \\ 
 \Upsilon & 
 \Upsilon \\ 
 \beta & 
                    \zeta \\ 
 \Lambda & 
 \Delta
 \end{array} \right)
 \left( \begin{array}
 {
 c
 }
 \gamma \\ 
 \beta
 \end{array} \right)
=
  \left( \begin{array}
 {
 c
 }
 \Phi \times  \gamma   +  \gamma \times  \beta \\ 
 \Upsilon \times  \gamma   +  \Upsilon \times  \beta \\ 
 \beta \times  \gamma   +                     \zeta \times  \beta \\ 
 \Lambda \times  \gamma   +  \Delta \times  \beta
 \end{array} \right)
$
 
 
 
 
 
\noindent\vspace{0.1in}{\textbf{\Large{Solution: }}}

 
 
  
\vspace{0.2in}
  
{\textbf{\Large{QUESTION
31.8 
 (          7,         14,         50)
}}}
  
  
 
 
\noindent\vspace{0.05in}{\textbf{\Large{Auto-answer:}}}
 
 
\noindent{\textbf{\large{
C.}}}
  The accelaration is $  %
(
.862,
8.6 \times 10^{-2},
-51.724)
ms^{-2} $.
 
 
 
 
 
 
\noindent\vspace{0.1in}{\textbf{\Large{Solution: }}}

We will use the Newton's Second Law:
 
\[
\mathbf{f}=m\mathbf{a}.
\]
 
Since $\mathbf{f}= % 
(50.0 , 5.0 , -3000.0) N$
and $m= % 
58.0kg$, bring them into the above equation, then we get
 
\begin{eqnarray*}
\mathbf{a}&=&\frac{\mathbf{f}}m  \\
&=&\frac{ % 
(50.0 , 5.0 , -3000.0) N}{ % 
58.0kg}  \\
&=& % 
(.862 , 8.6 \times 10^{-2} , -51.724) ms^{-2}
\end{eqnarray*}
 
 
 
  
\vspace{0.2in}
  
{\textbf{\Large{QUESTION
31.9 
 (          9,         16,         70)
}}}
  
  


 
 
\noindent\vspace{0.05in}{\textbf{\Large{Answer:}}}

-7,  % 
-1
 
 
 
 
 
\noindent\vspace{0.1in}{\textbf{\Large{Solution: }}}

Roots to the equation
\begin{eqnarray*}
9 \times x^2  % 
+  % 
72
                 \times x    % 
+  % 
63 =0
\end{eqnarray*}
are  % 
-7 and  % 
-1 .
 
Let us verity  % 
-7 first:
$  % 
9 \times x^2  % 
+  % 
72
                 \times x    % 
+  % 
63
  = % 
441+( % 
-504)+( % 
63)
  = % 
-63+( % 
63)
  = % 
0
$
 
Then verity  % 
-1:
$  % 
9 \times x^2  % 
+  % 
72
                 \times x    % 
+  % 
63
  = % 
9+( % 
-72)+( % 
63)
  = % 
-63+( % 
63)
  = % 
0
$
 
 
 
   
   
 \vspace{0.2in}
Here are still some constants for use:
 
 
\noindent\begin{tabular}{|l|l|l|}
\hline
Constant & Symbol & Value \\
\hline
 
Mass of proton &
$m_p$ &
 $ 1.6726231 \times 10^{-27} $
kg \\
\hline
 
Boltzmann's constant &
$k$ &
 $ 1.381 \times 10^{-23} $
J/K \\
\hline
 
\end{tabular}
 
Thank you very much for answering these questions!
 
{\textbf{\large{Please be advised}}} that in this paper there are questions from
31.1 through
31.9.
And any one of them may contain more than one sub-question, thus the total number
of sub-questions here is around 14, of which
13 should be answered.
 
   
   
   
   
\vspace{1.0in} 
{\textbf{\large{ *** END OF PAPER, THANKS *** }}} 
   
   
\hspace{1.0in} By: 
         239(         26,          34)
   
   
   
   
\newpage 
\setcounter{page}{ 
    32001 } 
   
   
\noindent{\textbf{\large{THIS IS THE ANSWER AND SOLUTION FOR}}}
   
   
 {\textbf{ \Large{ PAPER NUMBER          32 }}}
   
   
\vspace{0.2in}
   
   
\markboth{Answer and solution NOT for examinees !!!{\today}}{Answer and solution NOT for examinees !!! {\today}}
   
   
   
   
 \vspace{0.2in}
 
 
{\Huge  THIS IS AN EXAMPLE OF}
 
{\Huge  PERSONALIZED TESTS. }
 
If needed, please use the following constants.
 
 
 
\noindent\begin{tabular}{|l|l|l|}
\hline
Constant & Symbol & Value \\
\hline
Acceleration due to earth's gravity &
$g$ &
 $ 9.80 $
m/s$^2$ \\
\hline
Avogadro's number &
$N_A$ &
 $ 6.0221367 \times 10^{23} $
mol$^{-1}$ \\
\hline
Boltzmann's constant &
$k$ &
 $ 1.380658 \times 10^{-23} $
J/K \\
\hline
Coulomb's constant &
$k$ &
 $ 8.99 \times 10^{9} $
N$\cdot $m$^2$/C$^2$ \\
\hline
Electron charge magnitiude &
$e$ &
 $ 1.60217733 \times 10^{-19} $
C \\
\hline
Permeability of free space &
$\mu _0$ &
 $ 1.25663706 \times 10^{-6} $
T$\cdot $m/A \\
\hline
Permittivity of free space &
$\epsilon _0$ &
 $ 8.854187817 \times 10^{-12} $
C$^2$/(N$\cdot $m$^2$) \\
\hline
Pi &
$\pi$ &
 $ 3.14159265 $
$ $ \\
\hline
Planck's constant &
$h$ &
 $ 6.6260755 \times 10^{-34} $
J$\cdot $s \\
\hline
Mass of electron &
$m_e$ &
 $ 9.1093897 \times 10^{-31} $
kg \\
\hline
\end{tabular}
 
 
\noindent\begin{tabular}{|l|l|l|}
\hline
Constant & Symbol & Value \\
\hline
Mass of neutron &
$m_n$ &
 $ 1.6749286 \times 10^{-27} $
kg \\
\hline
Mass of proton &
$m_p$ &
 $ 1.6726231 \times 10^{-27} $
kg \\
\hline
Speed of light in vacuum &
$c$ &
 $ 299792458. $
m/s \\
\hline
Universal gravitational constant &
$G$ &
 $ 6.67259 \times 10^{-11} $
N$\cdot $m$^2$/kg$^2$ \\
\hline
Universal gas constant &
$R$ &
 $ 8.314510 $
J/(mol$\cdot $K) \\
\hline
\end{tabular}
 
 
{\textbf{\large{Please be advised}}} that in this paper there are questions from
32.1 through
32.9.
And any one of them may contain more than one sub-question, thus the total number
of sub-questions here is around 14, of which
13 should be answered.
 
\vspace{0.3in}
 
 
   
   
   
\vspace{0.2in}
   
In this paper, big questions will be generated in the following order: 
   
   
            1(          6)
 ,
            2(          5)
 ,
            3(          4)
 ,
            4(          2)
 ,
            5(          3)
 ,
            6(          1)
 ,
            7(          8)
 ,
            8(          7)
 ,
            9(          9)
 .
  
\vspace{0.2in}
  
{\textbf{\Large{QUESTION
32.1 
 (          6)
}}}
  
  
 
{\textbf{\Large{Please answer ONLY
5 of the following
6 questions (Questions
32.1.1 through
32.1.6). }}}
 
Here are still some constants for use in the following questions:
 
 
\noindent\begin{tabular}{|l|l|l|}
\hline
Constant & Symbol & Value \\
\hline
 
Boltzmann's constant &
$k$ &
 $ 1.381 \times 10^{-23} $
J/K \\
\hline
 
Avogadro's number &
$N_A$ &
 $ 6.022 \times 10^{23} $
mol$^{-1}$ \\
\hline
 
Mass of electron &
$m_e$ &
 $ 9.1093897 \times 10^{-31} $
kg \\
\hline
 
\end{tabular}
 
   
\vspace{0.2in}
   
 In this big question of CHOOSE structure,           6 questions will be generat
 ed: 
  
  
            1(         12,         27)
 ,
            2(          8,         23)
 ,
            3(          9,         24)
 ,
            4(          7,         22)
 ,
            5(         10,         25)
 ,
            6(          6,         21)
 .
  
\vspace{0.2in}
  
{\textbf{\Large{Question
32.1.1 
 (          6,         12,         27)
}}}
  
  
 
 
\noindent\vspace{0.1in}{\textbf{\Large{Solution: }}}

Since the possiblity of  % 
 non-smoking customer is $ a =  % 
.460 $,
and the possiblity of  % 
equal-or-above 30 years old customer is $ b =  % 
.7000 $,
the possiblity of  % 
smoking customer is $ c = 1.0 - a = 1.0 -
.460
=  % 
.540 $ and the possiblity of  % 
under 30 years old
customer is $ d = 1.0 - b = 1.0 -  % 
.7000 =  % 
.3000  $.
Then
 
\noindent
\begin{tabular}{|l|l|}
\hline
Customer & Possibility \\
\hline
smoking  and  % 
equal-or-above 30 years old  &
  $ % 
.540 \times  % 
.7000 =  % 
.378$ \\
\hline
smoking  and  % 
under 30 years old &
  $ % 
.540 \times  % 
.3000 =  % 
.162$ \\
\hline
 non-smoking and  % 
equal-or-above 30 years old  &
  $ % 
.460 \times  % 
.7000 =  % 
.322$ \\
\hline
 non-smoking and  % 
under 30 years old &
  $ % 
.460 \times  % 
.3000 =  % 
.138$ \\
\hline
\end{tabular}
 
\noindent
And the total summation of all possibilities is $  % 
1.000 $.
 
 
 
 
 
 
\noindent\vspace{0.05in}{\textbf{\Large{Answer:}}}

 
\noindent
\begin{tabular}{|l|l|}
\hline
Customer & Possibility \\
\hline
smoking  and  % 
equal-or-above 30 years old &
  $ % 
.378$ \\
\hline
smoking  and  % 
under 30 years old &
  $ % 
.162$ \\
\hline
 non-smoking and  % 
equal-or-above 30 years old &
  $ % 
.322$ \\
\hline
 non-smoking and  % 
under 30 years old &
  $ % 
.138$ \\
\hline
\end{tabular}
 
\noindent
 And the total summation of all possibilities is $  % 
1.000 $.
 
 
 
  
\vspace{0.2in}
  
{\textbf{\Large{Question
32.1.2 
 (          6,          8,         23)
}}}
  
  
 
 
\noindent\vspace{0.05in}{\textbf{\Large{Auto-answer:}}}
 
 
\noindent{\textbf{\large{
C.}}}
The accelaration is
$(
1.6667ms^{-2},
9.2593 \times 10^{-2}ms^{-2},
-1.2000 \times 10^{6}km/h^2
).
$
 
 
 
 
 
 
\noindent\vspace{0.1in}{\textbf{\Large{Solution: }}}

We will use the Newton's Second Law:
 
\[
\mathbf{f}=m\mathbf{a}.
\]
 
Since $\mathbf{f}=( % 
90.0,  % 
5.0,  % 
-5000.0 )N$
and $m= % 
54.0kg$, bring them into the above equation, then we get
 
\begin{eqnarray*}
\mathbf{a}&=&\frac{\mathbf{f}}m  \\
&=&\frac{(
90.0 ,
5.0 ,
-5000.0 )N
}{ % 
54.0 kg}  \\
&=&(
1.6667 ,
9.2593 \times 10^{-2},
-92.593
)ms^{-2} \\
&=&(
21600. ,
1200.0 ,
-1.2000 \times 10^{6}
)km/h^2.
\end{eqnarray*}
 
 
 
  
\vspace{0.2in}
  
{\textbf{\Large{Question
32.1.3 
 (          6,          9,         24)
}}}
  
  
 
 
\noindent\vspace{0.1in}{\textbf{\Large{Solution: }}}

By using Newton's Law of Universal Gravitation:
\[
F=G \frac{(Sun's \hspace{0.1in} mass) \times (Planet's \hspace{0.1in} mass)} { (distance)^2},
\]
where
$ G= % 
6.67 \times 10^{-11}N m^{2}(kg)^{-2}$ , the forces can be easily calculated as
 
\vspace{0.2in}
 
 
\begin{tabular}{|l|l|l|l|}
\hline
The Planet & Mass ($kg$) & Distanace from Sun ($m$) & The Force ($N$)\\
\hline
Mercury  &
           $ % 
2.00000000 \times 10^{24} $   &
             $ % 
6.000000000 \times 10^{24} $    & $ % 
2.59 \times 10^{-11} $
\\  \hline
Venus    &
           $  % 
6.00 \times 10^{24}  $     &
             $ % 
3.00 \times 10^{24} $    & $ % 
3.11 \times 10^{-10} $
\\  \hline
Earth    &
           $  % 
8.00 \times 10^{24}  $     &
             $ % 
5.00 \times 10^{24} $    & $ % 
1.49 \times 10^{-10} $
\\   \hline
Mars     &
           $  % 
5.00 \times 10^{24} $     &
             $ % 
2.00 \times 10^{24} $    & $ % 
5.84 \times 10^{-10} $
\\   \hline
Jupiter  &
           $  % 
3.00 \times 10^{24} $    &
             $ % 
9.00 \times 10^{24} $    & $ % 
1.73 \times 10^{-11} $
\\  \hline
Saturn   &
           $  % 
8.00 \times 10^{24} $    &
             $ % 
9.00 \times 10^{24}  $    & $ % 
4.61 \times 10^{-11} $
\\  \hline
Uranus   &
           $  % 
5.00 \times 10^{24} $    &
             $ % 
4.00 \times 10^{24} $    & $ % 
1.46 \times 10^{-10} $
\\  \hline
Neptune  &
           $  % 
3.00 \times 10^{24} $    &
             $ % 
8.00 \times 10^{24} $    & $ % 
2.19 \times 10^{-11} $
\\  \hline
 
\end{tabular}
 
 
 
 
 
 
\noindent\vspace{0.05in}{\textbf{\Large{Answer:}}}

By using Newton's Law of Universal Gravitation:
\[
F=G \frac{(Sun's \hspace{0.1in} mass) \times (Planet's \hspace{0.1in} mass)} { (distance)^2},
\]
where
$ G= % 
6.67 \times 10^{-11} N m^{2}(kg)^{-2}$ , the forces can be easily calculated as
 
\vspace{0.2in}
 
 
\begin{tabular}{|l|l|l|l|}
\hline
The Planet & Mass ($kg$) & Distanace from Sun ($m$) & The Force ($N$)\\
\hline
Mercury  &
           $ % 
2.00000000 \times 10^{24}  $   &
             $ % 
6.000000000 \times 10^{24}$    & $ % 
2.59 \times 10^{-11} $
\\  \hline
Venus    &
           $  % 
6.00 \times 10^{24}  $     &
             $ % 
3.00 \times 10^{24} $    & $ % 
3.11 \times 10^{-10} $
\\  \hline
Earth    &
           $  % 
8.00 \times 10^{24}$     &
             $ % 
5.00 \times 10^{24} $    & $ % 
1.49 \times 10^{-10} $
\\   \hline
Mars     &
           $  % 
5.00 \times 10^{24} $     &
             $ % 
2.00 \times 10^{24}$    & $ % 
5.84 \times 10^{-10} $
\\   \hline
Jupiter  &
           $  % 
3.00 \times 10^{24}  $    &
             $ % 
9.00 \times 10^{24} $    & $ % 
1.73 \times 10^{-11}3 $
\\  \hline
Saturn   &
           $  % 
8.00 \times 10^{24}   $    &
             $ % 
9.00 \times 10^{24}  $    & $ % 
4.61 \times 10^{-11} $
\\  \hline
Uranus   &
           $  % 
5.00 \times 10^{24} $    &
             $ % 
4.00 \times 10^{24}$    & $ % 
1.46 \times 10^{-10} $
\\  \hline
Neptune  &
           $  % 
3.00 \times 10^{24}  $    &
             $ % 
8.00 \times 10^{24} $    & $ % 
2.19 \times 10^{-11} $
\\  \hline
 
\end{tabular}
 
 
 
 
  
\vspace{0.2in}
  
{\textbf{\Large{Question
32.1.4 
 (          6,          7,         22)
}}}
  
  
 
 
\noindent\vspace{0.05in}{\textbf{\Large{Auto-answer:}}}
 
 
\noindent{\textbf{\large{
E.}}}
The accelaration (vector) is
$(
12960.,
1814.4 ,
-1.5552 \times 10^{6}
)km/h^2.
$
 
 
 
 
 
 
\noindent\vspace{0.1in}{\textbf{\Large{Solution: }}}

We will use the Newton's Second Law:
 
\[
\mathbf{f}=m\mathbf{a}.
\]
 
Since $\mathbf{f}=( % 
50.0,  % 
7.0,  % 
-6000.0 )N$
and $m= % 
50.0 kg$, bring them into the above equation, then we get
 
\begin{eqnarray*}
\mathbf{a}&=&\frac{\mathbf{f}}m  \\
&=&\frac{(
50.0 ,
7.0 ,
-6000.0 )N
}{ % 
50.0 kg}  \\
&=&(
1.0000 ,
.14000,
-120.00
)ms^{-2} \\
&=&(
12960. ,
1814.4 ,
-1.5552 \times 10^{6}
)km/h^2.
\end{eqnarray*}
 
 
 
  
\vspace{0.2in}
  
{\textbf{\Large{Question
32.1.5 
 (          6,         10,         25)
}}}
  
  
 
 
\noindent\vspace{0.05in}{\textbf{\Large{Auto-answer:}}}
 
 
\noindent{\textbf{\large{
A.}}}
A truck
 
 
\noindent{\textbf{\large{
C.}}}
An airplane
 
 
 
 
  
\vspace{0.2in}
  
{\textbf{\Large{Question
32.1.6 
 (          6,          6,         21)
}}}
  
  
 
 
\noindent\vspace{0.05in}{\textbf{\Large{Answer:}}}

We will use the Newton's Second Law:
 
\[
\mathbf{f}=m\mathbf{a}.
\]
 
Since $\mathbf{f}=( % 
50.0,  % 
5.0,  % 
-3000.0 )N$
and $m= % 
54.0 kg$, bring them into the above equation, then we get
 
\begin{eqnarray*}
\mathbf{a}&=&\frac{\mathbf{f}}m  \\
&=&\frac{(
50.0 ,
5.0 ,
-3000.0 )N
}{ % 
54.0 kg}  \\
&=&(
.92593 ,
9.2593 \times 10^{-2},
-55.556
)ms^{-2} \\
&=&(
12000. ,
1200.0 ,
-720000.
)km/h^2.
\end{eqnarray*}
 
 
 
 
 
\noindent\vspace{0.1in}{\textbf{\Large{Solution: }}}

We will use the Newton's Second Law:
 
\[
\mathbf{f}=m\mathbf{a}.
\]
 
Since $\mathbf{f}=( % 
50.0,  % 
5.0,  % 
-3000.0 )N$
and $m= % 
54.0 kg$, bring them into the above equation, then we get
 
\begin{eqnarray*}
\mathbf{a}&=&\frac{\mathbf{f}}m  \\
&=&\frac{(
50.0 ,
5.0 ,
-3000.0 )N
}{ % 
54.0 kg}  \\
&=&(
.92593 ,
9.2593 \times 10^{-2},
-55.556
)ms^{-2} \\
&=&(
12000. ,
1200.0 ,
-720000.
)km/h^2.
\end{eqnarray*}
 
 
 
   
   
\vspace{0.3in}
{\textbf{\LARGE{You have done all the above? A very good beginning, please go ahead.}}}
More constants the
Mass of electron
$m_e$$ =
9.109390 \times 10^{-31} $
kg
,
Universal gas constant
$R$$ =
8.315 $
J/(mol$\cdot $K)
,
$e$$ =
1.60217733 \times 10^{-19} $
C
, and
$m_p$$ =
1.6726231 \times 10^{-27} $
kg
%
may be very helpful.
\vspace{0.3in}
   
   
  
\vspace{0.2in}
  
{\textbf{\Large{QUESTION
32.2 
 (          5,          5,          5)
}}}
  
  
 
 
\noindent\vspace{0.05in}{\textbf{\Large{Answer:}}}

 
\noindent\begin{tabular}{|l|l|}\hline The correct & \\
          answer &  % 
$T$ \\ \hline \end{tabular}
1. $ % 
5$ is an  % 
odd number.
 
\noindent\begin{tabular}{|l|l|}\hline The correct & \\
          answer &  % 
$T$ \\ \hline \end{tabular}
2.  % 
Kingston is in  % 
Ontario province.
 
\noindent\begin{tabular}{|l|l|}\hline The correct & \\
          answer &  % 
$T$ \\ \hline \end{tabular}
3.  % 
$\mathbf{F}=m\mathbf{a}$ is a mathmatical form of  % 
the Newton's Second Law.
 
 
 
  
\vspace{0.2in}
  
{\textbf{\Large{QUESTION
32.3 
 (          4,          4,          4)
}}}
  
  
 
 
\noindent\vspace{0.05in}{\textbf{\Large{Auto-answer:}}}
  
  
\begin{tabular}{|l|l|l|}
 \hline
 Column Left & Column Right  & Answers       \\ 
 \hline
{\textbf{\large{
A.}}}
yjh
  & 
eR
 & 
{\textbf{\large{
C.}}}
, 
{\textbf{\large{
D.}}}
 \\ 
 \hline
{\textbf{\large{
B.}}}
C
  & 
b
 & 
{\textbf{\large{
E.}}}
 \\ 
 \hline
{\textbf{\large{
C.}}}
er
  & 
YJH
 & 
{\textbf{\large{
A.}}}
 \\ 
 \hline
{\textbf{\large{
D.}}}
Er
  & 
ER
 & 
{\textbf{\large{
C.}}}
, 
{\textbf{\large{
D.}}}
 \\ 
 \hline
{\textbf{\large{
E.}}}
B
  & 
c
 & 
{\textbf{\large{
B.}}}
 \\ 
 \hline
 \end{tabular}
  
  
 
 
 
 
  
\vspace{0.2in}
  
{\textbf{\Large{QUESTION
32.4 
 (          2,          2,          2)
}}}
  
  
 
 
\noindent\vspace{0.05in}{\textbf{\Large{Auto-answer:}}}
 
 
\noindent{\textbf{\large{
E.}}}
The accelaration is
$(
.34483ms^{-2},
2234.5km/h^2,
-155.17ms^{-2}
).
$
 
 
 
 
 
 
\noindent\vspace{0.1in}{\textbf{\Large{Solution: }}}

We will use the Newton's Second Law:
 
\[
\mathbf{f}=m\mathbf{a}.
\]
 
Since $\mathbf{f}=( % 
20.000,  % 
10.0000,  % 
-9000.0 )N$
and $m= % 
58.0000kg$, bring them into the above equation, then we get
 
\begin{eqnarray*}
\mathbf{a}&=&\frac{\mathbf{f}}m  \\
&=&\frac{(
20.000 ,
10.0000 ,
-9000.0 )N
}{ % 
58.0000 kg}  \\
&=&(
.34483 ,
.17241,
-155.17
)ms^{-2} \\
&=&(
4469.0 ,
2234.5 ,
-2.0110 \times 10^{6}
)km/h^2.
\end{eqnarray*}
 
 
 
  
\vspace{0.2in}
  
{\textbf{\Large{QUESTION
32.5 
 (          3,          3,          3)
}}}
  
  
 
 
\noindent\vspace{0.05in}{\textbf{\Large{Auto-answer:}}}
 
 
\noindent{\textbf{\large{
A.}}}
Canada has  %
10 provinces and  %
3 territories.
 
 
 
 
  
\vspace{0.2in}
  
{\textbf{\Large{QUESTION
32.6 
 (          1,          1,          1)
}}}
  
  


 
 
\noindent\vspace{0.05in}{\textbf{\Large{Auto-answer:}}}
 
 
\noindent{\textbf{\large{
F.}}}
The accelaration is $  %
(
.800,
.16,
-120.00)
ms^{-2} $.
 
 
 
 
 
 
\noindent\vspace{0.05in}{\textbf{\Large{Answer:}}}

The correct answer from the choices is


\noindent{\textbf{\large{
F.}}}
The accelaration is $  %
(
.800,
.16,
-120.00)
ms^{-2} $.
 
 
 
 
 
\noindent\vspace{0.1in}{\textbf{\Large{Solution: }}}

We will use the Newton's Second Law:
 
\[
\mathbf{f}=m\mathbf{a}.
\]
 
Since $\mathbf{f}= % 
(40.0 , 8.0 , -6000.0) N$
and $m= % 
50.0000kg$, bring them into the above equation, then we get
 
\begin{eqnarray*}
\mathbf{a}&=&\frac{\mathbf{f}}m  \\
&=&\frac{ % 
(40.0 , 8.0 , -6000.0) N}{ % 
50.0000kg}  \\
&=& % 
(.800 , .16 , -120.00) ms^{-2}
\end{eqnarray*}
 
 
 
   
   
\vspace{0.3in}
{\textbf{\LARGE{You have done all the above? Excellent! Not much left, please continue.}}}
\vspace{0.3in}
   
   
  
\vspace{0.2in}
  
{\textbf{\Large{QUESTION
32.7 
 (          8,         15,         60)
}}}
  
  
 
 
\noindent\vspace{0.05in}{\textbf{\Large{Answer:}}}

 
$\left( \begin{array}{ccccccccccccccc}
           7 & 
           4 & 
           4 & 
           7 \\ 
           6 & 
           4 & 
           5 & 
           7 \\ 
           5 & 
           6 & 
           6 & 
           5
\end{array}\right) \times
\left( \begin{array}{c}
           2 \\ 
           2 \\ 
           2 \\ 
           2
\end{array}\right)  =
\left( \begin{array}{c}
          44 \\ 
          44 \\ 
          44
\end{array}\right)  $
 
$  % 
 \left( \begin{array}
 {
 c
 c
 }
                    \Xi & 
 \eta \\ 
 \Upsilon & 
 \Lambda \\ 
 \delta & 
 \delta \\ 
 \rho & 
 \sigma
 \end{array} \right)
 \left( \begin{array}
 {
 c
 }
 \beta \\ 
 \beta
 \end{array} \right)
=
  \left( \begin{array}
 {
 c
 }
                    \Xi \times  \beta   +  \eta \times  \beta \\ 
 \Upsilon \times  \beta   +  \Lambda \times  \beta \\ 
 \delta \times  \beta   +  \delta \times  \beta \\ 
 \rho \times  \beta   +  \sigma \times  \beta
 \end{array} \right)
$
 
 
 
 
 
\noindent\vspace{0.1in}{\textbf{\Large{Solution: }}}

 
 
  
\vspace{0.2in}
  
{\textbf{\Large{QUESTION
32.8 
 (          7,         14,         50)
}}}
  
  
 
 
\noindent\vspace{0.05in}{\textbf{\Large{Auto-answer:}}}
 
 
\noindent{\textbf{\large{
D.}}}
  The accelaration is $  %
(
1.21,
.10,
-86.207)
ms^{-2} $.
 
 
 
 
 
 
\noindent\vspace{0.1in}{\textbf{\Large{Solution: }}}

We will use the Newton's Second Law:
 
\[
\mathbf{f}=m\mathbf{a}.
\]
 
Since $\mathbf{f}= % 
(70.0 , 6.0 , -5000.0) N$
and $m= % 
58.0kg$, bring them into the above equation, then we get
 
\begin{eqnarray*}
\mathbf{a}&=&\frac{\mathbf{f}}m  \\
&=&\frac{ % 
(70.0 , 6.0 , -5000.0) N}{ % 
58.0kg}  \\
&=& % 
(1.21 , .10 , -86.207) ms^{-2}
\end{eqnarray*}
 
 
 
  
\vspace{0.2in}
  
{\textbf{\Large{QUESTION
32.9 
 (          9,         16,         70)
}}}
  
  


 
 
\noindent\vspace{0.05in}{\textbf{\Large{Answer:}}}

-3,  % 
5
 
 
 
 
 
\noindent\vspace{0.1in}{\textbf{\Large{Solution: }}}

Roots to the equation
\begin{eqnarray*}
1 \times x^2  % 
-2
                 \times x    % 
-15 =0
\end{eqnarray*}
are  % 
-3 and  % 
5 .
 
Let us verity  % 
-3 first:
$  % 
1 \times x^2  % 
-2
                 \times x    % 
-15
  = % 
9+( % 
6)+( % 
-15)
  = % 
15+( % 
-15)
  = % 
0
$
 
Then verity  % 
5:
$  % 
1 \times x^2  % 
-2
                 \times x    % 
-15
  = % 
25+( % 
-10)+( % 
-15)
  = % 
15+( % 
-15)
  = % 
0
$
 
 
 
   
   
 \vspace{0.2in}
Here are still some constants for use:
 
 
\noindent\begin{tabular}{|l|l|l|}
\hline
Constant & Symbol & Value \\
\hline
 
Mass of proton &
$m_p$ &
 $ 1.6726231 \times 10^{-27} $
kg \\
\hline
 
Boltzmann's constant &
$k$ &
 $ 1.381 \times 10^{-23} $
J/K \\
\hline
 
\end{tabular}
 
Thank you very much for answering these questions!
 
{\textbf{\large{Please be advised}}} that in this paper there are questions from
32.1 through
32.9.
And any one of them may contain more than one sub-question, thus the total number
of sub-questions here is around 14, of which
13 should be answered.
 
   
   
   
   
\vspace{1.0in} 
{\textbf{\large{ *** END OF PAPER, THANKS *** }}} 
   
   
\hspace{1.0in} By: 
         239(         26,          34)
   
   
   
   
\newpage 
\setcounter{page}{ 
    33001 } 
   
   
\noindent{\textbf{\large{THIS IS THE ANSWER AND SOLUTION FOR}}}
   
   
 {\textbf{ \Large{ PAPER NUMBER          33 }}}
   
   
\vspace{0.2in}
   
   
\markboth{Answer and solution NOT for examinees !!!{\today}}{Answer and solution NOT for examinees !!! {\today}}
   
   
   
   
 \vspace{0.2in}
 
 
{\Huge  THIS IS AN EXAMPLE OF}
 
{\Huge  PERSONALIZED TESTS. }
 
If needed, please use the following constants.
 
 
 
\noindent\begin{tabular}{|l|l|l|}
\hline
Constant & Symbol & Value \\
\hline
Acceleration due to earth's gravity &
$g$ &
 $ 9.80 $
m/s$^2$ \\
\hline
Avogadro's number &
$N_A$ &
 $ 6.0221367 \times 10^{23} $
mol$^{-1}$ \\
\hline
Boltzmann's constant &
$k$ &
 $ 1.380658 \times 10^{-23} $
J/K \\
\hline
Coulomb's constant &
$k$ &
 $ 8.99 \times 10^{9} $
N$\cdot $m$^2$/C$^2$ \\
\hline
Electron charge magnitiude &
$e$ &
 $ 1.60217733 \times 10^{-19} $
C \\
\hline
Permeability of free space &
$\mu _0$ &
 $ 1.25663706 \times 10^{-6} $
T$\cdot $m/A \\
\hline
Permittivity of free space &
$\epsilon _0$ &
 $ 8.854187817 \times 10^{-12} $
C$^2$/(N$\cdot $m$^2$) \\
\hline
Pi &
$\pi$ &
 $ 3.14159265 $
$ $ \\
\hline
Planck's constant &
$h$ &
 $ 6.6260755 \times 10^{-34} $
J$\cdot $s \\
\hline
Mass of electron &
$m_e$ &
 $ 9.1093897 \times 10^{-31} $
kg \\
\hline
\end{tabular}
 
 
\noindent\begin{tabular}{|l|l|l|}
\hline
Constant & Symbol & Value \\
\hline
Mass of neutron &
$m_n$ &
 $ 1.6749286 \times 10^{-27} $
kg \\
\hline
Mass of proton &
$m_p$ &
 $ 1.6726231 \times 10^{-27} $
kg \\
\hline
Speed of light in vacuum &
$c$ &
 $ 299792458. $
m/s \\
\hline
Universal gravitational constant &
$G$ &
 $ 6.67259 \times 10^{-11} $
N$\cdot $m$^2$/kg$^2$ \\
\hline
Universal gas constant &
$R$ &
 $ 8.314510 $
J/(mol$\cdot $K) \\
\hline
\end{tabular}
 
 
{\textbf{\large{Please be advised}}} that in this paper there are questions from
33.1 through
33.9.
And any one of them may contain more than one sub-question, thus the total number
of sub-questions here is around 14, of which
13 should be answered.
 
\vspace{0.3in}
 
 
   
   
   
\vspace{0.2in}
   
In this paper, big questions will be generated in the following order: 
   
   
            1(          6)
 ,
            2(          3)
 ,
            3(          5)
 ,
            4(          1)
 ,
            5(          2)
 ,
            6(          4)
 ,
            7(          8)
 ,
            8(          7)
 ,
            9(          9)
 .
  
\vspace{0.2in}
  
{\textbf{\Large{QUESTION
33.1 
 (          6)
}}}
  
  
 
{\textbf{\Large{Please answer ONLY
5 of the following
6 questions (Questions
33.1.1 through
33.1.6). }}}
 
Here are still some constants for use in the following questions:
 
 
\noindent\begin{tabular}{|l|l|l|}
\hline
Constant & Symbol & Value \\
\hline
 
Boltzmann's constant &
$k$ &
 $ 1.381 \times 10^{-23} $
J/K \\
\hline
 
Avogadro's number &
$N_A$ &
 $ 6.022 \times 10^{23} $
mol$^{-1}$ \\
\hline
 
Mass of electron &
$m_e$ &
 $ 9.1093897 \times 10^{-31} $
kg \\
\hline
 
\end{tabular}
 
   
\vspace{0.2in}
   
 In this big question of CHOOSE structure,           6 questions will be generat
 ed: 
  
  
            1(         12,         27)
 ,
            2(         11,         26)
 ,
            3(         13,         28)
 ,
            4(          9,         24)
 ,
            5(          8,         23)
 ,
            6(         10,         25)
 .
  
\vspace{0.2in}
  
{\textbf{\Large{Question
33.1.1 
 (          6,         12,         27)
}}}
  
  
 
 
\noindent\vspace{0.1in}{\textbf{\Large{Solution: }}}

Since the possiblity of  % 
smoking customer is $ a =  % 
.440 $,
and the possiblity of  % 
 under 30 years old customer is $ b =  % 
2.00 \times 10^{-2} $,
the possiblity of  % 
non-smoking customer is $ c = 1.0 - a = 1.0 -
.440
=  % 
.560 $ and the possiblity of  % 
equal-or-above 30 years old
customer is $ d = 1.0 - b = 1.0 -  % 
2.00 \times 10^{-2} =  % 
.9800  $.
Then
 
\noindent
\begin{tabular}{|l|l|}
\hline
Customer & Possibility \\
\hline
smoking  and  % 
equal-or-above 30 years old  &
  $ % 
.440 \times  % 
.9800 =  % 
.431$ \\
\hline
smoking  and  % 
under 30 years old &
  $ % 
.440 \times  % 
2.000 \times 10^{-2} =  % 
8.80 \times 10^{-3}$ \\
\hline
 non-smoking and  % 
equal-or-above 30 years old  &
  $ % 
.560 \times  % 
.9800 =  % 
.549$ \\
\hline
 non-smoking and  % 
under 30 years old &
  $ % 
.560 \times  % 
2.000 \times 10^{-2} =  % 
1.12 \times 10^{-2}$ \\
\hline
\end{tabular}
 
\noindent
And the total summation of all possibilities is $  % 
1.0000 $.
 
 
 
 
 
 
\noindent\vspace{0.05in}{\textbf{\Large{Answer:}}}

 
\noindent
\begin{tabular}{|l|l|}
\hline
Customer & Possibility \\
\hline
smoking  and  % 
equal-or-above 30 years old &
  $ % 
.431$ \\
\hline
smoking  and  % 
under 30 years old &
  $ % 
8.80 \times 10^{-3}$ \\
\hline
 non-smoking and  % 
equal-or-above 30 years old &
  $ % 
.549$ \\
\hline
 non-smoking and  % 
under 30 years old &
  $ % 
1.12 \times 10^{-2}$ \\
\hline
\end{tabular}
 
\noindent
 And the total summation of all possibilities is $  % 
1.0000 $.
 
 
 
  
\vspace{0.2in}
  
{\textbf{\Large{Question
33.1.2 
 (          6,         11,         26)
}}}
  
  
 
 
\noindent\vspace{0.1in}{\textbf{\Large{Solution: }}}

Since the possiblity of  % 
smoking customer is $ a =  % 
.810 $,
and the possiblity of  % 
equal or above 30 years old customer is $ b =  % 
.5200 $,
the possiblity of  % 
non-smoking customer is $ c = 1.0 - a = 1.0 -
.810
=  % 
.190 $ and the possiblity of  % 
under 30 years old
customer is $ d = 1.0 - b = 1.0 -  % 
.5200 =  % 
.4800  $.
So the possibility of  % 
 non-smoking and  % 
under 30 years old
customer is $ c \times d =  % 
9.12 \times 10^{-2} $.
 
 
 
 
 
\noindent\vspace{0.05in}{\textbf{\Large{Answer:}}}

The possibility of  % 
 non-smoking and  % 
under 30 years old
customer is $ (1-a)(1-b) =  % 
9.12 \times 10^{-2} $.
 
 
  
\vspace{0.2in}
  
{\textbf{\Large{Question
33.1.3 
 (          6,         13,         28)
}}}
  
  
 
 
\noindent\vspace{0.05in}{\textbf{\Large{Answer:}}}

5;
 
2;
 
The operation is  % 
MULTIPLICATION and the result is
$ % 
10.000$.
 
 
 
  
\vspace{0.2in}
  
{\textbf{\Large{Question
33.1.4 
 (          6,          9,         24)
}}}
  
  
 
 
\noindent\vspace{0.1in}{\textbf{\Large{Solution: }}}

By using Newton's Law of Universal Gravitation:
\[
F=G \frac{(Sun's \hspace{0.1in} mass) \times (Planet's \hspace{0.1in} mass)} { (distance)^2},
\]
where
$ G= % 
6.67 \times 10^{-11}N m^{2}(kg)^{-2}$ , the forces can be easily calculated as
 
\vspace{0.2in}
 
 
\begin{tabular}{|l|l|l|l|}
\hline
The Planet & Mass ($kg$) & Distanace from Sun ($m$) & The Force ($N$)\\
\hline
Mercury  &
           $ % 
3.00000000 \times 10^{24} $   &
             $ % 
2.000000000 \times 10^{24} $    & $ % 
1.00 \times 10^{-10} $
\\  \hline
Venus    &
           $  % 
7.00 \times 10^{24}  $     &
             $ % 
5.00 \times 10^{24} $    & $ % 
3.74 \times 10^{-11} $
\\  \hline
Earth    &
           $  % 
7.00 \times 10^{24}  $     &
             $ % 
9.00 \times 10^{24} $    & $ % 
1.15 \times 10^{-11} $
\\   \hline
Mars     &
           $  % 
6.00 \times 10^{24} $     &
             $ % 
5.00 \times 10^{24} $    & $ % 
3.20 \times 10^{-11} $
\\   \hline
Jupiter  &
           $  % 
6.00 \times 10^{24} $    &
             $ % 
4.00 \times 10^{24} $    & $ % 
5.00 \times 10^{-11} $
\\  \hline
Saturn   &
           $  % 
7.00 \times 10^{24} $    &
             $ % 
7.00 \times 10^{24}  $    & $ % 
1.91 \times 10^{-11} $
\\  \hline
Uranus   &
           $  % 
8.00 \times 10^{24} $    &
             $ % 
5.00 \times 10^{24} $    & $ % 
4.27 \times 10^{-11} $
\\  \hline
Neptune  &
           $  % 
5.00 \times 10^{24} $    &
             $ % 
5.00 \times 10^{24} $    & $ % 
2.67 \times 10^{-11} $
\\  \hline
 
\end{tabular}
 
 
 
 
 
 
\noindent\vspace{0.05in}{\textbf{\Large{Answer:}}}

By using Newton's Law of Universal Gravitation:
\[
F=G \frac{(Sun's \hspace{0.1in} mass) \times (Planet's \hspace{0.1in} mass)} { (distance)^2},
\]
where
$ G= % 
6.67 \times 10^{-11} N m^{2}(kg)^{-2}$ , the forces can be easily calculated as
 
\vspace{0.2in}
 
 
\begin{tabular}{|l|l|l|l|}
\hline
The Planet & Mass ($kg$) & Distanace from Sun ($m$) & The Force ($N$)\\
\hline
Mercury  &
           $ % 
3.00000000 \times 10^{24}  $   &
             $ % 
2.000000000 \times 10^{24}$    & $ % 
1.00 \times 10^{-10} $
\\  \hline
Venus    &
           $  % 
7.00 \times 10^{24}  $     &
             $ % 
5.00 \times 10^{24} $    & $ % 
3.74 \times 10^{-11} $
\\  \hline
Earth    &
           $  % 
7.00 \times 10^{24}$     &
             $ % 
9.00 \times 10^{24} $    & $ % 
1.15 \times 10^{-11} $
\\   \hline
Mars     &
           $  % 
6.00 \times 10^{24} $     &
             $ % 
5.00 \times 10^{24}$    & $ % 
3.20 \times 10^{-11} $
\\   \hline
Jupiter  &
           $  % 
6.00 \times 10^{24}  $    &
             $ % 
4.00 \times 10^{24} $    & $ % 
5.00 \times 10^{-11}3 $
\\  \hline
Saturn   &
           $  % 
7.00 \times 10^{24}   $    &
             $ % 
7.00 \times 10^{24}  $    & $ % 
1.91 \times 10^{-11} $
\\  \hline
Uranus   &
           $  % 
8.00 \times 10^{24} $    &
             $ % 
5.00 \times 10^{24}$    & $ % 
4.27 \times 10^{-11} $
\\  \hline
Neptune  &
           $  % 
5.00 \times 10^{24}  $    &
             $ % 
5.00 \times 10^{24} $    & $ % 
2.67 \times 10^{-11} $
\\  \hline
 
\end{tabular}
 
 
 
 
  
\vspace{0.2in}
  
{\textbf{\Large{Question
33.1.5 
 (          6,          8,         23)
}}}
  
  
 
 
\noindent\vspace{0.05in}{\textbf{\Large{Auto-answer:}}}
 
 
\noindent{\textbf{\large{
B.}}}
The accelaration is
$(
1.4000ms^{-2},
.18000ms^{-2},
-2.0736 \times 10^{6}km/h^2
).
$
 
 
 
 
 
 
\noindent\vspace{0.1in}{\textbf{\Large{Solution: }}}

We will use the Newton's Second Law:
 
\[
\mathbf{f}=m\mathbf{a}.
\]
 
Since $\mathbf{f}=( % 
70.0,  % 
9.0,  % 
-8000.0 )N$
and $m= % 
50.0kg$, bring them into the above equation, then we get
 
\begin{eqnarray*}
\mathbf{a}&=&\frac{\mathbf{f}}m  \\
&=&\frac{(
70.0 ,
9.0 ,
-8000.0 )N
}{ % 
50.0 kg}  \\
&=&(
1.4000 ,
.18000,
-160.00
)ms^{-2} \\
&=&(
18144. ,
2332.8 ,
-2.0736 \times 10^{6}
)km/h^2.
\end{eqnarray*}
 
 
 
  
\vspace{0.2in}
  
{\textbf{\Large{Question
33.1.6 
 (          6,         10,         25)
}}}
  
  
 
 
\noindent\vspace{0.05in}{\textbf{\Large{Auto-answer:}}}
 
 
\noindent{\textbf{\large{
A.}}}
An airplane
 
 
 
 
   
   
\vspace{0.3in}
{\textbf{\LARGE{You have done all the above? A very good beginning, please go ahead.}}}
More constants the
Mass of electron
$m_e$$ =
9.109390 \times 10^{-31} $
kg
,
Universal gas constant
$R$$ =
8.315 $
J/(mol$\cdot $K)
,
$e$$ =
1.60217733 \times 10^{-19} $
C
, and
$m_p$$ =
1.6726231 \times 10^{-27} $
kg
%
may be very helpful.
\vspace{0.3in}
   
   
  
\vspace{0.2in}
  
{\textbf{\Large{QUESTION
33.2 
 (          3,          3,          3)
}}}
  
  
 
 
\noindent\vspace{0.05in}{\textbf{\Large{Auto-answer:}}}
 
 
\noindent{\textbf{\large{
A.}}}
Canada has  %
10 provinces and  %
3 territories.
 
 
 
 
  
\vspace{0.2in}
  
{\textbf{\Large{QUESTION
33.3 
 (          5,          5,          5)
}}}
  
  
 
 
\noindent\vspace{0.05in}{\textbf{\Large{Answer:}}}

 
\noindent\begin{tabular}{|l|l|}\hline The correct & \\
          answer &  % 
$T$ \\ \hline \end{tabular}
1. $ % 
60$ is an  % 
even number.
 
\noindent\begin{tabular}{|l|l|}\hline The correct & \\
          answer &  % 
$T$ \\ \hline \end{tabular}
2.  % 
Kingston is in  % 
Ontario province.
 
\noindent\begin{tabular}{|l|l|}\hline The correct & \\
          answer &  % 
$T$ \\ \hline \end{tabular}
3.  % 
$\mathbf{F}=m\mathbf{a}$ is a mathmatical form of  % 
the Newton's Second Law.
 
 
 
  
\vspace{0.2in}
  
{\textbf{\Large{QUESTION
33.4 
 (          1,          1,          1)
}}}
  
  


 
 
\noindent\vspace{0.05in}{\textbf{\Large{Auto-answer:}}}
 
 
\noindent{\textbf{\large{
G.}}}
The accelaration is $  %
(
.385,
.17,
-76.923)
ms^{-2} $.
 
 
 
 
 
 
\noindent\vspace{0.05in}{\textbf{\Large{Answer:}}}

The correct answer from the choices is


\noindent{\textbf{\large{
G.}}}
The accelaration is $  %
(
.385,
.17,
-76.923)
ms^{-2} $.
 
 
 
 
 
\noindent\vspace{0.1in}{\textbf{\Large{Solution: }}}

We will use the Newton's Second Law:
 
\[
\mathbf{f}=m\mathbf{a}.
\]
 
Since $\mathbf{f}= % 
(20.0 , 9.0 , -4000.0) N$
and $m= % 
52.0000kg$, bring them into the above equation, then we get
 
\begin{eqnarray*}
\mathbf{a}&=&\frac{\mathbf{f}}m  \\
&=&\frac{ % 
(20.0 , 9.0 , -4000.0) N}{ % 
52.0000kg}  \\
&=& % 
(.385 , .17 , -76.923) ms^{-2}
\end{eqnarray*}
 
 
 
  
\vspace{0.2in}
  
{\textbf{\Large{QUESTION
33.5 
 (          2,          2,          2)
}}}
  
  
 
 
\noindent\vspace{0.05in}{\textbf{\Large{Auto-answer:}}}
 
 
\noindent{\textbf{\large{
G.}}}
 None of these.
 
 
 
 
 
 
\noindent\vspace{0.1in}{\textbf{\Large{Solution: }}}

We will use the Newton's Second Law:
 
\[
\mathbf{f}=m\mathbf{a}.
\]
 
Since $\mathbf{f}=( % 
100.000,  % 
2.0000,  % 
-9000.0 )N$
and $m= % 
50.0000kg$, bring them into the above equation, then we get
 
\begin{eqnarray*}
\mathbf{a}&=&\frac{\mathbf{f}}m  \\
&=&\frac{(
100.000 ,
2.0000 ,
-9000.0 )N
}{ % 
50.0000 kg}  \\
&=&(
2.0000 ,
4.0000 \times 10^{-2},
-180.00
)ms^{-2} \\
&=&(
25920. ,
518.40 ,
-2.3328 \times 10^{6}
)km/h^2.
\end{eqnarray*}
 
 
 
  
\vspace{0.2in}
  
{\textbf{\Large{QUESTION
33.6 
 (          4,          4,          4)
}}}
  
  
 
 
\noindent\vspace{0.05in}{\textbf{\Large{Auto-answer:}}}
  
  
\begin{tabular}{|l|l|l|}
 \hline
 Column Left & Column Right  & Answers       \\ 
 \hline
{\textbf{\large{
A.}}}
B
  & 
ER
 & 
{\textbf{\large{
C.}}}
 \\ 
 \hline
{\textbf{\large{
B.}}}
asdf(:)
  & 
 a= %
2
 & 
{\textbf{\large{
E.}}}
 \\ 
 \hline
{\textbf{\large{
C.}}}
er
  & 
YJH
 & 
{\textbf{\large{
D.}}}
 \\ 
 \hline
{\textbf{\large{
D.}}}
yjh
  & 
b
 & 
{\textbf{\large{
A.}}}
 \\ 
 \hline
{\textbf{\large{
E.}}}
 A= %
4/ %
2

  & 
ASDF(:)
 & 
{\textbf{\large{
B.}}}
 \\ 
 \hline
 \end{tabular}
  
  
 
 
 
 
   
   
\vspace{0.3in}
{\textbf{\LARGE{You have done all the above? Excellent! Not much left, please continue.}}}
\vspace{0.3in}
   
   
  
\vspace{0.2in}
  
{\textbf{\Large{QUESTION
33.7 
 (          8,         15,         60)
}}}
  
  
 
 
\noindent\vspace{0.05in}{\textbf{\Large{Answer:}}}

 
$\left( \begin{array}{ccccccccccccccc}
           6 & 
           6 & 
           6 & 
           4 \\ 
           5 & 
           4 & 
           5 & 
           6 \\ 
           4 & 
           4 & 
           5 & 
           4
\end{array}\right) \times
\left( \begin{array}{c}
           2 \\ 
           2 \\ 
           2 \\ 
           2
\end{array}\right)  =
\left( \begin{array}{c}
          44 \\ 
          40 \\ 
          34
\end{array}\right)  $
 
$  % 
 \left( \begin{array}
 {
 c
 c
 }
 \Theta & 
 \eta \\ 
 \rho & 
 \Gamma \\ 
                    \zeta & 
 \Delta \\ 
 \alpha & 
 \Theta
 \end{array} \right)
 \left( \begin{array}
 {
 c
 }
 \beta \\ 
 \beta
 \end{array} \right)
=
  \left( \begin{array}
 {
 c
 }
 \Theta \times  \beta   +  \eta \times  \beta \\ 
 \rho \times  \beta   +  \Gamma \times  \beta \\ 
                    \zeta \times  \beta   +  \Delta \times  \beta \\ 
 \alpha \times  \beta   +  \Theta \times  \beta
 \end{array} \right)
$
 
 
 
 
 
\noindent\vspace{0.1in}{\textbf{\Large{Solution: }}}

 
 
  
\vspace{0.2in}
  
{\textbf{\Large{QUESTION
33.8 
 (          7,         14,         50)
}}}
  
  
 
 
\noindent\vspace{0.05in}{\textbf{\Large{Auto-answer:}}}
 
 
\noindent{\textbf{\large{
B.}}}
  The accelaration is $  %
(
.370,
7.4 \times 10^{-2},
-55.556)
ms^{-2} $.
 
 
 
 
 
 
\noindent\vspace{0.1in}{\textbf{\Large{Solution: }}}

We will use the Newton's Second Law:
 
\[
\mathbf{f}=m\mathbf{a}.
\]
 
Since $\mathbf{f}= % 
(20.0 , 4.0 , -3000.0) N$
and $m= % 
54.0kg$, bring them into the above equation, then we get
 
\begin{eqnarray*}
\mathbf{a}&=&\frac{\mathbf{f}}m  \\
&=&\frac{ % 
(20.0 , 4.0 , -3000.0) N}{ % 
54.0kg}  \\
&=& % 
(.370 , 7.4 \times 10^{-2} , -55.556) ms^{-2}
\end{eqnarray*}
 
 
 
  
\vspace{0.2in}
  
{\textbf{\Large{QUESTION
33.9 
 (          9,         16,         70)
}}}
  
  


 
 
\noindent\vspace{0.05in}{\textbf{\Large{Answer:}}}

9,  % 
-19
 
 
 
 
 
\noindent\vspace{0.1in}{\textbf{\Large{Solution: }}}

Roots to the equation
\begin{eqnarray*}
3 \times x^2  % 
+  % 
30
                 \times x    % 
-513 =0
\end{eqnarray*}
are  % 
9 and  % 
-19 .
 
Let us verity  % 
9 first:
$  % 
3 \times x^2  % 
+  % 
30
                 \times x    % 
-513
  = % 
243+( % 
270)+( % 
-513)
  = % 
513+( % 
-513)
  = % 
0
$
 
Then verity  % 
-19:
$  % 
3 \times x^2  % 
+  % 
30
                 \times x    % 
-513
  = % 
1083+( % 
-570)+( % 
-513)
  = % 
513+( % 
-513)
  = % 
0
$
 
 
 
   
   
 \vspace{0.2in}
Here are still some constants for use:
 
 
\noindent\begin{tabular}{|l|l|l|}
\hline
Constant & Symbol & Value \\
\hline
 
Mass of proton &
$m_p$ &
 $ 1.6726231 \times 10^{-27} $
kg \\
\hline
 
Boltzmann's constant &
$k$ &
 $ 1.381 \times 10^{-23} $
J/K \\
\hline
 
\end{tabular}
 
Thank you very much for answering these questions!
 
{\textbf{\large{Please be advised}}} that in this paper there are questions from
33.1 through
33.9.
And any one of them may contain more than one sub-question, thus the total number
of sub-questions here is around 14, of which
13 should be answered.
 
   
   
   
   
\vspace{1.0in} 
{\textbf{\large{ *** END OF PAPER, THANKS *** }}} 
   
   
\hspace{1.0in} By: 
         239(         26,          34)
   
   
   
   
\newpage 
\setcounter{page}{ 
    34001 } 
   
   
\noindent{\textbf{\large{THIS IS THE ANSWER AND SOLUTION FOR}}}
   
   
 {\textbf{ \Large{ PAPER NUMBER          34 }}}
   
   
\vspace{0.2in}
   
   
\markboth{Answer and solution NOT for examinees !!!{\today}}{Answer and solution NOT for examinees !!! {\today}}
   
   
   
   
 \vspace{0.2in}
 
 
{\Huge  THIS IS AN EXAMPLE OF}
 
{\Huge  PERSONALIZED TESTS. }
 
If needed, please use the following constants.
 
 
 
\noindent\begin{tabular}{|l|l|l|}
\hline
Constant & Symbol & Value \\
\hline
Acceleration due to earth's gravity &
$g$ &
 $ 9.80 $
m/s$^2$ \\
\hline
Avogadro's number &
$N_A$ &
 $ 6.0221367 \times 10^{23} $
mol$^{-1}$ \\
\hline
Boltzmann's constant &
$k$ &
 $ 1.380658 \times 10^{-23} $
J/K \\
\hline
Coulomb's constant &
$k$ &
 $ 8.99 \times 10^{9} $
N$\cdot $m$^2$/C$^2$ \\
\hline
Electron charge magnitiude &
$e$ &
 $ 1.60217733 \times 10^{-19} $
C \\
\hline
Permeability of free space &
$\mu _0$ &
 $ 1.25663706 \times 10^{-6} $
T$\cdot $m/A \\
\hline
Permittivity of free space &
$\epsilon _0$ &
 $ 8.854187817 \times 10^{-12} $
C$^2$/(N$\cdot $m$^2$) \\
\hline
Pi &
$\pi$ &
 $ 3.14159265 $
$ $ \\
\hline
Planck's constant &
$h$ &
 $ 6.6260755 \times 10^{-34} $
J$\cdot $s \\
\hline
Mass of electron &
$m_e$ &
 $ 9.1093897 \times 10^{-31} $
kg \\
\hline
\end{tabular}
 
 
\noindent\begin{tabular}{|l|l|l|}
\hline
Constant & Symbol & Value \\
\hline
Mass of neutron &
$m_n$ &
 $ 1.6749286 \times 10^{-27} $
kg \\
\hline
Mass of proton &
$m_p$ &
 $ 1.6726231 \times 10^{-27} $
kg \\
\hline
Speed of light in vacuum &
$c$ &
 $ 299792458. $
m/s \\
\hline
Universal gravitational constant &
$G$ &
 $ 6.67259 \times 10^{-11} $
N$\cdot $m$^2$/kg$^2$ \\
\hline
Universal gas constant &
$R$ &
 $ 8.314510 $
J/(mol$\cdot $K) \\
\hline
\end{tabular}
 
 
{\textbf{\large{Please be advised}}} that in this paper there are questions from
34.1 through
34.9.
And any one of them may contain more than one sub-question, thus the total number
of sub-questions here is around 14, of which
13 should be answered.
 
\vspace{0.3in}
 
 
   
   
   
\vspace{0.2in}
   
In this paper, big questions will be generated in the following order: 
   
   
            1(          6)
 ,
            2(          2)
 ,
            3(          1)
 ,
            4(          3)
 ,
            5(          5)
 ,
            6(          4)
 ,
            7(          8)
 ,
            8(          7)
 ,
            9(          9)
 .
  
\vspace{0.2in}
  
{\textbf{\Large{QUESTION
34.1 
 (          6)
}}}
  
  
 
{\textbf{\Large{Please answer ONLY
5 of the following
6 questions (Questions
34.1.1 through
34.1.6). }}}
 
Here are still some constants for use in the following questions:
 
 
\noindent\begin{tabular}{|l|l|l|}
\hline
Constant & Symbol & Value \\
\hline
 
Boltzmann's constant &
$k$ &
 $ 1.381 \times 10^{-23} $
J/K \\
\hline
 
Avogadro's number &
$N_A$ &
 $ 6.022 \times 10^{23} $
mol$^{-1}$ \\
\hline
 
Mass of electron &
$m_e$ &
 $ 9.1093897 \times 10^{-31} $
kg \\
\hline
 
\end{tabular}
 
   
\vspace{0.2in}
   
 In this big question of CHOOSE structure,           6 questions will be generat
 ed: 
  
  
            1(          8,         23)
 ,
            2(          9,         24)
 ,
            3(          7,         22)
 ,
            4(         11,         26)
 ,
            5(          6,         21)
 ,
            6(         10,         25)
 .
  
\vspace{0.2in}
  
{\textbf{\Large{Question
34.1.1 
 (          6,          8,         23)
}}}
  
  
 
 
\noindent\vspace{0.05in}{\textbf{\Large{Auto-answer:}}}
 
 
\noindent{\textbf{\large{
C.}}}
The accelaration is
$(
1.3462ms^{-2},
3.8462 \times 10^{-2}ms^{-2},
-498462.km/h^2
).
$
 
 
 
 
 
 
\noindent\vspace{0.1in}{\textbf{\Large{Solution: }}}

We will use the Newton's Second Law:
 
\[
\mathbf{f}=m\mathbf{a}.
\]
 
Since $\mathbf{f}=( % 
70.0,  % 
2.0,  % 
-2000.0 )N$
and $m= % 
52.0kg$, bring them into the above equation, then we get
 
\begin{eqnarray*}
\mathbf{a}&=&\frac{\mathbf{f}}m  \\
&=&\frac{(
70.0 ,
2.0 ,
-2000.0 )N
}{ % 
52.0 kg}  \\
&=&(
1.3462 ,
3.8462 \times 10^{-2},
-38.462
)ms^{-2} \\
&=&(
17446. ,
498.46 ,
-498462.
)km/h^2.
\end{eqnarray*}
 
 
 
  
\vspace{0.2in}
  
{\textbf{\Large{Question
34.1.2 
 (          6,          9,         24)
}}}
  
  
 
 
\noindent\vspace{0.1in}{\textbf{\Large{Solution: }}}

By using Newton's Law of Universal Gravitation:
\[
F=G \frac{(Sun's \hspace{0.1in} mass) \times (Planet's \hspace{0.1in} mass)} { (distance)^2},
\]
where
$ G= % 
6.67 \times 10^{-11}N m^{2}(kg)^{-2}$ , the forces can be easily calculated as
 
\vspace{0.2in}
 
 
\begin{tabular}{|l|l|l|l|}
\hline
The Planet & Mass ($kg$) & Distanace from Sun ($m$) & The Force ($N$)\\
\hline
Mercury  &
           $ % 
7.00000000 \times 10^{24} $   &
             $ % 
8.000000000 \times 10^{24} $    & $ % 
4.38 \times 10^{-11} $
\\  \hline
Venus    &
           $  % 
4.00 \times 10^{24}  $     &
             $ % 
6.00 \times 10^{24} $    & $ % 
4.45 \times 10^{-11} $
\\  \hline
Earth    &
           $  % 
5.00 \times 10^{24}  $     &
             $ % 
7.00 \times 10^{24} $    & $ % 
4.08 \times 10^{-11} $
\\   \hline
Mars     &
           $  % 
6.00 \times 10^{24} $     &
             $ % 
7.00 \times 10^{24} $    & $ % 
4.90 \times 10^{-11} $
\\   \hline
Jupiter  &
           $  % 
4.00 \times 10^{24} $    &
             $ % 
4.00 \times 10^{24} $    & $ % 
1.00 \times 10^{-10} $
\\  \hline
Saturn   &
           $  % 
4.00 \times 10^{24} $    &
             $ % 
7.00 \times 10^{24}  $    & $ % 
3.27 \times 10^{-11} $
\\  \hline
Uranus   &
           $  % 
3.00 \times 10^{24} $    &
             $ % 
3.00 \times 10^{24} $    & $ % 
1.33 \times 10^{-10} $
\\  \hline
Neptune  &
           $  % 
7.00 \times 10^{24} $    &
             $ % 
3.00 \times 10^{24} $    & $ % 
3.11 \times 10^{-10} $
\\  \hline
 
\end{tabular}
 
 
 
 
 
 
\noindent\vspace{0.05in}{\textbf{\Large{Answer:}}}

By using Newton's Law of Universal Gravitation:
\[
F=G \frac{(Sun's \hspace{0.1in} mass) \times (Planet's \hspace{0.1in} mass)} { (distance)^2},
\]
where
$ G= % 
6.67 \times 10^{-11} N m^{2}(kg)^{-2}$ , the forces can be easily calculated as
 
\vspace{0.2in}
 
 
\begin{tabular}{|l|l|l|l|}
\hline
The Planet & Mass ($kg$) & Distanace from Sun ($m$) & The Force ($N$)\\
\hline
Mercury  &
           $ % 
7.00000000 \times 10^{24}  $   &
             $ % 
8.000000000 \times 10^{24}$    & $ % 
4.38 \times 10^{-11} $
\\  \hline
Venus    &
           $  % 
4.00 \times 10^{24}  $     &
             $ % 
6.00 \times 10^{24} $    & $ % 
4.45 \times 10^{-11} $
\\  \hline
Earth    &
           $  % 
5.00 \times 10^{24}$     &
             $ % 
7.00 \times 10^{24} $    & $ % 
4.08 \times 10^{-11} $
\\   \hline
Mars     &
           $  % 
6.00 \times 10^{24} $     &
             $ % 
7.00 \times 10^{24}$    & $ % 
4.90 \times 10^{-11} $
\\   \hline
Jupiter  &
           $  % 
4.00 \times 10^{24}  $    &
             $ % 
4.00 \times 10^{24} $    & $ % 
1.00 \times 10^{-10}3 $
\\  \hline
Saturn   &
           $  % 
4.00 \times 10^{24}   $    &
             $ % 
7.00 \times 10^{24}  $    & $ % 
3.27 \times 10^{-11} $
\\  \hline
Uranus   &
           $  % 
3.00 \times 10^{24} $    &
             $ % 
3.00 \times 10^{24}$    & $ % 
1.33 \times 10^{-10} $
\\  \hline
Neptune  &
           $  % 
7.00 \times 10^{24}  $    &
             $ % 
3.00 \times 10^{24} $    & $ % 
3.11 \times 10^{-10} $
\\  \hline
 
\end{tabular}
 
 
 
 
  
\vspace{0.2in}
  
{\textbf{\Large{Question
34.1.3 
 (          6,          7,         22)
}}}
  
  
 
 
\noindent\vspace{0.05in}{\textbf{\Large{Auto-answer:}}}
 
 
\noindent{\textbf{\large{
A.}}}
The accelaration (vector) is
$(
7200.0,
1920.0 ,
-1.9200 \times 10^{6}
)km/h^2.
$
 
 
 
 
 
 
\noindent\vspace{0.1in}{\textbf{\Large{Solution: }}}

We will use the Newton's Second Law:
 
\[
\mathbf{f}=m\mathbf{a}.
\]
 
Since $\mathbf{f}=( % 
30.0,  % 
8.0,  % 
-8000.0 )N$
and $m= % 
54.0 kg$, bring them into the above equation, then we get
 
\begin{eqnarray*}
\mathbf{a}&=&\frac{\mathbf{f}}m  \\
&=&\frac{(
30.0 ,
8.0 ,
-8000.0 )N
}{ % 
54.0 kg}  \\
&=&(
.55556 ,
.14815,
-148.15
)ms^{-2} \\
&=&(
7200.0 ,
1920.0 ,
-1.9200 \times 10^{6}
)km/h^2.
\end{eqnarray*}
 
 
 
  
\vspace{0.2in}
  
{\textbf{\Large{Question
34.1.4 
 (          6,         11,         26)
}}}
  
  
 
 
\noindent\vspace{0.1in}{\textbf{\Large{Solution: }}}

Since the possiblity of  % 
smoking customer is $ a =  % 
.130 $,
and the possiblity of  % 
 under 30 years old customer is $ b =  % 
.9200 $,
the possiblity of  % 
non-smoking customer is $ c = 1.0 - a = 1.0 -
.130
=  % 
.870 $ and the possiblity of  % 
equal or above 30 years old
customer is $ d = 1.0 - b = 1.0 -  % 
.9200 =  % 
8.000 \times 10^{-2}  $.
So the possibility of  % 
 non-smoking and  % 
equal or above 30 years old
customer is $ c \times d =  % 
6.96 \times 10^{-2} $.
 
 
 
 
 
\noindent\vspace{0.05in}{\textbf{\Large{Answer:}}}

The possibility of  % 
 non-smoking and  % 
equal or above 30 years old
customer is $ (1-a)(1-b) =  % 
6.96 \times 10^{-2} $.
 
 
  
\vspace{0.2in}
  
{\textbf{\Large{Question
34.1.5 
 (          6,          6,         21)
}}}
  
  
 
 
\noindent\vspace{0.05in}{\textbf{\Large{Answer:}}}

We will use the Newton's Second Law:
 
\[
\mathbf{f}=m\mathbf{a}.
\]
 
Since $\mathbf{f}=( % 
20.0,  % 
3.0,  % 
-6000.0 )N$
and $m= % 
54.0 kg$, bring them into the above equation, then we get
 
\begin{eqnarray*}
\mathbf{a}&=&\frac{\mathbf{f}}m  \\
&=&\frac{(
20.0 ,
3.0 ,
-6000.0 )N
}{ % 
54.0 kg}  \\
&=&(
.37037 ,
5.5556 \times 10^{-2},
-111.11
)ms^{-2} \\
&=&(
4800.0 ,
720.00 ,
-1.4400 \times 10^{6}
)km/h^2.
\end{eqnarray*}
 
 
 
 
 
\noindent\vspace{0.1in}{\textbf{\Large{Solution: }}}

We will use the Newton's Second Law:
 
\[
\mathbf{f}=m\mathbf{a}.
\]
 
Since $\mathbf{f}=( % 
20.0,  % 
3.0,  % 
-6000.0 )N$
and $m= % 
54.0 kg$, bring them into the above equation, then we get
 
\begin{eqnarray*}
\mathbf{a}&=&\frac{\mathbf{f}}m  \\
&=&\frac{(
20.0 ,
3.0 ,
-6000.0 )N
}{ % 
54.0 kg}  \\
&=&(
.37037 ,
5.5556 \times 10^{-2},
-111.11
)ms^{-2} \\
&=&(
4800.0 ,
720.00 ,
-1.4400 \times 10^{6}
)km/h^2.
\end{eqnarray*}
 
 
 
  
\vspace{0.2in}
  
{\textbf{\Large{Question
34.1.6 
 (          6,         10,         25)
}}}
  
  
 
 
\noindent\vspace{0.05in}{\textbf{\Large{Auto-answer:}}}
 
 
\noindent{\textbf{\large{
C.}}}
A truck
 
 
\noindent{\textbf{\large{
D.}}}
An airplane
 
 
 
 
   
   
\vspace{0.3in}
{\textbf{\LARGE{You have done all the above? A very good beginning, please go ahead.}}}
More constants the
Mass of electron
$m_e$$ =
9.109390 \times 10^{-31} $
kg
,
Universal gas constant
$R$$ =
8.315 $
J/(mol$\cdot $K)
,
$e$$ =
1.60217733 \times 10^{-19} $
C
, and
$m_p$$ =
1.6726231 \times 10^{-27} $
kg
%
may be very helpful.
\vspace{0.3in}
   
   
  
\vspace{0.2in}
  
{\textbf{\Large{QUESTION
34.2 
 (          2,          2,          2)
}}}
  
  
 
 
\noindent\vspace{0.05in}{\textbf{\Large{Auto-answer:}}}
 
 
\noindent{\textbf{\large{
C.}}}
The accelaration is
$(
1.0000ms^{-2},
1555.2km/h^2,
-100.00ms^{-2}
).
$
 
 
 
 
 
 
\noindent\vspace{0.1in}{\textbf{\Large{Solution: }}}

We will use the Newton's Second Law:
 
\[
\mathbf{f}=m\mathbf{a}.
\]
 
Since $\mathbf{f}=( % 
50.000,  % 
6.0000,  % 
-5000.0 )N$
and $m= % 
50.0000kg$, bring them into the above equation, then we get
 
\begin{eqnarray*}
\mathbf{a}&=&\frac{\mathbf{f}}m  \\
&=&\frac{(
50.000 ,
6.0000 ,
-5000.0 )N
}{ % 
50.0000 kg}  \\
&=&(
1.0000 ,
.12000,
-100.00
)ms^{-2} \\
&=&(
12960. ,
1555.2 ,
-1.2960 \times 10^{6}
)km/h^2.
\end{eqnarray*}
 
 
 
  
\vspace{0.2in}
  
{\textbf{\Large{QUESTION
34.3 
 (          1,          1,          1)
}}}
  
  


 
 
\noindent\vspace{0.05in}{\textbf{\Large{Auto-answer:}}}
 
 
\noindent{\textbf{\large{
G.}}}
The accelaration is $  %
(
.714,
.18,
-142.86)
ms^{-2} $.
 
 
 
 
 
 
\noindent\vspace{0.05in}{\textbf{\Large{Answer:}}}

The correct answer from the choices is


\noindent{\textbf{\large{
G.}}}
The accelaration is $  %
(
.714,
.18,
-142.86)
ms^{-2} $.
 
 
 
 
 
\noindent\vspace{0.1in}{\textbf{\Large{Solution: }}}

We will use the Newton's Second Law:
 
\[
\mathbf{f}=m\mathbf{a}.
\]
 
Since $\mathbf{f}= % 
(40.0 , 10.0 , -8000.0) N$
and $m= % 
56.0000kg$, bring them into the above equation, then we get
 
\begin{eqnarray*}
\mathbf{a}&=&\frac{\mathbf{f}}m  \\
&=&\frac{ % 
(40.0 , 10.0 , -8000.0) N}{ % 
56.0000kg}  \\
&=& % 
(.714 , .18 , -142.86) ms^{-2}
\end{eqnarray*}
 
 
 
  
\vspace{0.2in}
  
{\textbf{\Large{QUESTION
34.4 
 (          3,          3,          3)
}}}
  
  
 
 
\noindent\vspace{0.05in}{\textbf{\Large{Auto-answer:}}}
 
 
\noindent{\textbf{\large{
E.}}}
Canada has  %
10 provinces and  %
3 territories.
 
 
 
 
  
\vspace{0.2in}
  
{\textbf{\Large{QUESTION
34.5 
 (          5,          5,          5)
}}}
  
  
 
 
\noindent\vspace{0.05in}{\textbf{\Large{Answer:}}}

 
\noindent\begin{tabular}{|l|l|}\hline The correct & \\
          answer &  % 
$T$ \\ \hline \end{tabular}
1. $ % 
97$ is an  % 
odd number.
 
\noindent\begin{tabular}{|l|l|}\hline The correct & \\
          answer &  % 
$T$ \\ \hline \end{tabular}
2.  % 
Kingston is in  % 
Ontario province.
 
\noindent\begin{tabular}{|l|l|}\hline The correct & \\
          answer &  % 
$T$ \\ \hline \end{tabular}
3.  % 
$\mathbf{F}=m\mathbf{a}$ is a mathmatical form of  % 
the Newton's Second Law.
 
 
 
  
\vspace{0.2in}
  
{\textbf{\Large{QUESTION
34.6 
 (          4,          4,          4)
}}}
  
  
 
 
\noindent\vspace{0.05in}{\textbf{\Large{Auto-answer:}}}
  
  
\begin{tabular}{|l|l|l|}
 \hline
 Column Left & Column Right  & Answers       \\ 
 \hline
{\textbf{\large{
A.}}}
C
  & 
YJH
 & 
{\textbf{\large{
E.}}}
 \\ 
 \hline
{\textbf{\large{
B.}}}
A
  & 
a
 & 
{\textbf{\large{
B.}}}
 \\ 
 \hline
{\textbf{\large{
C.}}}
B
  & 
c
 & 
{\textbf{\large{
A.}}}
 \\ 
 \hline
{\textbf{\large{
D.}}}
asdf(:)
  & 
ASDF(:)
 & 
{\textbf{\large{
D.}}}
 \\ 
 \hline
{\textbf{\large{
E.}}}
yjh
  & 
b
 & 
{\textbf{\large{
C.}}}
 \\ 
 \hline
 \end{tabular}
  
  
 
 
 
 
   
   
\vspace{0.3in}
{\textbf{\LARGE{You have done all the above? Excellent! Not much left, please continue.}}}
\vspace{0.3in}
   
   
  
\vspace{0.2in}
  
{\textbf{\Large{QUESTION
34.7 
 (          8,         15,         60)
}}}
  
  
 
 
\noindent\vspace{0.05in}{\textbf{\Large{Answer:}}}

 
$\left( \begin{array}{ccccccccccccccc}
           5 & 
           5 & 
           4 & 
           6 \\ 
           6 & 
           4 & 
           7 & 
           5 \\ 
           7 & 
           7 & 
           7 & 
           7
\end{array}\right) \times
\left( \begin{array}{c}
           2 \\ 
           2 \\ 
           2 \\ 
           2
\end{array}\right)  =
\left( \begin{array}{c}
          40 \\ 
          44 \\ 
          56
\end{array}\right)  $
 
$  % 
 \left( \begin{array}
 {
 c
 c
 }
                    \zeta & 
 \varepsilon \\ 
 \gamma & 
 \Gamma \\ 
 \Theta & 
 \varepsilon \\ 
 \gamma & 
                    \zeta
 \end{array} \right)
 \left( \begin{array}
 {
 c
 }
 \beta \\ 
 \beta
 \end{array} \right)
=
  \left( \begin{array}
 {
 c
 }
                    \zeta \times  \beta   +  \varepsilon \times  \beta \\ 
 \gamma \times  \beta   +  \Gamma \times  \beta \\ 
 \Theta \times  \beta   +  \varepsilon \times  \beta \\ 
 \gamma \times  \beta   +                     \zeta \times  \beta
 \end{array} \right)
$
 
 
 
 
 
\noindent\vspace{0.1in}{\textbf{\Large{Solution: }}}

 
 
  
\vspace{0.2in}
  
{\textbf{\Large{QUESTION
34.8 
 (          7,         14,         50)
}}}
  
  
 
 
\noindent\vspace{0.05in}{\textbf{\Large{Auto-answer:}}}
 
 
\noindent{\textbf{\large{
C.}}}
  The accelaration is $  %
(
1.67,
.17,
-74.074)
ms^{-2} $.
 
 
 
 
 
 
\noindent\vspace{0.1in}{\textbf{\Large{Solution: }}}

We will use the Newton's Second Law:
 
\[
\mathbf{f}=m\mathbf{a}.
\]
 
Since $\mathbf{f}= % 
(90.0 , 9.0 , -4000.0) N$
and $m= % 
54.0kg$, bring them into the above equation, then we get
 
\begin{eqnarray*}
\mathbf{a}&=&\frac{\mathbf{f}}m  \\
&=&\frac{ % 
(90.0 , 9.0 , -4000.0) N}{ % 
54.0kg}  \\
&=& % 
(1.67 , .17 , -74.074) ms^{-2}
\end{eqnarray*}
 
 
 
  
\vspace{0.2in}
  
{\textbf{\Large{QUESTION
34.9 
 (          9,         16,         70)
}}}
  
  


 
 
\noindent\vspace{0.05in}{\textbf{\Large{Answer:}}}

21,  % 
20
 
 
 
 
 
\noindent\vspace{0.1in}{\textbf{\Large{Solution: }}}

Roots to the equation
\begin{eqnarray*}
-5 \times x^2  % 
+  % 
205
                 \times x    % 
-2100 =0
\end{eqnarray*}
are  % 
21 and  % 
20 .
 
Let us verity  % 
21 first:
$  % 
-5 \times x^2  % 
+  % 
205
                 \times x    % 
-2100
  = % 
-2205+( % 
4305)+( % 
-2100)
  = % 
2100+( % 
-2100)
  = % 
0
$
 
Then verity  % 
20:
$  % 
-5 \times x^2  % 
+  % 
205
                 \times x    % 
-2100
  = % 
-2000+( % 
4100)+( % 
-2100)
  = % 
2100+( % 
-2100)
  = % 
0
$
 
 
 
   
   
 \vspace{0.2in}
Here are still some constants for use:
 
 
\noindent\begin{tabular}{|l|l|l|}
\hline
Constant & Symbol & Value \\
\hline
 
Mass of proton &
$m_p$ &
 $ 1.6726231 \times 10^{-27} $
kg \\
\hline
 
Boltzmann's constant &
$k$ &
 $ 1.381 \times 10^{-23} $
J/K \\
\hline
 
\end{tabular}
 
Thank you very much for answering these questions!
 
{\textbf{\large{Please be advised}}} that in this paper there are questions from
34.1 through
34.9.
And any one of them may contain more than one sub-question, thus the total number
of sub-questions here is around 14, of which
13 should be answered.
 
   
   
   
   
\vspace{1.0in} 
{\textbf{\large{ *** END OF PAPER, THANKS *** }}} 
   
   
\hspace{1.0in} By: 
         239(         26,          34)
   
   
   
\vspace{0.2in}
\vspace{0.2in}
   
   
 \newpage
\setcounter{page}{1} 
   
   
 {\LARGE{STATISTICS}}
   
\vspace{0.2in}
   
 \begin{tabular}{|l|l|}
 \hline
 Initial seed for random numbers &        239 \\
\hline
 First paper number &         26 \\
\hline
 Last  paper number &         34 \\
\hline
 Total papers to be generated &          9 \\
\hline
Total marks from input file & 100.00 \\
\hline
Total actual marks & 100.00 \\
\hline
 Total lines of the input file &        915 \\
 \hline
 Total QUESTIONs in input file &         16 \\
\hline
 Total CHOOSEs in input file &          1 \\
\hline
 Total NOTEs in input file &          2 \\
\hline
 Total (big) questions in each paper &          9 \\
\hline
 Total actual (sub)questions in each paper &         14 \\
\hline
 Total (sub)questions to be answered in each paper &         13 \\
\hline
 \end{tabular}
   
   
 \newpage
   
{\LARGE{For each big question}}
   
   
\vspace{0.2in}
   
   
\noindent\hspace{-0.4in}\begin{tabular}{|l|l|l|l|l|}
\hline
 Big question & Choose? & Questions needed & Questions from & Question IDs \\ 
\hline
           1(          4,3.13
 ) &  No   & 
           1(          1,           1) &           1(          1
,3.13
 ,10.00) &           1 \\
 \hline
           2(          4,1.56
 ) &  No   & 
           1(          1,           1) &           2(          0
,1.56
 ,5.00) &           2 \\
 \hline
           3(          4,1.56
 ) &  No   & 
           1(          1,           1) &           3(          1
,1.56
 ,5.00) &           3 \\
 \hline
           4(          4,3.13
 ) &  No   & 
           1(          1,           1) &           4(          0
,3.13
 ,10.00) &           4 \\
 \hline
           5(          4,1.56
 ) &  No   & 
           1(          1,           1) &           5(          0
,1.56
 ,5.00) &           5 \\
 \hline
           6(          2,62.50
,40.00
 ) &           1 & 
           6(          5,           8) &           6(          0
,12.50
 ,5.00) &          21 \\
  & & &           7(          0
 ,12.50
 ,5.00) &          22 \\
  & & &           8(          0
 ,12.50
 ,6.00) &          23 \\
  & & &           9(          0
 ,12.50
 ,8.00) &          24 \\
  & & &          10(          1
 ,12.50
 ,5.70) &          25 \\
  & & &          11(          0
 ,12.50
 ,12.40) &          26 \\
  & & &          12(          0
 ,12.50
 ,24.50) &          27 \\
 \hline
 \end{tabular}
   
   
 \vspace{0.2in}
   
   
\noindent\hspace{-0.4in}\begin{tabular}{|l|l|l|l|l|}
\hline
 Big question & Choose? & Questions needed & Questions from & Question IDs \\ 
\hline
  & & &          13(          0
 ,12.50
 ,67.20) &          28 \\
 \hline
           7(          8,12.50
 ) &  No   & 
           1(          1,           1) &          14(          1
,12.50
 ,40.00) &          50 \\
 \hline
           8(          8,12.50
 ) &  No   & 
           1(          1,           1) &          15(          0
,12.50
 ,40.00) &          60 \\
 \hline
           9(         14,1.56
 ) &  No   & 
           1(          1,           1) &          16(          0
,1.56
 ,5.00) &          70 \\
 \hline
 \end{tabular}
 
 
\end{document}
